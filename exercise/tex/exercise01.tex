%%%%%%%%%%%%%%%%%%%%%%%%%%%%%%%%%%%%%%%%%%%%%%%%%%%%%%%%%%%%%
%% Begin exercise %%
%%%%%%%%%%%%%%%%%%%%%%%%%%%%%%%%%%%%%%%%%%%%%%%%%%%%%%%%%%%%%
\ex{Step-down converter and power loss calculation}


%%%%%%%%%%%%%%%%%%%%%%%%%%%%%%%%%%%%%%%%%%%%%%%%%%%%%%%%%%%%%
%% Task 1: Step-down converter without output filter %%
%%%%%%%%%%%%%%%%%%%%%%%%%%%%%%%%%%%%%%%%%%%%%%%%%%%%%%%%%%%%%

\task{Step-down converter without output filter}

A switching transistor is used for low-loss and stepless control of a car's rear window heating.
transistor is used. By varying the duty cycle of the transistor, the average value of the heating power can be
of the heating power can be adjusted. The voltage in the car's electrical system is constant and
can be simulated with a voltage source $U_1 = \SI{14}{\volt}$. The heater is dimensioned in such a way
that at its nominal voltage $ U_{\mathrm{2N}} = \SI{14}{\volt}$ it absorbs a power of $ P_{\mathrm{LN}} = \SI{500}{\watt}$ and
can be simulated with an ohmic resistor.

\begin{center}
    \begin{circuitikz}[european currents,european resistors,american inductors]
        \draw
        (0,0) coordinate(N1) to [short,-o] ++(1,0) coordinate(U1p)
        ++(2,0) node[nigfete,rotate=90](Trans){}
        (U1p) to [short,i=$i_1$] (Trans.drain)
        (Trans.source) to [short,-o,i=$i_2$] ++(1,0) coordinate(U2p)
        to [short,-] ++(1,0) to [R,l^=$R_\text{L}$] ++(0,-3) to [short,-o] ++(-1,0) coordinate(U2n) to [short,-o] (1,-3) coordinate(U1n) to [short,-] ++(-1,0) coordinate(N10)
         (N1) to [V] (N10)
         (U1p) to [open,v^=$U_1$] (U1n)
         (U2p) to [open,v_=$u_2$] (U2n)
         (Trans.gate) to [short,o-] ++(0,-.3) to [sqV] ++(0,-1) 
         (Trans.gate) ++(-0.3,0) node(T1){$\text{T}_\text{1}$}
         
        ;
    \end{circuitikz}
\end{center}

%\begin{enumerate}
	\subtask{Draw qualitative current and voltage curves on the load resistor.}
        \begin{solutionblock}
            The graph below shows the voltages and currents at the load resistor with the period $T_{\mathrm{s}}$, the switch-on time $T_{\mathrm{on}}$ and the switch-off time $T_{\mathrm{off}}$. 
\begin{center}
      \begin{tikzpicture}
        \begin{axis}[
            domain=0:15,
            xmin=0, xmax=15,
            ymin=-1, ymax=2.5,
            samples=500,
            axis y line=center,
            axis x line=middle,
            xtick distance=1,
            ytick distance=2,
            extra y ticks=0,
            x label style={at={(axis description cs:1,0.25)},anchor=north},
            y label style={at={(axis description cs:0,.95)},anchor=south},
            width=0.8\textwidth,
            height=0.25\textwidth,
            xlabel={$t$},
            ylabel={{\color{blue}$u_\text{L}(t)$}\quad{\color{red}$i_\text{L}(t)$}},
            xtick={0},
            xticklabels={$0$},
            ytick={0,1,2},
            yticklabels={0,{\color{red}$\frac{U_1}{R_\text{L}}$},{\color{blue}$U_1$}},
            grid=none,
            grid style={line width=.1pt, draw=gray!10},
            major grid style={line width=.2pt,draw=gray!50},    ]
            \addplot[color=red,mark=none,solid] coordinates{
                (0, 0)
                (1, 0)
                (1, 1)
                (3, 1)
                (3, 0)
                (6, 0)
                (6, 1)
                (8, 1)
                (8, 0)
                (11, 0)
                (11, 1)
                (13, 1)
                (13, 0)
                (15, 0)
            };
            \addplot[color=blue,mark=none,dashed] coordinates{
                (0, 0)
                (1, 0)
                (1, 2)
                (3, 2)
                (3, 0)
                (6, 0)
                (6, 2)
                (8, 2)
                (8, 0)
                (11, 0)
                (11, 2)
                (13, 2)
                (13, 0)
                (15, 0)
            };
        \draw[>=triangle 45, <->] (axis cs:1,0.7) -- (axis cs:3,0.7);
        \node[anchor=north] at (axis cs:2,0.7){$T_\text{on}$};
        \draw[>=triangle 45, <->] (axis cs:3,0.7) -- (axis cs:6,0.7);
        \node[anchor=north] at (axis cs:4.5,0.7){$T_\text{off}$};
        \draw[>=triangle 45, <->] (axis cs:1,-0.3) -- (axis cs:6,-0.3);
        \node[anchor=north] at (axis cs:3.5,-0.3){$T_\text{s}$};
        \end{axis}
    \end{tikzpicture}
    % 
    %  \psfrag{u_L(t)}{$\textcolor{blue}{\mi uL (t)}$}
    %  	\psfrag{i_L(t)}{$\textcolor{red}{\mi iL (t)}$}
    %  	\psfrag{U1}{$\textcolor{blue}{U_1}$}
    %  	\psfrag{U1/R_L}[r][r]{$\textcolor{red}{\frac{U_1}{\mi RL}}$}
    %   	\includegraphics{bilder/loesung1_1.eps}
     \end{center}
         The duty cycle D is equal to the switch on time $T_{\mathrm{on}}$ divided by the switching cycle $T_{\mathrm{s}}$.
     $$D= \frac{T_\text{on}}{T_\text{s}}, \quad 1-D = \frac{T_\text{off}}{T_\text{s}} \Rightarrow T_\text{on} = D \cdot T_\text{s}, \quad T_\text{off} = (1-D) T_\text{s}$$
\end {solutionblock}

\subtask{ Draw the time curve of the instantaneous power at the load resistor.}


\begin {solutionblock}
The following graph shows the time curve of the instantaneous power at the load resistor.
\begin{center}
    \begin{tikzpicture}
       \begin{axis}[
           domain=0:15,
           xmin=0, xmax=15,
           ymin=-1, ymax=2.5,
           samples=500,
           axis y line=center,
           axis x line=middle,
           xtick distance=1,
           ytick distance=2,
           extra y ticks=0,
           x label style={at={(axis description cs:1,0.25)},anchor=north},
           y label style={at={(axis description cs:-.05,.97)},anchor=south},
           width=0.8\textwidth,
           height=0.25\textwidth,
           xlabel={$t$},
           ylabel={{\color{blue}$p(t)$}},
           xtick={0},
           xticklabels={$0$},
           ytick={0,2},
           yticklabels={0,{\color{blue}$\frac{U_1^2}{R}$}},
           grid=none,
           grid style={line width=.1pt, draw=gray!10},
           major grid style={line width=.2pt,draw=gray!50},    ]
           \addplot[color=blue,mark=none,solid] coordinates{
               (0, 0)
               (1, 0)
               (1, 2)
               (3, 2)
               (3, 0)
               (6, 0)
               (6, 2)
               (8, 2)
               (8, 0)
               (11, 0)
               (11, 2)
               (13, 2)
               (13, 0)
               (15, 0)
           };
       \draw[>=triangle 45, <->] (axis cs:1,0.7) -- (axis cs:3,0.7);
       \node[anchor=north] at (axis cs:2,0.7){$T_\text{on}$};
       \draw[>=triangle 45, <->] (axis cs:3,0.7) -- (axis cs:6,0.7);
       \node[anchor=north] at (axis cs:4.5,0.7){$T_\text{off}$};
       \draw[>=triangle 45, <->] (axis cs:1,-0.3) -- (axis cs:6,-0.3);
       \node[anchor=north] at (axis cs:3.5,-0.3){$T_\text{s}$};
       \end{axis}
   \end{tikzpicture}
   % 
   %  	\psfrag{p(t)}{$\textcolor{blue}{p(t)}$}
   %  	\psfrag{U1^2/R}[r][r]{$\textcolor{blue}{\frac {U_1^2}{R}}$}
   %   	\includegraphics{bilder/loesung1_2.eps}
    \end{center}

\end{solutionblock}

\subtask{ Derive the relationships for the mean voltage $\overline u_2$, the mean current $\overline i_2$ and the mean power}

\begin{solutionblock}
    The relationship between the average voltage $\overline u_2$, the average current $\overline i_2$ and the average power $P_{\mathrm{2}}$ as a function of the duty cycle D is derived below. 
    $$\overline u_2 = \frac{1}{T_{\mathrm{s}}} \int_0^{ T_{\mathrm{s}}} u_2 (t) dt = \frac{1}{ T_{\mathrm{s}}} \int_0^{D T_{\mathrm{s}}} U_1 dt + \frac{1}{ T_{\mathrm{s}}} \int_{D  T_{\mathrm{s}}}^{ T_{\mathrm{s}}} 0 dt$$
 $$= \left . \frac{U_1}{ T_{\mathrm{s}}} \cdot t \right |_0^{D  T_{\mathrm{s}}}  + 0 = \frac{U_1 D  T_{\mathrm{s}}}{ T_{\mathrm{s}}} = U_1 D$$
 $$\overline i_2 = \frac{1}{ T_{\mathrm{s}}} \int_0^{ T_{\mathrm{s}}} i_2(t) dt = \frac{1}{ T_{\mathrm{s}}}\int_0^{D  T_{\mathrm{s}}} \frac{U_1}{ R_{\mathrm{L}}} dt + \frac{1}{ T_{\mathrm{s}}}\int_{D  T_{\mathrm{s}}}^{ T_{\mathrm{s}}} 0 dt$$
 $$= \left . \frac{U_1}{ R_{\mathrm{L}}  T_{\mathrm{a}}} \cdot t \right |_0^{D  T_{\mathrm{s}}} = \frac{U_1 D  T_{\mathrm{s}}}{ R_{\mathrm{L}}  T_{\mathrm{s}}} = \frac{U_1}{ R_{\mathrm{L}}} \cdot D$$
 $$P_2 = \frac{1}{ T_{\mathrm{s}}} \int_0^{ T_{\mathrm{s}}} p(t) dt = \frac{1}{ T_{\mathrm{s}}} \int_0^{D T_{\mathrm{s}}s} \frac{U_1^2}{ R_{\mathrm{L}}} dt = \frac{U_1^2 D  T_{\mathrm{s}}}{ T_{\mathrm{s}} R_{\mathrm{L}}} = \frac{U_1^2 D}{ R_{\mathrm{L}}}$$
 $$P_2 \neq \overline u_2 \cdot \overline i_2 = U_1 D \cdot \frac{U_1}{ R_{\mathrm{L}}} \cdot D = \frac{U_1^2 D T_{\mathrm{s}}}{ T_{\mathrm{s}}  R_{\mathrm{L}}} = \frac{U_1^2 D^2}{ _{\mathrm{L}}} !$$
 Averages were used, not effective values!
\end{solutionblock}
	% power $P_2$ as a function of the duty cycle $D$.


\subtask{ How large should the duty cycle D be selected so that an average voltage of $\overline u_2 = \SI{8}{\volt}$ is applied to the heater? What is the mean value of the current $\overline i_2$? What power $P_2$ is converted into
converted into heat?}
\begin {solutionblock}
Duty cycle to obtain an average voltage of 8V.
    $$\overline u_2 = D \cdot U_1 \Leftrightarrow D = \frac{\overline u_2}{U_1} = \frac{8\ \si{V}}{14 \ \si{V}}=0.57$$
    
   $$\overline i_2 = \frac{U_1}{ R_{\mathrm{L}}} \cdot D$$
 $$ R_{\mathrm{L}}? \quad  P_{\mathrm{LN}} = \frac{{ U_{\mathrm{2N}}}^2}{ R_{\mathrm{L}}} 
 \Leftrightarrow  R_{\mathrm{L}} = \frac{ U_{\mathrm{2N}}^2}{ P_{\mathrm{LN}}} = \frac{(14\ \si{V})^2}{500\ \si{W}} = \SI{392}{\mohm}$$%
 Average value of the current.
 $$\Rightarrow \overline i_2 = \frac{14\ \si{V}}{392\ \si{\mohm}} \cdot 0.57 =  \SI{20.36}{\ampere}$$
 Power converted into heat.
 $$P_2 = \frac{U_1^2}{ R_{\mathrm{L}}} \cdot D = \frac{14\ \si{V}^2}{392\ \si{\mohm}} \cdot 0.57 = \SI{285}{\watt}$$
\end{solutionblock}
	
\subtask{ When starting the engine, the heater may draw a maximum average current ${\overline i}{an} = \SI{10}{\ampere}$ from the
draw from the vehicle electrical system. With which duty cycle $D$ should the transistor be switched in this case?
What is the average voltage $\overline u_2$ at the heater? What power $P_2$ is converted into heat?}

%\frac{8\ \si{V}}{14 \ \si{V}}=0.57$$

\begin{solutionblock}
Duty cycle transistor.
$$\overline i_1 = \overline i_2 = \frac{U_1}{ R_{\mathrm{L}}} \cdot D \overset ! \leq {\overline i_{\mathrm{an}}}$$
 $$\Leftrightarrow D \leq \frac{ {\overline i_{\mathrm{an}}} \cdot R_{\mathrm{L}}}{U_1} = \frac{10 \ \si{A} \cdot 392 \si{\mohm}}{14\ \si{V}} = 0.28$$
 Medium voltage at the heater.
 $$\overline u_2 = D \cdot u_1 = \SI{3.92}{\volt}$$
 Power converted into heat.
 $$P_2 = \frac{U_1^2 D}{ R_{\mathrm{L}}} = \frac{(14\ \si{V})^2 \cdot 0.28}{392\ \si{\mohm}}= \SI{140}{\watt}$$    
\end{solutionblock}

\subtask{During the journey, the heat output should be $ P_{\mathrm{2f}} = \SI{200} {\watt}$. How is the duty cycle
	 $D$ set? What are the mean values of the current $\overline i_2$ and the voltage $\overline u_2$?}
     \begin{solutionblock}
  Duty cycle D.
$$ P_{\mathrm{2f}} \overset ! = \frac{U_1^2}{ R_{\mathrm{L}}} \cdot D$$
 $$\Leftrightarrow D = \frac{ R_{\mathrm{L}} \cdot P_{\mathrm{2f}}}{U_1^2} = \frac{{392\ \si{\mohm}} \cdot 200 \ \si{W}}{(14\ \si{V})^2} = 0,4$$
 Average values of current and voltage.
 $$\overline i_2 = \frac{U_1}{ R_{\mathrm{L}}} \cdot D = \SI{14.29}{\ampere}$$
 $$\overline u_2 = U_1 \cdot D = \SI{5.6}{\volt}$$
\end{solutionblock}
%%%%%%%%%%%%%%%%%%%%%%%%%%%%%%%%%%%%%%%%%%%%%%%%%%%%%%%%%%%%%
%% Task 2: Step-down converter with output filter %%
%%%%%%%%%%%%%%%%%%%%%%%%%%%%%%%%%%%%%%%%%%%%%%%%%%%%%%%%%%%%%

\task{Step-down converter with output filter}

%%%%%%%%%%%%%%%%%%%%%%%%%%%%%%%%%%%%%%%%%%%%%%%%%%%%%%%%%%%%%
%% Task 3: Power losses within the step-down converter %%
%%%%%%%%%%%%%%%%%%%%%%%%%%%%%%%%%%%%%%%%%%%%%%%%%%%%%%%%%%%%%

\task{Power losses within the step-down converter}