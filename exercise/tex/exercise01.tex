%%%%%%%%%%%%%%%%%%%%%%%%%%%%%%%%%%%%%%%%%%%%%%%%%%%%%%%%%%%%%
%% Begin exercise %%
%%%%%%%%%%%%%%%%%%%%%%%%%%%%%%%%%%%%%%%%%%%%%%%%%%%%%%%%%%%%%
\ex{Step-down converter and power loss calculation}


%%%%%%%%%%%%%%%%%%%%%%%%%%%%%%%%%%%%%%%%%%%%%%%%%%%%%%%%%%%%%
%% Task 1: Step-down converter without output filter %%
%%%%%%%%%%%%%%%%%%%%%%%%%%%%%%%%%%%%%%%%%%%%%%%%%%%%%%%%%%%%%

\task{Step-down converter without output filter}

A switching transistor is used for low-loss and stepless control of a car's rear window heating.
transistor is used. By varying the duty cycle of the transistor, the average value of the heating power can be
of the heating power can be adjusted. The voltage in the car's electrical system is constant and
can be simulated with a voltage source $U_1 = \SI{14}{\volt}$. The heater is dimensioned in such a way
that at its nominal voltage $ U_{\mathrm{2N}} = \SI{14}{\volt}$ it absorbs a power of $ P_{\mathrm{LN}} = \SI{500}{\watt}$ and
can be simulated with an ohmic resistor.

\begin{center}
    \begin{circuitikz}[european currents,european resistors,american inductors]
        \draw
        (0,0) coordinate(N1) to [short,-o] ++(1,0) coordinate(U1p)
        ++(2,0) node[nigfete,rotate=90](Trans){}
        (U1p) to [short,i=$i_1$] (Trans.drain)
        (Trans.source) to [short,-o,i=$i_2$] ++(1,0) coordinate(U2p)
        to [short,-] ++(1,0) to [R,l^=$R_\text{L}$] ++(0,-3) to [short,-o] ++(-1,0) coordinate(U2n) to [short,-o] (1,-3) coordinate(U1n) to [short,-] ++(-1,0) coordinate(N10)
         (N1) to [V] (N10)
         (U1p) to [open,v^=$U_1$] (U1n)
         (U2p) to [open,v_=$u_2$] (U2n)
         (Trans.gate) to [short,o-] ++(0,-.3) to [sqV] ++(0,-1) 
         (Trans.gate) ++(-0.3,0) node(T1){$\text{T}_\text{1}$}
         
        ;
    \end{circuitikz}
\end{center}

%\begin{enumerate}
	\subtask{Draw qualitative current and voltage curves on the load resistor.}
        \begin{solutionblock}
            The graph below shows the voltages and currents at the load resistor with the period $T_{\mathrm{s}}$, the switch-on time $T_{\mathrm{on}}$ and the switch-off time $T_{\mathrm{off}}$. 
\begin{center}
      \begin{tikzpicture}
        \begin{axis}[
            domain=0:15,
            xmin=0, xmax=15,
            ymin=-1, ymax=2.5,
            samples=500,
            axis y line=center,
            axis x line=middle,
            xtick distance=1,
            ytick distance=2,
            extra y ticks=0,
            x label style={at={(axis description cs:1,0.25)},anchor=north},
            y label style={at={(axis description cs:0,.95)},anchor=south},
            width=0.8\textwidth,
            height=0.25\textwidth,
            xlabel={$t$},
            ylabel={{\color{blue}$u_\text{L}(t)$}\quad{\color{red}$i_\text{L}(t)$}},
            xtick={0},
            xticklabels={$0$},
            ytick={0,1,2},
            yticklabels={0,{\color{red}$\frac{U_1}{R_\text{L}}$},{\color{blue}$U_1$}},
            grid=none,
            grid style={line width=.1pt, draw=gray!10},
            major grid style={line width=.2pt,draw=gray!50},    ]
            \addplot[color=red,mark=none,solid] coordinates{
                (0, 0)
                (1, 0)
                (1, 1)
                (3, 1)
                (3, 0)
                (6, 0)
                (6, 1)
                (8, 1)
                (8, 0)
                (11, 0)
                (11, 1)
                (13, 1)
                (13, 0)
                (15, 0)
            };
            \addplot[color=blue,mark=none,dashed] coordinates{
                (0, 0)
                (1, 0)
                (1, 2)
                (3, 2)
                (3, 0)
                (6, 0)
                (6, 2)
                (8, 2)
                (8, 0)
                (11, 0)
                (11, 2)
                (13, 2)
                (13, 0)
                (15, 0)
            };
        \draw[>=triangle 45, <->] (axis cs:1,0.7) -- (axis cs:3,0.7);
        \node[anchor=north] at (axis cs:2,0.7){$T_\text{on}$};
        \draw[>=triangle 45, <->] (axis cs:3,0.7) -- (axis cs:6,0.7);
        \node[anchor=north] at (axis cs:4.5,0.7){$T_\text{off}$};
        \draw[>=triangle 45, <->] (axis cs:1,-0.3) -- (axis cs:6,-0.3);
        \node[anchor=north] at (axis cs:3.5,-0.3){$T_\text{s}$};
        \end{axis}
    \end{tikzpicture}
    % 
    %  \psfrag{u_L(t)}{$\textcolor{blue}{\mi uL (t)}$}
    %  	\psfrag{i_L(t)}{$\textcolor{red}{\mi iL (t)}$}
    %  	\psfrag{U1}{$\textcolor{blue}{U_1}$}
    %  	\psfrag{U1/R_L}[r][r]{$\textcolor{red}{\frac{U_1}{\mi RL}}$}
    %   	\includegraphics{bilder/loesung1_1.eps}
     \end{center}
         The duty cycle D is equal to the switch on time $T_{\mathrm{on}}$ divided by the switching cycle $T_{\mathrm{s}}$.
     $$D= \frac{T_\text{on}}{T_\text{s}}, \quad 1-D = \frac{T_\text{off}}{T_\text{s}} \Rightarrow T_\text{on} = D \cdot T_\text{s}, \quad T_\text{off} = (1-D) T_\text{s}$$
\end {solutionblock}

\subtask{ Draw the time curve of the instantaneous power at the load resistor.}


\begin {solutionblock}
The following graph shows the time curve of the instantaneous power at the load resistor.
\begin{center}
    \begin{tikzpicture}
       \begin{axis}[
           domain=0:15,
           xmin=0, xmax=15,
           ymin=-1, ymax=2.5,
           samples=500,
           axis y line=center,
           axis x line=middle,
           xtick distance=1,
           ytick distance=2,
           extra y ticks=0,
           x label style={at={(axis description cs:1,0.25)},anchor=north},
           y label style={at={(axis description cs:-.05,.97)},anchor=south},
           width=0.8\textwidth,
           height=0.25\textwidth,
           xlabel={$t$},
           ylabel={{\color{blue}$p(t)$}},
           xtick={0},
           xticklabels={$0$},
           ytick={0,2},
           yticklabels={0,{\color{blue}$\frac{U_1^2}{R}$}},
           grid=none,
           grid style={line width=.1pt, draw=gray!10},
           major grid style={line width=.2pt,draw=gray!50},    ]
           \addplot[color=blue,mark=none,solid] coordinates{
               (0, 0)
               (1, 0)
               (1, 2)
               (3, 2)
               (3, 0)
               (6, 0)
               (6, 2)
               (8, 2)
               (8, 0)
               (11, 0)
               (11, 2)
               (13, 2)
               (13, 0)
               (15, 0)
           };
       \draw[>=triangle 45, <->] (axis cs:1,0.7) -- (axis cs:3,0.7);
       \node[anchor=north] at (axis cs:2,0.7){$T_\text{on}$};
       \draw[>=triangle 45, <->] (axis cs:3,0.7) -- (axis cs:6,0.7);
       \node[anchor=north] at (axis cs:4.5,0.7){$T_\text{off}$};
       \draw[>=triangle 45, <->] (axis cs:1,-0.3) -- (axis cs:6,-0.3);
       \node[anchor=north] at (axis cs:3.5,-0.3){$T_\text{s}$};
       \end{axis}
   \end{tikzpicture}
   % 
   %  	\psfrag{p(t)}{$\textcolor{blue}{p(t)}$}
   %  	\psfrag{U1^2/R}[r][r]{$\textcolor{blue}{\frac {U_1^2}{R}}$}
   %   	\includegraphics{bilder/loesung1_2.eps}
    \end{center}

\end{solutionblock}

\subtask{ Derive the relationships for the mean voltage $\overline u_2$, the mean current $\overline i_2$ and the mean power}

\begin{solutionblock}
    The relationship between the average voltage $\overline u_2$, the average current $\overline i_2$ and the average power $P_{\mathrm{2}}$ as a function of the duty cycle D is derived below. 
    $$\overline u_2 = \frac{1}{T_{\mathrm{s}}} \int_0^{ T_{\mathrm{s}}} u_2 (t) dt = \frac{1}{ T_{\mathrm{s}}} \int_0^{D T_{\mathrm{s}}} U_1 dt + \frac{1}{ T_{\mathrm{s}}} \int_{D  T_{\mathrm{s}}}^{ T_{\mathrm{s}}} 0 dt$$
 $$= \left . \frac{U_1}{ T_{\mathrm{s}}} \cdot t \right |_0^{D  T_{\mathrm{s}}}  + 0 = \frac{U_1 D  T_{\mathrm{s}}}{ T_{\mathrm{s}}} = U_1 D$$
 $$\overline i_2 = \frac{1}{ T_{\mathrm{s}}} \int_0^{ T_{\mathrm{s}}} i_2(t) dt = \frac{1}{ T_{\mathrm{s}}}\int_0^{D  T_{\mathrm{s}}} \frac{U_1}{ R_{\mathrm{L}}} dt + \frac{1}{ T_{\mathrm{s}}}\int_{D  T_{\mathrm{s}}}^{ T_{\mathrm{s}}} 0 dt$$
 $$= \left . \frac{U_1}{ R_{\mathrm{L}}  T_{\mathrm{a}}} \cdot t \right |_0^{D  T_{\mathrm{s}}} = \frac{U_1 D  T_{\mathrm{s}}}{ R_{\mathrm{L}}  T_{\mathrm{s}}} = \frac{U_1}{ R_{\mathrm{L}}} \cdot D$$
 $$P_2 = \frac{1}{ T_{\mathrm{s}}} \int_0^{ T_{\mathrm{s}}} p(t) dt = \frac{1}{ T_{\mathrm{s}}} \int_0^{D T_{\mathrm{s}}s} \frac{U_1^2}{ R_{\mathrm{L}}} dt = \frac{U_1^2 D  T_{\mathrm{s}}}{ T_{\mathrm{s}} R_{\mathrm{L}}} = \frac{U_1^2 D}{ R_{\mathrm{L}}}$$
 $$P_2 \neq \overline u_2 \cdot \overline i_2 = U_1 D \cdot \frac{U_1}{ R_{\mathrm{L}}} \cdot D = \frac{U_1^2 D T_{\mathrm{s}}}{ T_{\mathrm{s}}  R_{\mathrm{L}}} = \frac{U_1^2 D^2}{ _{\mathrm{L}}} !$$
 Averages were used, not effective values!
\end{solutionblock}
	% power $P_2$ as a function of the duty cycle $D$.


\subtask{ How large should the duty cycle D be selected so that an average voltage of $\overline u_2 = \SI{8}{\volt}$ is applied to the heater? What is the mean value of the current $\overline i_2$? What power $P_2$ is converted into
converted into heat?}
\begin {solutionblock}
Duty cycle to obtain an average voltage of 8V.
    $$\overline u_2 = D \cdot U_1 \Leftrightarrow D = \frac{\overline u_2}{U_1} = \frac{8\ \si{V}}{14 \ \si{V}}=0.57$$
    
   $$\overline i_2 = \frac{U_1}{ R_{\mathrm{L}}} \cdot D$$
 $$ R_{\mathrm{L}}? \quad  P_{\mathrm{LN}} = \frac{{ U_{\mathrm{2N}}}^2}{ R_{\mathrm{L}}} 
 \Leftrightarrow  R_{\mathrm{L}} = \frac{ U_{\mathrm{2N}}^2}{ P_{\mathrm{LN}}} = \frac{(14\ \si{V})^2}{500\ \si{W}} = \SI{392}{\mohm}$$%
 Average value of the current.
 $$\Rightarrow \overline i_2 = \frac{14\ \si{V}}{392\ \si{\mohm}} \cdot 0.57 =  \SI{20.36}{\ampere}$$
 Power converted into heat.
 $$P_2 = \frac{U_1^2}{ R_{\mathrm{L}}} \cdot D = \frac{14\ \si{V}^2}{392\ \si{\mohm}} \cdot 0.57 = \SI{285}{\watt}$$
\end{solutionblock}
	
\subtask{ When starting the engine, the heater may draw a maximum average current ${\overline i}{an} = \SI{10}{\ampere}$ from the
draw from the vehicle electrical system. With which duty cycle $D$ should the transistor be switched in this case?
What is the average voltage $\overline u_2$ at the heater? What power $P_2$ is converted into heat?}

%\frac{8\ \si{V}}{14 \ \si{V}}=0.57$$

\begin{solutionblock}
Duty cycle transistor.
$$\overline i_1 = \overline i_2 = \frac{U_1}{ R_{\mathrm{L}}} \cdot D \overset ! \leq {\overline i_{\mathrm{an}}}$$
 $$\Leftrightarrow D \leq \frac{ {\overline i_{\mathrm{an}}} \cdot R_{\mathrm{L}}}{U_1} = \frac{10 \ \si{A} \cdot 392 \si{\mohm}}{14\ \si{V}} = 0.28$$
 Medium voltage at the heater.
 $$\overline u_2 = D \cdot u_1 = \SI{3.92}{\volt}$$
 Power converted into heat.
 $$P_2 = \frac{U_1^2 D}{ R_{\mathrm{L}}} = \frac{(14\ \si{V})^2 \cdot 0.28}{392\ \si{\mohm}}= \SI{140}{\watt}$$    
\end{solutionblock}

\subtask{During the journey, the heat output should be $ P_{\mathrm{2f}} = \SI{200} {\watt}$. How is the duty cycle
	 $D$ set? What are the mean values of the current $\overline i_2$ and the voltage $\overline u_2$?}
     \begin{solutionblock}
  Duty cycle D.
$$ P_{\mathrm{2f}} \overset ! = \frac{U_1^2}{ R_{\mathrm{L}}} \cdot D$$
 $$\Leftrightarrow D = \frac{ R_{\mathrm{L}} \cdot P_{\mathrm{2f}}}{U_1^2} = \frac{{392\ \si{\mohm}} \cdot 200 \ \si{W}}{(14\ \si{V})^2} = 0,4$$
 Average values of current and voltage.
 $$\overline i_2 = \frac{U_1}{ R_{\mathrm{L}}} \cdot D = \SI{14.29}{\ampere}$$
 $$\overline u_2 = U_1 \cdot D = \SI{5.6}{\volt}$$
 
\end{solutionblock}

%%%%%%%%%%%%%%%%%%%%%%%%%%%%%%%%%%%%%%%%%%%%%%%%%%%%%%%%%%%%%
%% Task 2: Step-down converter with output filter %%
%%%%%%%%%%%%%%%%%%%%%%%%%%%%%%%%%%%%%%%%%%%%%%%%%%%%%%%%%%%%%

\task{Step-down converter with output filter}
A step-down converter is used to charge a mobile phone from the vehicle electrical system with the vehicle electrical system voltage $U_1 = \SI{13.5}{\volt}$. The input voltage of the mobile phone is $U_2 = \SI{4.5}{\volt}$.
\begin{center}
\begin{circuitikz}[european currents,european resistors,american inductors]
	\draw
	(0,0) coordinate(U+) to [short,-] ++(0.5,0)  
	node[currarrow](Tl){} -- ++(1,0) ++(0.5,0) node[nigfete,rotate=90](Trans){} -- ++(1.5,0) coordinate(Tr)  to [short,-*] ++(0.5,0) coordinate(junc1)   -- ++(0.5,0) coordinate(Ll) to [L,l_=$L$,i^=$i_\text{L}$] ++(2,0) coordinate(Lr) to [short,-*] ++(0.5,0) coordinate(junc2)  -- ++(2,0)  coordinate(Rt) to [R,l_=$R$,i_=$i_\text{2}$,v^=$U_\text{2}$,voltage shift=0.5] ++(0,-3) coordinate(Rb) to [short,-*] ++(-2,0) coordinate(junc3) to [short,-*] ++(-3,0) coordinate(junc4) to [short] ++(-2,0) coordinate(junc5) to [short,-] ++(-2,0) coordinate(U-)
        (junc2) to [C,l_=$C$](junc3)
        (junc4) to [short]  ++(0,0.5) coordinate(Db) to[D-,l^=$D$]  ++(0,2) coordinate(Dt) to [short] (junc1)
         (Trans.G)  to [sqV] ++(0,-1)(junc5) 
	
	(U+) to [V=$U_1$] (U-)
	
  	(Trans)  node[anchor=south,color=black]{$T_1$}	
 	(Tl)  node[anchor=south,color=black]{$i_\text{1}$}	
	;
\end{circuitikz}
\end{center}

(both voltages are assumed to be constant). The inductance of the ideal coil is $L = \SI{10}{\milli\henry}$
The switching frequency is $f_s = \SI{ 100}{\kilo\hertz}$. All components are ideal.

\subtask{Draw the equivalent circuits for the two switching states.}
% Solution of subtask
\begin{solutionblock}
	\begin{tabular}{cc}
        	Transistor leitet & Diode leitet\\
       	\begin{circuitikz}[european currents,european resistors,american inductors,]
	        \draw
            (0,0) coordinate(u1o)
            to [V,v^=$U_2$] ++(0,-3) coordinate(u1u)
                (u1o) to [short,-,i^<=$i_2$] ++(-1,0) to [L,l_=$L$,v^<=$u_\text{L}$,mirror] ++(-3,0) to [short,-,i^<=$i_1$] ++(-1,0) coordinate(uqo)
                (u1u) to [short,-] ++(-5,0) coordinate(uqu)
                (uqo) to [V,v_=$U_1$] (uqu)
                ;
        \end{circuitikz}
&
      	\begin{circuitikz}[european currents,european resistors,american inductors,]
            \draw
            (0,0) coordinate(u2o)
            to [V,v^=$U_2$] ++(0,-3) coordinate(u2u)
                (u2o) to [short,-,i^<=$i_2$] ++(-1,0) to [L,l_=$L$,v^<=$u_\text{L}$,mirror] ++(-3,0) coordinate(Ll) ++(-1,0) coordinate(uqo)
                (u2u) to [short,-*] ++(-4,0) coordinate(Llu) to [short,-] ++(-1,0) coordinate(uqu)
                (Llu) to [short,-,i_=$i_1$] (Ll)
                (uqo) to [V,v_=$U_1$] (uqu)
                ;
        \end{circuitikz}
% \\
%         	\psfrag {U1}{$U_1$}
% 		\psfrag {U2}{$U_2$}
% 		\psfrag {i1}{$i_1$}
% 		\psfrag {i2}{$i_2$}
% 		\psfrag {uL}{$u_L$}
% 		\psfrag {L}{$L$}
%         	\includegraphics{bilder/loesung2_1a} 
%         	& 
%         	\psfrag {U1}{$U_1$}
% 		\psfrag {U2}{$U_2$}
% 		\psfrag {i1}{$i_1$}
% 		\psfrag {i2}{$i_2$}
% 		\psfrag {uL}{$u_L$}
% 		\psfrag {L}{$L$}
%         	\includegraphics{bilder/loesung2_1b}
	\end{tabular}

\end{solutionblock}

\subtask{At what duty cycle $D$ should the buck converter be operated?}
% Solution of subtask

\begin{solutionblock}
    \begin{equation}
        D = \frac{U_2}{U_1} = \frac{\SI{ 4.5}{\volt}}{\SI{ 13.5}{\volt}} = \frac{1}{3}
    \end{equation}

\end{solutionblock}
   
\subtask{Qualitatively draw the voltage and current waveforms in the components.}
% Solution of subtask

\begin{solutionblock}    
    \begin{center}
        \definecolor{orange}{rgb}{1.0,0.5,0}
        \begin{tikzpicture}
            \begin{groupplot}[group style={%
                group size=1 by 5,
            %     group name=plots,
                xlabels at=edge bottom,
                y descriptions at=edge left,
                },
            %     ybar,
                xmin=0, xmax=13.5,
                domain=0:13,
            %     enlarge x limits={abs=.5},
                width=0.6\textwidth,
            %     height=0.6\textwidth,
            %     scaled y ticks=base 10:-3,
            %     xticklabels from table={\first}{Criterion},
            %     x tick label style={rotate=90,anchor=east},
            %     xtick=data,
            %         x label style={at={(axis description cs:1,0)},anchor=north},
            %         xlabel={$t$},
                    xtick={0,1,4,5,8,9,12,13},
                    xticklabels={{},{},{},{},{},{},{},{},{},{}},
                    grid=both,
                    grid style={line width=.1pt, draw=gray!10},
                    major grid style={line width=.2pt,draw=gray!50},    
                    axis y line=center,
                    axis x line=middle,
                    ]
            
            %     ]
            
            
            \nextgroupplot[
            %         domain=0:13,
            %         xmin=0, xmax=13.5,
                    ymin=0, ymax=1.2,
                    samples=500,
            %         axis y line=center,
            %         axis x line=middle,
            %         xtick distance=1,
                    ytick distance=2,
                    extra y ticks=0,
                    x label style={at={(axis description cs:1,0)},anchor=north},
                    y label style={at={(axis description cs:-.05,.97)},anchor=south},
            %         width=0.6\textwidth,
                    height=0.2\textwidth,
                    xlabel={$t$},
                    ylabel={{\color{orange}$u_\text{T}$}},
            %         xtick={0,1,4,5,8,9,12,13},
            %         xticklabels={$0$,{},{},{},{},{},{},{},{},{}},
                    ytick={0,1},
                    yticklabels={0,$U_1$},
            %         grid=both,
            %         grid style={line width=.1pt, draw=gray!10},
            %         major grid style={line width=.2pt,draw=gray!50},    
            ]
                    \addplot[color=orange,mark=none,solid] coordinates{
                        (0, 0)
                        (1, 0)
                        (1, 1)
                        (4, 1)
                        (4, 0)
                        (5, 0)
                        (5, 1)
                        (8, 1)
                        (8, 0)
                        (9, 0)
                        (9, 1)
                        (12,1)
                        (12,0)
                        (13, 0)
                    };  
                    
            \nextgroupplot[
            %         domain=0:13,
            %         xmin=0, xmax=13.5,
                    ymin=0, ymax=1.2,
                    samples=500,
                    axis y line=center,
            %         axis x line=middle,
            %         xtick distance=1,
                    ytick distance=2,
                    extra y ticks=0,
                    x label style={at={(axis description cs:1,0)},anchor=north},
                    y label style={at={(axis description cs:-.05,.97)},anchor=south},
            %         width=0.6\textwidth,
                    height=0.2\textwidth,
                    xlabel={$t$},
                    ylabel={{\color{blue}$u_\text{D}$}},
            %         xtick={0,1,4,5,8,9,12,13},
            %         xticklabels={$0$,{},{},{},{},{},{},{},{},{}},
                    ytick={0,.33,1},
                    yticklabels={0,$U_2$,$U_1$},
            %         grid=both,
            %         grid style={line width=.1pt, draw=gray!10},
            %         major grid style={line width=.2pt,draw=gray!50},    
            ]
                    \addplot[color=blue,mark=none,solid] coordinates{
                        (0, 1)
                        (1, 1)
                        (1, 0)
                        (4, 0)
                        (4, 1)
                        (5, 1)
                        (5, 0)
                        (8, 0)
                        (8, 1)
                        (9, 1)
                        (9, 0)
                        (12,0)
                        (12,1)
                        (13, 1)
                        (13, 0)
                    };
                
            \nextgroupplot[
            %         domain=0:13,
            %         xmin=0, xmax=13.5,
                    ymin=-.5, ymax=1.2,
                    samples=500,
                    axis y line=center,
            %         axis x line=middle,
            %         xtick distance=1,
                    ytick distance=2,
                    extra y ticks=0,
                    x label style={at={(axis description cs:1,0.25)},anchor=north},
                    y label style={at={(axis description cs:-.05,.97)},anchor=south},
            %         width=0.6\textwidth,
                    height=0.25\textwidth,
                    xlabel={$t$},
                    ylabel={{\color{orange}$u_\text{L}$}},
            %         xtick={0,1,4,5,8,9,12,13},
            %         xticklabels={$0$,{},{},{},{},{},{},{},{},{}},
                    ytick={-.333,0,1},
                    yticklabels={$-U_2$,0,$U_1-U_2$},
            %         grid=both,
            %         grid style={line width=.1pt, draw=gray!10},
            %         major grid style={line width=.2pt,draw=gray!50},    
            ]
                    \draw [fill=green] (axis cs:4,0) rectangle (axis cs:5,1);
                    \draw [fill=cyan] (axis cs:5,0) rectangle (axis cs:8,-.3333);
                    \addplot[color=orange,mark=none,solid] coordinates{
                        (0, 1)
                        (1, 1)
                        (1, -.333)
                        (4, -.333)
                        (4, 1)
                        (5, 1)
                        (5, -.333)
                        (8, -.333)
                        (8, 1)
                        (9, 1)
                        (9, -.333)
                        (12, -.333)
                        (12, 1)
                        (13, 1)
                        (13, -.333)
                    };
                \node[anchor=center] at (axis cs:4.5,0.5){\tiny $ A_\text{ein}$};
                \node[anchor=center] at (axis cs:6.5,-0.1667){\tiny $ A_\text{aus}$};
            
            \nextgroupplot[
            %         domain=0:13,
            %         xmin=0, xmax=13.5,
                    ymin=0, ymax=1.2,
                    samples=500,
                    axis y line=center,
            %         axis x line=middle,
            %         xtick distance=1,
                    ytick distance=2,
                    extra y ticks=0,
                    x label style={at={(axis description cs:1,0)},anchor=north},
                    y label style={at={(axis description cs:-.05,.97)},anchor=south},
            %         width=0.6\textwidth,
                    height=0.2\textwidth,
                    xlabel={$t$},
                    ylabel={{\color{red}$i_\text{L}$}},
            %         xtick={0,1,4,5,8,9,12,13},
            %         xticklabels={$0$,{},{},{},{},{},{},{},{},{}},
                    ytick={0,.5,.75,1},
                    yticklabels={0,,,},
            %         grid=both,
            %         grid style={line width=.1pt, draw=gray!10},
            %         major grid style={line width=.2pt,draw=gray!50},    
            ]
                    \addplot[color=red,mark=none,solid] coordinates{
                        (0, 0.5)
                        (1, 1)
                        (4, 0.5)
                        (5, 1)
                        (8, 0.5)
                        (9, 1)
                        (12,0.5)
                        (13,1)
                    };
            
            \nextgroupplot[
            %         domain=0:13,
            %         xmin=0, xmax=13.5,
                    ymin=0, ymax=1.2,
                    samples=500,
                    axis y line=center,
            %         axis x line=middle,
            %         xtick distance=1,
                    ytick distance=2,
                    extra y ticks=0,
                    x label style={at={(axis description cs:1,0)},anchor=north},
                    y label style={at={(axis description cs:-.0,.97)},anchor=south},
            %         width=0.6\textwidth,
                    height=0.2\textwidth,
                    xlabel={$t$},
                    ylabel={{\color{orange}$i_\text{T}$}\quad{\color{blue}$i_\text{D}$}},
            %         xtick={0,1,4,5,8,9,12,13},
            %         xticklabels={$0$,{},{},{},{},{},{},{},{},{}},
                    ytick={0,.5,.75,1},
                    yticklabels={0,,,},
            %         grid=both,
            %         grid style={line width=.1pt, draw=gray!10},
            %         major grid style={line width=.2pt,draw=gray!50},    
            ]
                    \addplot[color=orange,mark=none,solid] coordinates{
                        (0,0)
                        (0, 0.5)
                        (1, 1)
                        (1, 0)
                        (4,0)
                        (4,0.5)
                        (5, 1)
                        (5,0)
                        (8,0)
                        (8, 0.5)
                        (9, 1)
                        (9,0)
                        (12,0)
                        (12,0.5)
                        (13,1)
                        (13,0)
                    };
                    \addplot[color=blue,mark=none,solid] coordinates{
                        (1,1)
                        (4, 0.5)
                    };
                    \addplot[color=blue,mark=none,solid] coordinates{
                        (5,1)
                        (8, 0.5)
                    };
                    \addplot[color=blue,mark=none,solid] coordinates{
                        (9,1)
                        (12, 0.5)
                    };
                    \addplot[color=blue,mark=none,solid] coordinates{
                        (0,0)
                        (1, 0)
                    };
                    \addplot[color=blue,mark=none,solid] coordinates{
                        (4,0)
                        (5,0)
                    };
                    \addplot[color=blue,mark=none,solid] coordinates{
                        (8,0)
                        (9,0)
                    };
                    \addplot[color=blue,mark=none,solid] coordinates{
                        (12,0)
                        (13,0)
                    };
                    \addplot[color=blue,mark=none,dashed] coordinates{
                        (1,0)
                        (1,1)
                    };
                    \addplot[color=blue,mark=none,dashed] coordinates{
                        (4,0)
                        (4,0.5)
                    };
                    \addplot[color=blue,mark=none,dashed] coordinates{
                        (5,0)
                        (5,1)
                    };
                    \addplot[color=blue,mark=none,dashed] coordinates{
                        (8,0)
                        (8,0.5)
                    };
                    \addplot[color=blue,mark=none,dashed] coordinates{
                        (9,0)
                        (9,1)
                    };
                    \addplot[color=blue,mark=none,dashed] coordinates{
                        (12,0)
                        (12,0.5)
                    };
                    \addplot[color=blue,mark=none,dashed] coordinates{
                        (13,0)
                        (13,1)
                    };
            
            \end{groupplot}
        \end{tikzpicture}
        
    
    % 		\psfrag{U1}{$U_1$}
    % 		\psfrag{U2}{$U_2$}
    %  		\psfrag{UT}{$\textcolor{orange}{\mi UT}$}
    % 		\psfrag{UL}{$\textcolor{orange}{\mi UL}$}
    % 		\psfrag{UD}{$\textcolor{blue}{\mi UD}$}
    % 		\psfrag{IT}{$\textcolor{orange}{\mi IT}$}
    % 		\psfrag{IL}{$\textcolor{red}{\mi IL}$}
    % 		\psfrag{ID}{$\textcolor{blue}{\mi ID}$}
    % 		\psfrag{Aein}{$\mi A{ein}$}
    % 		\psfrag{Aaus}{$\mi A{aus}$}
    % 		\includegraphics[width =8cm]{bilder/verlaeufe_aufg_2.eps}
    \end{center}
    
    
\end{solutionblock}

\subtask{How large is the current fluctuation range $\triangle i_L$ of the coil current in normal operation at a switching frequency of $f_s = \SI{ 100}{\kilo\hertz}$.}
% Solution of subtask
\begin{solutionblock}
    \begin{center}
        \begin{tikzpicture}
        \begin{axis}[
                domain=0:15,
                xmin=0, xmax=7,
                ymin=-1.5, ymax=2.5,
                samples=500,
                axis y line=center,
                axis x line=middle,
                xtick distance=1,
                ytick distance=2,
                extra y ticks=0,
                x label style={at={(axis description cs:1,0.33)},anchor=north},
                y label style={at={(axis description cs:-.05,.97)},anchor=south},
                width=0.8\textwidth,
                height=0.5\textwidth,
                xlabel={$t$},
                ylabel={$i_\text{L}$},
                xtick={0,2,5},
                xticklabels={$0$,{},{}},
                ytick={0,1,2},
                yticklabels={0,$\frac{\Delta i_\text{L}}{2}$,$\Delta i_\text{L}$},
                grid=both,
                grid style={line width=.1pt, draw=gray!10},
                major grid style={line width=.2pt,draw=gray!50},    ]
                \addplot[color=blue,mark=none,solid] coordinates{
                    (0, 0)
                    (2, 2)
                    (5, 0)
                    (7, 2)
                };
                \draw[>=triangle 45, <->] (axis cs:0,-.3) -- (axis cs:2,-.3);
                \node[anchor=north] at (axis cs:1,-.3){$D T_\text{s}$};
                \draw[>=triangle 45, <->] (axis cs:2,-.3) -- (axis cs:5,-.3);
                \node[anchor=north] at (axis cs:3.5,-.3){$(1-D) T_\text{s}$};
                \draw[>=triangle 45, <->] (axis cs:0,-1) -- (axis cs:5,-1);
                \node[anchor=north] at (axis cs:2.5,-1){$T_\text{s}$};
            \end{axis}
    \end{tikzpicture}
    % 
    % 		\psfrag{iL}{$\mi iL$}
    % 		\psfrag{delta iL}{$\triangle \mi iL$}
    % 		\psfrag{I2=delta iL/2}{$I_2 = \frac{\triangle \mi iL}{2}$}
    % 		\psfrag{DTs}{$D \mi Ts$}
    % 		\psfrag{(1-D)Ts}{$(1-D) \mi Ts$}
    % 		\psfrag{Ts}{$\mi Ts$}
    % 	 	\includegraphics{bilder/loesung2_4.eps}
    \end{center}
    \begin{equation}
        f_s = \SI{ 100}{\kilo\hertz} \Rightarrow T_s = \SI{ 10}{\micro\second} \Rightarrow T_e = D \cdot T_s = \frac{\SI{ 10}{\micro \second}}{3}
    \end{equation}
    \begin{equation}
        U_L = U_1 - U_2 = L \frac{d i_L}{dt} \overset {\begin{matrix}\text{Assumption of ideal coil}\\ \Rightarrow \text{linear current flow} \end{matrix}} = L \frac{\triangle i_L}{\triangle t} = L \frac{\triangle i_L}{T_e}
    \end{equation}
    \begin{equation}
        \Rightarrow \triangle i_L = \frac{U_1 - U_2}{L} T_e = \frac{\SI{13.5}{\volt} - \SI{4.5}{\volt}}{\SI{10}{\micro \henry} \cdot \frac{\SI{10}{\micro \second}}{3}} = \SI{3}{\ampere}
    \end{equation}
    \begin{equation}
        I_{2min} = \frac{\triangle i_L}{2} = \SI{1.5}{\ampere}
    \end{equation}
\end{solutionblock}

\subtask{When starting the engine, the input voltage drops to $U_{1min} = \SI{ 10}{\volt}$.
The voltage regulator of the buck converter changes the duty cycle so that the output voltage $U_2 = \SI{ 4.5}{\volt}$ is kept stable.
What duty cycle D is set?}
% Solution of subtask
\begin{solutionblock}
    \begin{equation}    
        D= \frac{U_2}{U_{1min}} = \frac{\SI{4.5}{\volt}}{\SI{10}{\volt}} = 0.45
    \end{equation}    
\end{solutionblock}

\subtask{Where is the boundary of discontinuous conduction mode?}
% Lösung des Unterpunktes
\begin{solutionblock}
    \begin{equation}
        \triangle i_L = \frac{U_{1min} - U_2}{L} D \cdot T_s = \frac{(\SI{10}{\volt} - \SI{4.5}{\volt}) \cdot 0.45 \cdot \SI{10}{\micro \second}}{\SI{10}{\micro \henry}} = \SI{2.475}{\ampere}
    \end{equation}
    \begin{equation}
        I_{2min} = \frac{\triangle i_L}{2} \approx \SI{1.24}{\ampere}
    \end{equation}
\end{solutionblock}

\subtask{In the next step, the input voltage is constant, and the output voltage is to be adjusted using the duty cycle.
         At what duty cycle $D$ will the peak-to-peak current ripple be maximal?}
% Lösung des Unterpunktes
\begin{solutionblock}
    \begin{equation}
        U_2 = D \cdot U_1
    \end{equation}
    \begin{equation}
        \triangle i_L (D) = \frac{U_1 - D \cdot U_1}{L} \cdot D \cdot T_s = \frac{U_1 T_s}{L} (D-D^2)
    \end{equation}
    \begin{equation}
        \triangle i_L' (D) \overset ! = 0 = \frac{U_1 T_s}{L}(1-2 D)
    \end{equation}
    \begin{equation}
        \Rightarrow 1 = 2 D \Rightarrow D = \frac{1}{2}
    \end{equation}
\end{solutionblock}
\subtask{Sketch the course of the current fluctuation width $\triangle i$ as a function of the duty cycle $D$.}
% Lösung des Unterpunktes
\begin{solutionblock}
    \begin{center}
        \begin{tikzpicture}
            \begin{axis}[
                domain=0:1,
                xmin=0, xmax=1.2,
                ymin=0, ymax=1.2,
                samples=500,
                axis y line=center,
                axis x line=middle,
                xtick distance=0.5,
                ytick distance=2,
                extra y ticks=0,
                x label style={at={(axis description cs:1,0)},anchor=north},
                y label style={at={(axis description cs:-.05,.95)},anchor=south},
                width=0.6\textwidth,
                height=0.4\textwidth,
                xlabel={$D$},
                ylabel={$\Delta i_\text{L}$},
        %          xtick={0, 1},
        %         xticklabels={$0$,$\tau$},
                ytick={0,1},
                yticklabels={0,$\frac{U_1 T_s}{4L}$},
                grid=both,
                grid style={line width=.1pt, draw=gray!10},
                major grid style={line width=.2pt,draw=gray!50},    ]
                \addplot+[color=blue,mark=none]{1-((2*(x-0.5))^2)};
            \end{axis}
        \end{tikzpicture}

    % 	\psfrag{delta iL}{$\triangle \mi iL$}
    % 	\psfrag{(U1Ts)/(4L)}{$\frac{U_1 \mi Ts}{4L}$}
    % 	\psfrag{0}{$0$}
    % 	\psfrag{0.5}{$\frac{1}{2}$}
    % 	\psfrag{1}{$1$}
    % 	\psfrag{D}{$D$}
    % 	 	\includegraphics[width = 7cm]{bilder/loesung2_8.eps}
    \end{center}
\end{solutionblock}

%%%%%%%%%%%%%%%%%%%%%%%%%%%%%%%%%%%%%%%%%%%%%%%%%%%%%%%%%%%%%
%% Task 3: Power losses within the step-down converter %%
%%%%%%%%%%%%%%%%%%%%%%%%%%%%%%%%%%%%%%%%%%%%%%%%%%%%%%%%%%%%%

\task{Power losses within the step-down converter}