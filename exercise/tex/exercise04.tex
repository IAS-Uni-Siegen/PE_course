%%%%%%%%%%%%%%%%%%%%%%%%%%%%%%%%%%%%%%%%%%%%%%%%%%%%%%%%%%%%%
%% Begin exercise %%
%%%%%%%%%%%%%%%%%%%%%%%%%%%%%%%%%%%%%%%%%%%%%%%%%%%%%%%%%%%%%

\ex{Flyback converter /}

%%%%%%%%%%%%%%%%%%%%%%%%%%%%%%%%%%%%%%%%%%%%%%%%%%%%%%%%%%%%%
%% Task 1: Flyback converter %%
%%%%%%%%%%%%%%%%%%%%%%%%%%%%%%%%%%%%%%%%%%%%%%%%%%%%%%%%%%%%%
\task{Flyback converter}
A flyback converter with an input voltage range $U_\mathrm{1} = \SI{300}{\volt} \, \dots \, \SI{900}{\volt}$ is used to supply the control electronics of a frequency inverter. The converter delivers a rated output power of  $P_\mathrm{2} = \SI{30}{\watt}$ at a regulated (constant) output voltage of  $U_\mathrm{2} = \SI{15}{\volt}$. The flyback converter is operated in discontinuous current mode with a constant frequency of  $f_\mathrm{p} = \SI{500}{\kilo\hertz}$. The transformation ratio of the transformer is $N_\mathrm{1}/N_\mathrm{2}=60/12$, the inductance of the primary winding is $L_\mathrm{1} = \SI{760}{\micro\henry}$. The coupling between the primary and secondary windings is ideal. You can assume stationary operation for all calculations.

%%%%%%%%%%%%%%%%%%%%%%%%%%%%%%%%%%%%%%%%%%%%%%%%%%%%%%%%%%%%%%%%%%%%%%%
 % Flyback converter Schematic
%%%%%%%%%%%%%%%%%%%%%%%%%%%%%%%%%%%%%%%%%%%%%%%%%%%%%%%%%%%%%%%%%%%%%%%
           
           \begin{figure}[htb]
                \begin{center}
                    \begin{circuitikz}[european currents,european resistors,american inductors]
                    \draw (0.5,0) to [short] ++(0.5,0)
                    to [diode, l=$D$]  ++(1.0,0)
                    to [short, -o, i=$i_2(t)$] ++(1.0,0)
                    to [open, o-o, v = $\hspace{2cm}u_2(t)$, voltage = straight] ++(0,-2) coordinate (A)
                    (-0.5,0) to [short, -o, i_<=$i_1(t)$] ++(-1.5,0)
                    to [open, o-o, v_= $u_1(t)\hspace{0.75cm}$, voltage = straight] ++(0,-3.75) coordinate (B)
                    (-0.5,0) to [inductor, n=l1] ++(0,-2) 
                    to [Tnpn, n=npn1, mirror] ++(0,-1.75) coordinate (C)
                    (0.5,0) to [inductor, n=l2, mirror] ++(0,-2) coordinate (D)
                    (D) to [short, -o] (A)
                    (C) to [short, -o] (B);
                    \draw let \p1 = (npn1.B) in node[anchor=south] at (\x1,\y1) {$T$};
                    \path (l1.ul dot) node[circ]{}
                        (l2.ur dot) node[circ]{};
                    \draw (l1.midtap) node[left]{$N_1$}
                    (l2.midtap) node[right]{$N_2$};
                    \draw[double, double distance=3pt, thick] let \p1=(l1.core west), \p2=(l2.core east) in (\x1/2+\x2/2, \y1) -- (\x1/2+\x2/2, \y2);
                \end{circuitikz}
            \end{center}
                \caption{Flyback converter topology}
                \label{fig:flyback_converter_topology}
            \end{figure}
        

\begin{table}[ht]
    \centering  % Zentriert die Tabelle
    \begin{tabular}{llll}
        \toprule
        
        Input voltage: &  $U_{\mathrm{1}} = \SI{300}{\volt} \, \dots \, \SI{900}{\volt}$ & Output voltage: & $U_{\mathrm{2}} = \SI{15}{\volt}$ \\ 
        Output power: & $P_2 = \SI{15}{\watt}$  & Transformation ratio: & $N_\mathrm{1}/N_\mathrm{2}=60/12$ \\ 
        Inductance of the primary winding: & $L_\mathrm{1} = \SI{760}{\micro\henry}$ & Switching frequency: & $f_{\mathrm{s}} = \SI{50}{\kilo\hertz}$ \\ 
        \bottomrule
    \end{tabular}
    \caption{Parameters of the boost converter.}  % Beschriftung der Tabelle
    \label{table:ex04_Parameters of the circuit}
\end{table}

\subtask{The input voltage is $U_\mathrm{1}=\SI{760}{\volt}$ at rated power at the output. What is the peak value $\hat I_\mathrm{1}$ of the primary current $i_\mathrm{1}$? What is the peak value $\hat I_\mathrm{2}$ of the primary current $i_\mathrm{2}$? Calculate the duty cycle of the transistor for this operating case.}

\subtask{The input voltage is  $U_\mathrm{1}=\SI{382}{\volt}$ at nominal load. Calculate and sketch the following voltage and current curves for this operating case over one cycle period: $u_\mathrm{T}(t), u_\mathrm{s}(t), i_\mathrm{2}(t), i_\mathrm{1}(t)$ (Note: corresponds to the switch-on time of the transistor).}

\subtask{The input voltage is  $U_\mathrm{1}=\SI{382}{\volt}$ at nominal load. Determine the mean value $\overline i_\mathrm{T}$ and the effective value of the current $i_\mathrm{T, rms}$ through the transistor. Determine the mean value $\overline i_\mathrm{D}$ and the effective value of the current $i_\mathrm{D, rms}$ through the diode. What is the maximum reverse voltage load $u_\mathrm{T, max}$ of the transistor? What is the maximum reverse voltage load $u_\mathrm{D, max}$ of the diode? Calculate the fluctuation range $\Delta i_\mathrm{C, pp}$ of the current $i_\mathrm{C}$ in the output capacitor.}

\subtask{The input voltage is  $U_\mathrm{1}=\SI{382}{\volt}$ at nominal load.How much energy is transferred from the input to the output per switching period $\Delta E$ and what is the resulting average power? What happens if there is no ideal voltage source on the output side but an unloaded capacitor and the circuit is operated with $D>0$?}