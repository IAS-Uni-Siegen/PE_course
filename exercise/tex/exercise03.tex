%%%%%%%%%%%%%%%%%%%%%%%%%%%%%%%%%%%%%%%%%%%%%%%%%%%%%%%%%%%%%
%% Begin exercise %%
%%%%%%%%%%%%%%%%%%%%%%%%%%%%%%%%%%%%%%%%%%%%%%%%%%%%%%%%%%%%%

\ex{Step-down converter and power loss calculation}

%%%%%%%%%%%%%%%%%%%%%%%%%%%%%%%%%%%%%%%%%%%%%%%%%%%%%%%%%%%%%
%% Task 1: Inverse converter %%
%%%%%%%%%%%%%%%%%%%%%%%%%%%%%%%%%%%%%%%%%%%%%%%%%%%%%%%%%%%%%
\task {Inverse converter}
An inverse converter (see Fig. 1) is used to generate the negative supply voltage of a control control electronics, an inverse converter (see Fig. 1) is used. The input voltage is specified as $U_\mathrm{1}=\SI{18}{\volt}$, the output voltage is regulated to $U_\mathrm{2}=\SI{12}{\volt}$. The output power can be values in the range $P_\mathrm{2}=\SI{2}{\watt}$ to $\SI{25}{\watt}$.

\begin{figure}[htb]
    \begin{center}
        \begin{circuitikz}[european currents,european resistors,american inductors]
            \draw 
                    % Base coordinates
                    (0,0) coordinate (jU1v)
                    (0,-3) coordinate (jU1g)
                    % Add components
                    % Add primary source U1
                    (jU1v) to [V,v_=$U_1$] (jU1g)
                    % Add current symbol
                    (jU1v) to [short,i^>=$i_\mathrm{1}$(t)] (1,0)
                    % Add T1 with Control
                    (jU1v) ++(2,0) node[nigfete,rotate=90](Trans1){} -- ++(2.5,0) coordinate(T1)
                    % At transistor label T1
                    (Trans1)  node[anchor=south,color=black]{$T_1$}                    
                    % Short horizontal line
                    (jU1v) to [short,-] (Trans1.D)
                    % Add junction point for inductor
                    (T1) to [short,-*] ++(0,0) coordinate (jLv)
                    % Add junction point for P-buck diode
                    (jLv) ++(0,-3) coordinate (jLg)
                    % Add inductance
                    (jLv) to  [L=$L$] (jLg)
                    % Add current IL symbol
                    (jLv)  to [short,i^>=$i_\mathrm{L}$(t)] (4.5,-0.7)
                    % Continue horizontal line
                    (jLv) to  [short,-] ++ (1.5,0) coordinate (jPtv)
                    % Continue vertical line
                    (jPtv) to  [short,-, crossing] ++ (0,-6) coordinate (jPtg)
                    % Add positive Diode connection point
                    (jPtg)  ++ (3,0) coordinate  (jDP)
                    % Add Diode
                    (jDP) to  [D-,l^=$D$] (jPtg)
                    % Add current Diode
                    (jPtg)  to [short,i^<=$i_\mathrm{D}$(t)] (7,-6)
                    % Add junction point for P-buck diode
                    (jDP) to [short,-*] ++(0,-0) 
                    % Add junction point for capacitor
                    (jDP)  ++(0,3) coordinate (jCv)                    
                    % Add line to secondary source U2
                    (jDP) to  [short,-] ++ (2,0) coordinate (jU2g)
                    % Add positive connection point
                    (jU2g) ++(0,3) coordinate (jU2v)                    
                    % Add secondary source U2
                    (jU2v)  to [V=$U_2$] (jU2g)
                    % Add current I2
                    (jCv)  to [short,i^>=$i_\mathrm{2}$(t)] (11,-3)
                    % Add line to capacitor
                    (jU2v)  to [short,-*] (jCv)
                    % Add capacitor
                    (jCv)  to [C=$C_2$] (jDP)
                    % Add current capacitor
                    (jCv)  to [short,i^>=$i_\mathrm{C}$(t)] (9,-4)
                    % Add horizontal from C to L
                    (jCv) to [short,-*] (jLg)

                    % Connect U1
                    (jLg) to [short,-] (jU1g)
           ;
        \end{circuitikz}
    \end{center}
    \caption{Inverting buck-boost converter.}
    \label{fig:Inverting buck-boost converter}
\end{figure}


\subtask{The system should operate at the limit of gap-free operation throughout the entire output power range. How should the inductance be selected so that the switching frequency is always above the hearing threshold $f_\mathrm{P}=\SI{20}{\kilo\hertz}$?}

\begin{solutionblock}
    \begin{equation}
        U_{\mathrm{1}} = u_{\mathrm{L}} = L \frac{\Delta i_{\mathrm{L}} }{\Delta t}. 
    \end{equation}
\begin{equation}
    D = \frac{U_\mathrm{2}}{U_\mathrm{1}+U_\mathrm{2}} = \frac{\SI{12}{\volt}}{\SI{18}{\volt}+\SI{12}{\volt}} = 0.4
\end{equation}
BCM is assumed
\begin{equation}
    I_\mathrm{2,min} = \frac{P_\mathrm{2}}{U_\mathrm{2}}=\frac{\SI{25}{\watt}}{\SI{12}{\volt}}=\SI{2.083}{\volt}
\end{equation}
The maximum current at BCM $\Delta i_{\mathrm{L}}$ is twice  $I_\mathrm{2,min}$.
\begin{equation}
    \Delta i_{\mathrm{L}} = 2  I_\mathrm{2,min} = \SI{4.16}{\ampere}
\end{equation}
\begin{equation}
    \Delta t = \frac{D}{f_\mathrm{P}}
\end{equation}
The inductance can therefore be determined using the following equation:
\begin{equation}
    L = \frac{\Delta t  u_{\mathrm{L}}}{\Delta i_{\mathrm{L}}}= \frac{D u_{\mathrm{L}}}{f_\mathrm{P}\Delta i_{\mathrm{L}}}= \frac{D u_{\mathrm{1}}}{f_\mathrm{P}\Delta i_{\mathrm{L}}} = \frac{0.4 \cdot \SI{18}{\volt}}{\SI{20}{\kilo\hertz}\cdot \SI{4.16}{\ampere }} = \SI{86.4}{\micro\henry}
\end{equation}

\end{solutionblock}
\subtask{In what range does the switching frequency $f_\mathrm{P}$ vary?}

 \begin{solutionblock}
     The inductance equation is used again for this task:
     \begin{equation}
        L = \frac{\Delta t  u_{\mathrm{L}}}{\Delta i_{\mathrm{L}}}= \frac{D u_{\mathrm{L}}}{f_\mathrm{P}\Delta i_{\mathrm{L}}}
     \end{equation}
     This equation can be converted to $f_\mathrm{P}$ and used to determine the frequency as follows:
     \begin{equation}
        f_\mathrm{P} = \frac{Du_{\mathrm{1}}}{L\Delta i_{\mathrm{L}}} \label{equation 1.8}
     \end{equation}
     As the output power is given as a value range, the highest and lowest values can be used. The lowest and highest current should be determined from these two values, from which the value range of the frequency can then be determined.
     \begin{equation}
        I_\mathrm{1,min}= \frac{P_\mathrm{2}}{U_\mathrm{1}}=\frac{\SI{2}{\watt}}{\SI{18}{\volt}}=\SI{0.1}{\volt}
     \end{equation}
     \begin{equation}
        I_\mathrm{1,max}= \frac{P_\mathrm{2}}{U_\mathrm{1}}=\frac{\SI{25}{\watt}}{\SI{18}{\volt}}=\SI{1.38}{\volt}
     \end{equation}
     Now we can use these two terms in \ref{equation 1.8}:
     \begin{equation}
        f_\mathrm{P,min}=\frac{0.4\cdot\SI{18}{\volt}}{\SI{86.4}{\micro\henry}\cdot 2\cdot \SI{0.1}{\volt}}=\SI{416.667}{\kilo \hertz}
     \end{equation}
     \begin{equation}
        f_\mathrm{P,max}=\frac{0.4\cdot\SI{18}{\volt}}{\SI{86.4}{\micro\henry}\cdot 2\cdot \SI{1.38}{\volt}}=\SI{30.193}{\kilo \hertz}
     \end{equation}
 \end{solutionblock}

 \subtask{What is the peak value $\hat I_1$ of the transistor current?}
 \begin{solutionblock}
    As the maximum current through the transistor is the current $I_1$, these values are the same. Since the BCM is considered in this circuit, the maximum current is the peak current, which corresponds to the value of $\SI{4.16}{\ampere}$.
 \end{solutionblock}

%%%%%%%%%%%%%%%%%%%%%%%%%%%%%%%%%%%%%%%%%%%%%%%%%%%%%%%%%%%%%
%% Task 2: Boost-Buck converter and SEPIC topology
%%%%%%%%%%%%%%%%%%%%%%%%%%%%%%%%%%%%%%%%%%%%%%%%%%%%%%%%%%%%%

\task {Boost-Buck converter and SEPIC topology}

The supply of a plasma coating system is realized by a boost converter and a buck converter (with common capacitance).
The converter is connected to a voltage $U_\mathrm{1}$ and provides a variable output voltage $U_\mathrm{2}$.
\vspace{2em}\par

\par
% Schematic of Buck Boost Converter cascade

\begin{figure}[htb]
    \begin{center}
        \begin{circuitikz}[european currents,european resistors,american inductors]
            \draw 
                    % Base coordinates
                    (0,0) coordinate (jU1v)
                    (0,-3) coordinate (jU1g)
                    % Add primary source U1
                    (jU1v) to [V=$U_1$] (jU1g)
                    % Add line to L1
                    (jU1v) to [short,-] ++ (0.5,0) coordinate (jL1v)
                    % Add Inductance L1
                    (jL1v) to  [L, l=$L_1$] ++ (3,0) coordinate (jT1v)
                    % Add arrow and Text
                    ++(-0.5,0) node[currarrow](IL1){}
                    (IL1)  node[anchor=south,color=black]{$i_\mathrm{L1}$}
                    % Add junction for T1
                    (jT1v) to [short,-*] ++(0,0)
                    % Add coordinate for Transistor
                    (jT1v) ++(0,-3) coordinate (jT1g)
                    % Add transistor T2
                    (jT1v) ++ (0,-1.5) node[nigfete](Trans1){}
                    % At transistor label T2
                    (Trans1)  node[anchor=west,color=black]{$T_1$}                     
                    % Connect Transistor
                    (jT1v) to [short,-] (Trans1.D)
                    (jT1g) to [short,-] (Trans1.S)
                    (Trans1.G) to [sqV] ++(-1,0)
                    % Add diode D1
                    (jT1v) to  [D,l=$D_1$] ++ (2,0) coordinate (jCv)
                    % Add junction for C
                    (jCv) to [short,-*] ++(0,0)
                    % Add coordinate jCv and jCg
                    (jCv)  ++(0,-3) coordinate (jCg)
                    % Add  capacitor C
                    (jCv)  to [C, l=$C$] (jCg)
                    % Add current symbol and T2 with Control
                    (jCv) ++(2,0) node[nigfete,rotate=90](Trans2){} -- ++(2.5,0) coordinate(T2)
                    (Trans2.G)  to [sqV] ++(0,-1)
                    % At transistor label T2
                    (Trans2)  node[anchor=south,color=black]{$T_2$}                    
                    % Short horizontal line
                    (jCv) to [short,-] (Trans2.D)
                    % Add junction point for M-buck diode
                    (T2) to [short,-*] ++(0,0) coordinate (jD2M)
                    % Add junction point for diode D2
                    (jD2M) ++(0,-3) coordinate (jD2P)
                    % Add diode D2
                    (jD2P) to  [D,l=$D_2$] (jD2M)                                      
                    % Add inductor L2
                    (jD2M) to  [L,l^=$L_2$] ++ (3,0) coordinate (jU2v)
                    % Add arrow and Text
                    (jU2v) ++(-0.5,0) node[currarrow](IL2){}
                    (IL2)  node[anchor=south,color=black]{$i_\mathrm{L2}$}
                    % Add coordinate jU2v and jU2g
                    (jU2v) to [short,-] ++(0,-3) coordinate (jU2g)
                    % Add secondary source U2
                    (jU2v) to [V=$U_2$] (jU2g)
                    % Add line and junction to D2
                    (jU2g) to [short,-*] (jD2P)
                    % Add line and junction to C
                    (jD2P) to [short,-*] (jCg)                    
                    % Add line and junction to T1
                    (jCg) to [short,-*] (jT1g)                    
                    % Add line and junction to U1
                    (jT1g) to [short,-] (jU1g)                    
           ;
        \end{circuitikz}
    \end{center}
    \caption{Buck-boost converter circuit.}
    \label{fig:ex02_step_down_with_load_resistor}
\end{figure}

\par
%Table of parameter values 
%%%%%%%%%%%%%%%%%%%%%%%%%%%%%%%%%%%%%%%%%%%%%%%%%%%%%%%%%%%%%%%%%%%%%%%%%%
%  Boost buck converter parameter
%%%%%%%%%%%%%%%%%%%%%%%%%%%%%%%%%%%%%%%%%%%%%%%%%%%%%%%%%%%%%%%%%%%%%%%%%%

\begin{table}[ht]
    \centering  % Zentriert die Tabelle
    \begin{tabular}{llll}
        \toprule
        
        Input voltage: &  $U_{\mathrm{1}} = \SI{380}{\volt}$ & Output voltage: & $U_{\mathrm{2}} = \SI{285}{\volt}$  to $\SI{450}{\volt}$\\ 
        Output power: & $P_{\mathrm{2}} = \SI{3000}{\watt}$ & Switching frequency: & $f_{\mathrm{s}} = \SI{50}{\kilo\hertz}$ \\ 
        \multicolumn{4}{l}{$P_{\mathrm{2}}$ is constant (unless otherwise stated)} \\
        \bottomrule
    \end{tabular}
    \caption{Parameters of the circuit.}  % Beschriftung der Tabelle
    \label{table:Parameters of the buck-boost converter.}
\end{table}

The output voltage is set to the specified value by adjusting the duty cycles $D_\mathrm{1}$ (of transistor $T_\mathrm{1}$) 
and $D_\mathrm{2}$ (of transistor $T_\mathrm{2}$) using a control system. Both transistors operate at the same switching frequency. 
The switching frequency fluctuation of the intermediate circuit voltage and the currents in the inductors can be ignored unless otherwise stated. 
The currents in $L_\mathrm{1}$ and $L_\mathrm{2}$ show a continuous course. Both transistors are operated with the 
same relative duty cycle $D_\mathrm{1}$ = $D_\mathrm{2}$ = $D$.

\subtask{Calculate the relative duty cycle $D$ of the transistors $T_\mathrm{1}$ and $T_\mathrm{1}$ to be set to achieve an output voltage $U_\mathrm{2}$.}

\subtask{Which intermediate circuit voltage U is determined depending on $U_\mathrm{2}$?}

\subtask{Plot $D$ and the voltage against $U_\mathrm{2}$ and calculate the numerical values for $U_\mathrm{2}$ = 285V, $U_\mathrm{2}$ =
380V and $U_\mathrm{2}$ = 450V.}

\subtask{What blocking voltage strength must the transistors $T_\mathrm{1}$ and $T_\mathrm{2}$ and the diodes DBoost and
DBuck have?}
\vspace{2em}\par
The control continues with $D_\mathrm{1}$ = $D_\mathrm{2}$ = $D$. The input and output inductances 
have the same value $L_\mathrm{1}$ = $L_\mathrm{2}$ = $L$ = $\SI{0.5}{m\henry}$.

\subtask{Write the formula for the dependence of the input and output inductance ripple $\Delta i_{L1}$ and $\Delta i_{L2}$ from the duty cycle}

\subtask{To what minimum value can the output power be reduced while still ensuring continuous operation across the entire output voltage range? 
(i.e. continuous current flow in $L_\mathrm{1}$  and L$L_\mathrm{2}$)?}

\subtask{In which section of the output voltage range is this limit reached first on the input side and in which section is it reached first on the output side?}
\vspace{2em}\par
The single ended primary inductance converter shall be considered under similar conditions.

% Schematic of Buck Boost Converter cascade

%%%%%%%%%%%%%%%%%%%%%%%%%%%%%%%%%%%%%%%%%%%%%%%%%%%%%%%%%%%%%%%%%%%%%%%%%%
%  Single Ended Primary Inductance Converter Schematic
%%%%%%%%%%%%%%%%%%%%%%%%%%%%%%%%%%%%%%%%%%%%%%%%%%%%%%%%%%%%%%%%%%%%%%%%%%

\begin{figure}[ht]
    \begin{center}
        \begin{circuitikz}[european currents,european resistors,american inductors]
            \draw 
                    % Base coordinates
                    (0,0) coordinate (jU1v)
                    (0,-3) coordinate (jU1g)
                    % Add primary source U1
                    (jU1v) to [V=$U_1$] (jU1g)
                    % Add line to L1
                    (jU1v) to [short,-] ++ (0.5,0) coordinate (jL1v)
                    % Add Inductance L1
                    (jL1v) to  [L, l=$L_1$] ++ (3,0) coordinate (jTv)
                    % Add arrow and Text
                    ++(-0.5,0) node[currarrow](IL){}
                    (IL)  node[anchor=south,color=black]{$i_\mathrm{L}$}
                    % Add junction for Transistor
                    (jTv) ++(0,-3) coordinate (jTg)
                    % Add junction point for Transistor
                    (jTv) to [short,-*] ++(0,0)                    
                    % Add transistor T2
                    (jTv) ++ (0,-1.5) node[nigfete](Trans1){}
                    % At transistor label T2
                    (Trans1)  node[anchor=west,color=black]{$T$}                     
                    % Connect Transistor
                    (jTv) to [short,-] (Trans1.D)
                    (jTg) to [short,-] (Trans1.S)
                    (Trans1.G) to [sqV] ++(-1,0)
                    % Add capacitor C
                    (jTv) to  [C,l=$C$] ++ (2,0) coordinate (jL2v)
                    % Add junction for L2
                    (jL2v) to [short,-*] ++(0,0)
                    % Add coordinate jL2v and jL2g
                    (jL2v)  ++(0,-3) coordinate (jL2g)
                    % Add  L2
                    (jL2v)  to [L, l=$L_2$] (jL2g)
                    % Add Diode
                    (jL2v) to  [D-,l^=$D$] ++ (2,0) coordinate (jU2v) 
                    % Add coordinate jU2v and jU2g
                    (jU2v) to [short,-] ++(0,-3) coordinate (jU2g)
                    % Add secondary source U2
                    (jU2v) to [V=$U_2$] (jU2g)
                    % Add Junction point for jL2g
                    (jU2g) to [short,-*] (jL2g)
                    % Add line and junction point for T1g
                    (jL2g) to [short,-*] (jTg)
                    % Add line to U1g
                    (jTg) to [short,-] (jU1g)
           ;
        \end{circuitikz}
    \end{center}
    \caption{Single ended  primary inductance converter circuit.}
    \label{fig:ex03_SEPIC}
\end{figure}


\subtask{What blocking voltage strength must the transistors $T_\mathrm{1}$ and the diode $D$ have?}

\subtask{Write the formula for the dependence of the input and output inductance ripple $\Delta i_{L1}$ and $\Delta i_{L2}$ from the duty cycle}

\subtask{To what minimum value can the output power be reduced while still ensuring continuous operation across the entire output voltage range? 
(i.e. continuous current flow in $L_\mathrm{1}$  and L$L_\mathrm{2}$)?}
