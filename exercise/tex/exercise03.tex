%%%%%%%%%%%%%%%%%%%%%%%%%%%%%%%%%%%%%%%%%%%%%%%%%%%%%%%%%%%%%
%% Begin exercise %%
%%%%%%%%%%%%%%%%%%%%%%%%%%%%%%%%%%%%%%%%%%%%%%%%%%%%%%%%%%%%%

\ex{Step-down converter and power loss calculation}

%%%%%%%%%%%%%%%%%%%%%%%%%%%%%%%%%%%%%%%%%%%%%%%%%%%%%%%%%%%%%
%% Task 1: Inverse converter %%
%%%%%%%%%%%%%%%%%%%%%%%%%%%%%%%%%%%%%%%%%%%%%%%%%%%%%%%%%%%%%
\task {Inverse converter}
An inverse converter (see Fig. 1) is used to generate the negative supply voltage of a control control electronics, an inverse converter (see Fig. 1) is used. The input voltage is specified as $U_\mathrm{1}=\SI{18}{\volt}$, the output voltage is regulated to $U_\mathrm{2}=\SI{12}{\volt}$. The output power can be values in the range $P_\mathrm{2}=\SI{2}{\watt}$ to $\SI{15}{\watt}$.

\begin{figure}[htb]
    \begin{center}
        \begin{circuitikz}[european currents,european resistors,american inductors]
            \draw 
                    % Base coordinates
                    (0,0) coordinate (jU1v)
                    (0,-3) coordinate (jU1g)
                    % Add components
                    % Add primary source U1
                    (jU1v) to [V,v_=$U_1$] (jU1g)
                    % Add current symbol
                    (jU1v) to [short,i^>=$i_\mathrm{1}$(t)] (1,0)
                    % Add T1 with Control
                    (jU1v) ++(2,0) node[nigfete,rotate=90](Trans1){} -- ++(2.5,0) coordinate(T1)
                    % At transistor label T1
                    (Trans1)  node[anchor=south,color=black]{$T_1$}                    
                    % Short horizontal line
                    (jU1v) to [short,-] (Trans1.D)
                    % Add junction point for inductor
                    (T1) to [short,-*] ++(0,0) coordinate (jLv)
                    % Add junction point for P-buck diode
                    (jLv) ++(0,-3) coordinate (jLg)
                    % Add inductance
                    (jLv) to  [L=$L$] (jLg)
                    % Add current IL symbol
                    (jLv)  to [short,i^>=$i_\mathrm{L}$(t)] (4.5,-0.7)
                    % Continue horizontal line
                    (jLv) to  [short,-] ++ (1.5,0) coordinate (jPtv)
                    % Continue vertical line
                    (jPtv) to  [short,-, crossing] ++ (0,-6) coordinate (jPtg)
                    % Add positive Diode connection point
                    (jPtg)  ++ (3,0) coordinate  (jDP)
                    % Add Diode
                    (jDP) to  [D-,l^=$D$] (jPtg)
                    % Add current Diode
                    (jPtg)  to [short,i^<=$i_\mathrm{D}$(t)] (7,-6)
                    % Add junction point for P-buck diode
                    (jDP) to [short,-*] ++(0,-0) 
                    % Add junction point for capacitor
                    (jDP)  ++(0,3) coordinate (jCv)                    
                    % Add line to secondary source U2
                    (jDP) to  [short,-] ++ (2,0) coordinate (jU2g)
                    % Add positive connection point
                    (jU2g) ++(0,3) coordinate (jU2v)                    
                    % Add secondary source U2
                    (jU2v)  to [V=$U_2$] (jU2g)
                    % Add current I2
                    (jCv)  to [short,i^>=$i_\mathrm{2}$(t)] (11,-3)
                    % Add line to capacitor
                    (jU2v)  to [short,-*] (jCv)
                    % Add capacitor
                    (jCv)  to [C=$C_2$] (jDP)
                    % Add current capacitor
                    (jCv)  to [short,i^>=$i_\mathrm{C}$(t)] (9,-4)
                    % Add horizontal from C to L
                    (jCv) to [short,-*] (jLg)

                    % Connect U1
                    (jLg) to [short,-] (jU1g)
           ;
        \end{circuitikz}
    \end{center}
    \caption{Inverting buck-boost converter.}
    \label{fig:Inverting buck-boost converter}
\end{figure}


\subtask{The system should operate at the limit of gap-free operation throughout the entire output power range. How should the inductance be selected so that the switching frequency is always above the hearing threshold $f_\mathrm{P}=\SI{20}{\kilo\hertz}$?}

\begin{solutionblock}
    \begin{equation}
        U_{\mathrm{1}} = u_{\mathrm{L}} = L \frac{\Delta i_{\mathrm{L}} }{\Delta t}. 
    \end{equation}
\begin{equation}
    D = \frac{U_\mathrm{2}}{U_\mathrm{1}+U_\mathrm{2}} = \frac{\SI{12}{\volt}}{\SI{18}{\volt}+\SI{12}{\volt}} = 0.4
\end{equation}
BCM is assumed. A current $I_\mathrm{2}$ is only present when the transistor is open, otherwise the diode blocks. This is taken into account in the following equation as the duty cycle $D = 0.6$ is used for $T_\mathrm{off}$. The equation thus looks as follows:
\begin{equation}
    I_\mathrm{2,min} = \frac{P_\mathrm{2}}{U_\mathrm{2}} \frac{1}{D}=\frac{\SI{15}{\watt}}{\SI{12}{\volt}}\frac{5}{3}=\SI{2.083}{\volt}
\end{equation}
The maximum current at BCM $\Delta i_{\mathrm{L}}$ is twice  $I_\mathrm{2,min}$.
\begin{equation}
    \Delta i_{\mathrm{L}} = 2  I_\mathrm{2,min} = \SI{4.16}{\ampere}
\end{equation}
\begin{equation}
    \Delta t = \frac{D}{f_\mathrm{P}}
\end{equation}
The inductance can therefore be determined using the following equation:
\begin{equation}
    L = \frac{\Delta t  u_{\mathrm{L}}}{\Delta i_{\mathrm{L}}}= \frac{D u_{\mathrm{L}}}{f_\mathrm{P}\Delta i_{\mathrm{L}}}= \frac{D u_{\mathrm{1}}}{f_\mathrm{P}\Delta i_{\mathrm{L}}} = \frac{0.4 \cdot \SI{18}{\volt}}{\SI{20}{\kilo\hertz}\cdot \SI{4.16}{\ampere }} = \SI{86.4}{\micro\henry}
\end{equation}

\end{solutionblock}
\subtask{In what range does the switching frequency $f_\mathrm{P}$ vary?}

 \begin{solutionblock}
     The inductance equation is used again for this task:
     \begin{equation}
        L = \frac{\Delta t  u_{\mathrm{L}}}{\Delta i_{\mathrm{L}}}= \frac{D u_{\mathrm{L}}}{f_\mathrm{P}\Delta i_{\mathrm{L}}}
     \end{equation}
     This equation can be converted to $f_\mathrm{P}$ and used to determine the frequency as follows:
     \begin{equation}
        f_\mathrm{P} = \frac{Du_{\mathrm{1}}}{L\Delta i_{\mathrm{L}}} \label{equation 1.8}
     \end{equation}
      
As the output power is specified as a value range, the highest and lowest values can be used. The lowest and highest current should be determined from these two values, from which the value range of the frequency $f_\mathrm{P}$ can then be determined. Here the inverse converter is considered on the output side and therefore a duty cycle $D = 0.6$ is used again. The following equations can be derived from this:

     \begin{equation}
        I_\mathrm{1,min}= \frac{P_\mathrm{2}}{U_\mathrm{2}}\frac{1}{D}=\frac{\SI{2}{\watt}}{\SI{12}{\volt}}\frac{5}{3}=\SI{0.278}{\volt}
     \end{equation}
     \begin{equation}
        I_\mathrm{1,max}= \frac{P_\mathrm{2}}{U_\mathrm{2}}=\frac{\SI{15}{\watt}}{\SI{12}{\volt}}=\SI{2.0833}{\volt}
     \end{equation}
     Now we can use these two terms in \ref{equation 1.8}:
     \begin{equation}
        f_\mathrm{P,min}=\frac{0.4\cdot\SI{18}{\volt}}{\SI{86.4}{\micro\henry}\cdot 2\cdot \SI{0.1}{\volt}}=\SI{416.667}{\kilo \hertz}
     \end{equation}
     \begin{equation}
        f_\mathrm{P,max}=\frac{0.4\cdot\SI{18}{\volt}}{\SI{86.4}{\micro\henry}\cdot 2\cdot \SI{1.38}{\volt}}=\SI{30.193}{\kilo \hertz}
     \end{equation}
 \end{solutionblock}

 \subtask{What is the peak value $\hat I_1$ of the transistor current?}
 \begin{solutionblock}
    As the maximum current through the transistor is the current $I_1$, these values are the same. Since the BCM is considered in this circuit, the maximum current is the peak current, which corresponds to the value of $\SI{4.16}{\ampere}$.
 \end{solutionblock}

 \subtask{How does the duty cycle $D$ change with the output power? Enter the duty cycle values and the absolute values of the transistor switch-on times $T_\mathrm{on} = D T_\mathrm{s}$ for minimum and maximum output power.}

 \subtask{Sketch the course of the current $i_\mathrm{L}$ in the inductance for minimum and maximum output power.}

 \subtask{At which operating point does the maximum switching frequency fluctuation of the output voltage $\Delta u_\mathrm{2pp}$ occurs (the load approximately draws a constant current)?} 

 \subtask{How high should the output capacitance be selected to ensure $Delta u_\mathrm{2pp} < 0.02 U_\mathrm{2}?$}

 \subtask{What is the maximum effective value $i_\mathrm{C,RMS}$ of the output capacitor current?}

 \subtask{Sketch the time curve of the voltage $u_\mathrm{T}$ at the power transistor and the current $i_\mathrm{D}$ in the output diode for $P_\mathrm{2}=\SI{2}{\watt}$. What is the maximum reverse voltage of the transistor?}

 %%%%%%%%%%%%%%%%%%%%%%%%%%%%%%%%%%%%%%%%%%%%%%%%%%%%%%%%%%%%%
%% Task 2: Boost-Buck converter and SEPIC topology
%%%%%%%%%%%%%%%%%%%%%%%%%%%%%%%%%%%%%%%%%%%%%%%%%%%%%%%%%%%%%

\task {Boost-Buck converter and SEPIC topology}

Versorgung einer Plasmabeschichtungsanlage mit variabler Spannung U2 ausgehend von einer
geregelten Spannung U1 über Kopplung eines Hochsetzstellers und eines Tiefsetzstellers (mit
gemeinsamer Kapazität).


Bild


Angaben:
T2
DBoost
T1
C
U
L2
DBuck
U2
U1 = 380V
U2 = 285V ... 450V
P2 = 3000W konstant (falls nicht anders angegeben)
Schaltfrequenz: fP = 50kHz


Die Ausgangsspannung wird durch entsprechende Einstellung der Tastverhältnisse D1 (von Transistor
T1) und D2 (von Transistor T2) mittels einer Regelung auf den vorgegebenen Wert eingestellt. Beide
Transistoren arbeiten mit gleicher Schaltfrequenz. Die schaltfrequente Schwankung der
Zwischenkreisspannung und der Ströme in den Induktivitäten können falls nicht anders
angegeben vernachlässigt werden. Die Ströme in L1 und L2 zeigen kontinuierlichen Verlauf.

Beide Transistoren werden mit gleicher relativer Einschaltdauer D1 = D2 = D betrieben.

\subtask{Berechnen Sie die zur Realisierung einer Ausgangsspannung U2 einzustellende relative
Einschaltdauer D der Transistoren T1 und T2.}

\subtask{Welche Zwischenkreisspannung U stellt sich in Abhängigkeit von U2 ein?}

\subtask{Stellen Sie D und U über U2 graphisch dar und geben Sie die Zahlenwerte für U2 = 285V, U2 =
380V und U2 = 450V an.}

\subtask{Welche Sperrspannungsfestigkeit müssen die Transistoren T1 und T2 und die Dioden DBoost und
DBuck aufweisen?}

Die Steuerung erfolge weiterhin mit D1 = D2 = D. Die ein- und ausgangsseitigen Induktivitäten
weisen gleichen Wert L1 = L2 = L = 0.5mH auf.

% \subtask{Berechnen Sie allgemein die Abhängigkeit des eingangs- und ausgangsseitigen Induktivitätsstromrippels $\Delta$iL1,pp(D) und $\Delta$i_{L2},pp(D) ausschliesslich vom Tastverhältnis D.}
\subtask {Berechnen Sie allgemein die Abhängigkeit des eingangs- und ausgangsseitigen Induktivitätsrippels $\Delta i_{L1}$, pp(D) und $\Delta i_{L2}$,pp(D)} 

\subtask{Wie weit darf die Ausgangleistung reduziert werden, um gerade noch kontinuierlichen Betrieb für
den gesamten Ausgangsspannungsbereich sicherzustellen (d.h. kontinuierlicher Stromverlauf in L1
und L2)?}

\subtask{In welchem Abschnitt des Ausgangsspannungsbereiches wird diese Grenze zuerst
eingangsseitig, in welchem zuerst ausgangsseitig erreicht?}