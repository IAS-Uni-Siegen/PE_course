%%%%%%%%%%%%%%%%%%%%%%%%%%%%%%%%%%%%%%%%%%%%%%%%%%%%%%%%%%%%%
%% Begin exercise %%
%%%%%%%%%%%%%%%%%%%%%%%%%%%%%%%%%%%%%%%%%%%%%%%%%%%%%%%%%%%%%
\ex{Rectifiers}

%%%%%%%%%%%%%%%%%%%%%%%%%%%%%%%%%%%%%%%%%%%%%%%%%%%%%%%%%%%%%
%% Task 1: M3C converter at a RL-load                     %%
%%%%%%%%%%%%%%%%%%%%%%%%%%%%%%%%%%%%%%%%%%%%%%%%%%%%%%%%%%%%%

\task{M3C converter at an RL-load}
A controlled three-pulse midpoint circuit feeds an ohmic-inductive load. The load inductance $L$ is large, such that a pure direct current $I_\mathrm{2}$ is taken from the converter. The load resistance is $R = \SI{5}{\Omega}$. The converter transformer 
is connected to the symmetrical three-phase network with $U_\mathrm{N} = \SI{\nicefrac{230}{400}}{\volt} $ (effective value of phase voltage / line-to-line voltage). The secondary side phase voltages point an effective value of 
$U_\mathrm{1i} = \SI{230}{\volt}, \forall i=a,b,c$. The thyristors and commutation can be assumed to be ideal.

%%%%%%%%%%%%%%%%%%%%%%%%%%%%%%%%%%%%%%%%%%%%%%%%%%%%%%%%%%%%%%%%%%%%%%%
 % M3C rectifier with RL Load
%%%%%%%%%%%%%%%%%%%%%%%%%%%%%%%%%%%%%%%%%%%%%%%%%%%%%%%%%%%%%%%%%%%%%%%
\begin{figure}[htb]
  \begin{center}
    \begin{circuitikz}
      \def\vd{1cm} % vertical distance inductors
      \def\htraf{0.75cm} % horizontal distance transformer coils
      \draw (0,0) to [short, o-] ++(0.5,0) coordinate (L1astart) to [short] ++(0.5,0) to [L] ++(2,0) coordinate (L1aend)
      (0,-1*\vd) to [short, o-] ++(1,0) coordinate (L1bstart) to [L] ++(2,0) coordinate (L1bend)
      (0,-2*\vd) to [short, o-] ++(1,0) coordinate (L1cstart) to [L] ++(2,0) coordinate (L1cend) -- ++(0,-0.5*\vd) to (\tikztostart -| L1astart) 
      to [crossing] ++(0, 1*\vd) to [crossing] ++(0, 1*\vd) to [short, -*] (L1astart)
      (L1aend) -- ++(0,-0.5*\vd) to (\tikztostart -| L1bstart) to [short, -*] (L1bstart)
      (L1bend) -- ++(0,-0.5*\vd) to (\tikztostart -| L1cstart) to [short, -*] (L1cstart);
      \draw let \p1=(L1aend) in (\x1 + \htraf, \y1) coordinate (L2astart) to [L, v^<=$u_{1\mathrm{a}}(t)$, voltage = straight] ++(2,0) to [short, i=$i_{1\mathrm{a}}(t)$] ++(0.5,0) coordinate (L2aend);
      \draw let \p1=(L1bend) in (\x1 + \htraf, \y1) coordinate (L2bstart) to [L, v^<=$u_{1\mathrm{b}}(t)$, voltage = straight] ++(2,0) to [short, i=$i_{1\mathrm{b}}(t)$] ++(0.5,0) coordinate (L2bend);
      \draw let \p1=(L1cend) in (\x1 + \htraf, \y1) coordinate (L2cstart) to [L, v^<=$u_{1\mathrm{c}}(t)$, voltage = straight] ++(2,0) to [short, i=$i_{1\mathrm{c}}(t)$] ++(0.5,0)  coordinate (L2cend);
      \draw (L2astart) to [short, -*] (L2bstart) to [short, -*] (L2cstart) -- ++(0, -1*\vd) -- ++(5,0) coordinate (Rend);
      \draw[double, double distance=3pt, thick] let \p1=(L1aend), \p2=(L2cstart) in (\x1/2+\x2/2, \y1) -- (\x1/2+\x2/2, \y2);
      \draw (L2aend) to [thyristor] ++(1.25,0) coordinate (D1end);
      \draw (L2bend) to [thyristor] ++(1.25,0) coordinate (D2end);
      \draw (L2cend) to [thyristor] ++(1.25,0) coordinate (D3end) to [short, -*] (D2end) to [short, -*] (D1end);
      \draw (D1end) to [short] ++(0.5,0) coordinate (u2) to [short, i=$i_2(t)$] ++(0.75,0) to [L, l=$L$] ++(2,0) coordinate (Ctop) to [short, i = $i_\mathrm{R}(t)$] ++(1.5,0) to [R, l=$R$] (Rend -| \tikztostart) to (Rend); 
      \draw (u2) to [open, v^>=$\hspace{0.5cm}u_2(t)$, voltage = straight] (Rend -| \tikztostart);
    \end{circuitikz}%
  \end{center}
  \caption{M3C topology with an input three-phase transformer and an RL-load.}
  \label{fig:M3C_topology_RL_no_filter}
\end{figure}


% Subtask1
\subtask{Calculate the firing angle $\alpha$, so that an active power of $P = \SI{6}{\kilo\watt}$ is delivered to the load. How big is the load current $i_\mathrm{2} = I_\mathrm{2}$?}
% Subtask2
\subtask{Draw the normalized control characteristic curve $\nicefrac{U_\mathrm{2}(\alpha)}{U_\mathrm{2}(\alpha=0)}$ and plot the  operating point $P = \SI{6}{\kilo\watt}$ at $R = \SI{5}{\Omega}$.}
% Subtask3
\subtask{Draw the curve of the converters' output voltage $u_\mathrm{2}(t)$ for the calculated control angle $\alpha$ from subtask 6.1.1.}
\subtask{Calculate the effective value $I^\mathrm{(1)}_\mathrm{1a}$ of the fundamental current wave $i^\mathrm{(1)}_\mathrm{1a}(t)$ and draw the fundamental wave. How big is the phase shift $\varphi_\mathrm{1a}$ between $u_\mathrm{1a}(t)$ and $i^\mathrm{(1)}_\mathrm{1a}(t)$.}
\subtask{Calculate the fundamental reactive power $Q^\mathrm{(1)}_\mathrm{1}$.}

%%%%%%%%%%%%%%%%%%%%%%%%%%%%%%%%%%%%%%%%%%%%%%%%%%%%%%%%%%%%%
%% Task 2: B6C converter at a motor load                   %%
%%%%%%%%%%%%%%%%%%%%%%%%%%%%%%%%%%%%%%%%%%%%%%%%%%%%%%%%%%%%%
\task{B6C converter at a motor load}

In a lifting drive, a permanent magnet DC motor is supplied by a B6C converter circuit. The B6C-topology is connected to the three-phase network.
When lifting as well as lowering the load, the motor is operated with nominal current and nominal voltage. 
This corresponds to a terminal voltage of $\overline{u}_\mathrm{2}$ when lifting the load and $-\overline{u}_\mathrm{2}$ when lowering it.
In order to generate the necessary torque $M$, the motor absorbs the current $\overline{i}_\mathrm{2}$.

%%%%%%%%%%%%%%%%%%%%%%%%%%%%%%%%%%%%%%%%%%%%%%%%%%%%%%%%%%%%%%%%%%%%%%%
 % B2U rectifier with capacitive output filtering
%%%%%%%%%%%%%%%%%%%%%%%%%%%%%%%%%%%%%%%%%%%%%%%%%%%%%%%%%%%%%%%%%%%%%%%
    \begin{figure}[htb]
        \begin{center}
            \begin{circuitikz}
                \def\vd{1.5cm} % vertical distance AC sources
                \def\hd{1.5cm} % horizontal distance diode bridge
                \def\h1d{5.0cm} % horizontal position first diode string
                % Base point for voltage supplies
                \coordinate (orig) at (0,0);
                % Voltage sources and neutral connection
                \draw 
                % draw the neutral connection
                (0,0) to [short, -*] ++(0,-1.5) to [short] ++(0,-1.5)
                % draw first phase ua
                (0,0) to [sinusoidal voltage source, v^<=$u_{1\mathrm{a}}$] ++(1.5, 0) to [short, i=$i_{1\mathrm{a}}(t)$]++(0.75,0) -- ++(0.25,0) coordinate (A)
                % draw second phase ub
                (0,-1*\vd) to [sinusoidal voltage source, v^<=$u_{1\mathrm{b}}$] ++(1.5, 0) to [short, i=$i_{1\mathrm{b}}(t)$]++(0.75,0) -- ++(0.25,0) coordinate (B)
                % draw third phase uc
                (0,-2*\vd) to [sinusoidal voltage source, v^<=$u_{1\mathrm{c}}$] ++(1.5,0) to [short, i=$i_{1\mathrm{c}}(t)$]++(0.75,0) -- ++(0.25,0) coordinate (C)
                %thyristor bridge
                % Add thyristor T1
                (\h1d,0) to [thyristor, l=$T_1$, name=D1] ++(0,1.25) coordinate (D1top)
                % Add thyristor T2
                (\h1d,-4.25) coordinate (D2bot) to [thyristor, l=$T_2$, name=D2] ++(0,1.25) to [short] (\h1d, 0)
                % Add connection to junction A
                (\h1d, 0) to [short, *-] (A)
                % Add thyristor T3
                (\h1d+\hd,0) to [thyristor, l=$T_3$, name=D3] ++(0,1.25) coordinate (D3top)
                % Add thyristor T4
                (\h1d+\hd,-4.25) coordinate (D4bot) to [thyristor, l=$T_4$, name=D4] ++(0,1.25) to [short] (\h1d+\hd, 0)
                % Add thyristor T5
                (\h1d+2*\hd,0) to [thyristor, l=$T_5$, name=D5] ++(0,1.25) coordinate (D5top)
                % Add thyristor T6
                (\h1d+2*\hd,-4.25) coordinate (D6bot) to [thyristor, l=$T_6$, name=D6] ++(0,1.25) to [short] (\h1d+2*\hd, 0)
                % Add connection to junction B
                (B -| D3) to [crossing, *-, mirror] ++(-2*\hd,0) -- (B)
                % Add connection to junction C
                (C -| D5) to [short, *-] ++(-\hd/2,0) to [crossing, mirror] ++(-\hd,0) to [crossing, mirror] ++(-\hd,0) -- (C)
                % Add wire T1-T3-T5
                (D1top) to [short, -*] (D3top) to [short, -*] (D5top) to [short, -] ++(0.5,0) coordinate (jL1)
                % Add inductor L and motor current
                (jL1) to [L, l=$L$, name = L] ++(2,0) to [short,i=$\overline{i}_\mathrm{mot}$] ++(0.5,0)  coordinate (jL2)
                % Add DC-motor and motor voltage
                (jL2) to [R, l=$R$, name = R, v_>=$\overline{u}_\mathrm{mot}$, voltage = straight]  (D6bot -| \tikztostart) to (D6bot)
                % Add wire T2-T3-T6
                (D2bot) to [short, -*] (D4bot) to [short, -*] (D6bot)
                % Add voltage arrow u2(t) between Dtop and Dbot
                (jL1) to [open, v^>=$\hspace{0.5cm}u_2(t)$, voltage = straight] (D6bot-|jL1)                
                % Add voltage arrow u2+n(t) between Dtop and neutral
                (D1top) ++(-0.2,0) to [open, v_>=$u_\mathrm{2,p}(t)$, voltage = straight] ++(-5.5,0)
                % Add voltage arrow u2-n(t) between Dbot and neutral
                (D2bot) ++(-0.2,0) to [open, v_>=$u_\mathrm{2,m}(t)$, voltage = straight] ++(-5.5,0)
                % Add voltage arrow between AC source a and b
                (A) to [open, v^>=$\hspace{0.75cm}u_{1\mathrm{ab}}(t)$, voltage = straight] (B)
                % Add voltage arrow between AC source b and c
                (B) to [open, v^>=$\hspace{0.75cm}u_{1\mathrm{bc}}(t)$, voltage = straight] (C)
                % Add voltage arrow between AC source a and c
                (-0.5,-2*\vd) to [open, v^>=$u_{1\mathrm{ca}}(t)\hspace{0.75cm}$, voltage = straight] (-0.5,0);
            \end{circuitikz}
        \end{center}
        \caption{B6C converter at a motor load}
        \label{fig:B6C_topology_WithMotor}
    \end{figure}




\begin{table}[ht]
    \centering  % Zentriert die Tabelle
    \begin{tabular}{ll}
        \toprule
        Input voltages: &  $u_\mathrm{a}(t) = \SI{230}{\volt}\cdot \sin(\omega t) \quad u_\mathrm{b}(t) = \SI{230}{\volt}\cdot \sin(\frac{2\pi}{3}\omega t) 
        \quad  u_\mathrm{c}(t) = \SI{230}{\volt}\cdot \sin(\frac{4\pi}{3}\omega t)$  \\
        Nom. motor current: & $\overline{i}_{\mathrm{mot}} = \SI{20}{\ampere}$ \\
        Nom. motor voltage: & $\overline{u}_\mathrm{mot} = \SI{466}{\volt}$ \\ 
        Frequency: & $f= \SI{50}{\hertz}$ \\ 
        \bottomrule
    \end{tabular}
    \caption{Parameters of the lifting drive with B6C converter.}  
    \label{table:ex06_Task2_ParametersOfTheCircuit}
\end{table}


% Subtask1
\subtask{Calculate the firing angle $\alpha_\mathrm{up}$ required for raising and the firing angle $\alpha_\mathrm{down}$ for lowering the load 
to operate the motor at rated voltage. The average voltage $\overline{u}_\mathrm{2}$  as a function of the firing angle $\alpha$ shall be determined 
by integrating the instantaneous output voltage $u_\mathrm{2}(t)$.}
% Subtask2
\subtask{Sketch following waveformes fpr tje two calculated firing angles $\alpha_\mathrm{up}$ and $\alpha_\mathrm{down}$:
\begin{itemize}
    \item The output voltage $u_\mathrm{2,p}(t)$ and $u_\mathrm{2,m}(t)$ the two partial converters (reference point is neutral)
          and shade the effective voltage-time area.
    \item The output voltage $u_\mathrm{2}(t)$ and the mean voltage $\overline{u}_\mathrm{2}(t)$.
    \item The current $i_\mathrm{1a}(t)$ and it's fundamental amplitude.
    \item The voltage of thyristor $u_\mathrm{T1}(t)$.
    \item Mark the conduction intervals of $T_\mathrm{1}$...$T_\mathrm{6}$.
\end{itemize}  
}
% Subtask3
\subtask{Calculate the mean value $P$ of the instantaneous active power $p(t)$, the fundamental reactive power $q_\mathrm{1a}(t)$
and the fundamental apparent power $s_\mathrm{1a}(t)$. Represent $p_\mathrm{1a}$, $q_\mathrm{1a}$ and $s_\mathrm{1a}$ in the complex plane.
}
% Subtask4
\subtask{Calculate the $s_\mathrm{1a}$, $p_\mathrm{1a}$ and $q_\mathrm{1a}$ and the fundamental component $g_\mathrm{1a}=\frac{i_\mathrm{1a,eff}}{i_\mathrm{1a}}$
         and the power factor $\lambda$ as a function of the effective values $u_\mathrm{1a}$, $i_\mathrm{1a}$, $i_\mathrm{1a,eff}$ and $\alpha$.}
