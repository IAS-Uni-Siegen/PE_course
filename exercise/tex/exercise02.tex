%%%%%%%%%%%%%%%%%%%%%%%%%%%%%%%%%%%%%%%%%%%%%%%%%%%%%%%%%%%%%
%% Begin exercise %%
%%%%%%%%%%%%%%%%%%%%%%%%%%%%%%%%%%%%%%%%%%%%%%%%%%%%%%%%%%%%%
\ex{Step-up and synchronous buck-boost converters}


%%%%%%%%%%%%%%%%%%%%%%%%%%%%%%%%%%%%%%%%%%%%%%%%%%%%%%%%%%%%%
%% Task 1: Step-down converter without output filter %%
%%%%%%%%%%%%%%%%%%%%%%%%%%%%%%%%%%%%%%%%%%%%%%%%%%%%%%%%%%%%%
\task {Boost converter}
The boost converter which is shown in \autoref{fig:boost converter} supplies an amplifier with a rated voltage of $U_{\mathrm{2}} = \SI{60}{\volt}$. For this exercise steady-state operation is assumed.

\begin{figure}[htb]
    \begin{center}
        
    %\begin{tikzpicture}
    \begin{circuitikz}[european currents,european resistors,american inductors]
        \draw
        (0,0) coordinate(u1o)
        to [V,v_=$U_1$] ++(0,-3) coordinate(u1u)
        (0,0) to[R, l=$R_L$] (2.5,0)
        (2.5,0) to[L, l=$\mathrm{L}$] (4.5,0)

        (4,0) coordinate(N1) to [short] ++(1.5,0) coordinate(Ud)
        ++(0,-1.5) node[nigfete](Trans){}
        (Ud) to [short] (Trans.drain)
        (Trans.source) to [short,-] ++(0,0) coordinate(U2p)
      
        (4,0) to [short,-] (9.5,0)

        (5.5,-1.5) to [short,-] ++(0,-1.5) 

        (6.5,0) to[D, l=$D$] (8,0)
                
        (8,0) to[C,l_=$C$] (8,-3)

        (9.5,0) to [R,l_=$R$,v^=$U_\text{2}$,voltage shift=0.5, voltage=straight] (9.5,-3)

        (9.5,-3) to [short,-] (0,-3)

        (0.75,0) to [short,i^<=$i_1$(t)] (u1o)
        (4.5,0) to [short,-*,i^>=$i_L$(t)] (5.5,0)
        
        (5.5,-0.5) to [short,i^<=$i_\mathrm{T}$(t)] (5.5,-1)
        
        (9,0) to [short,i^<=$I_2$(t)] (8.5,0)
        ;
% (u3o) to [short,-] ++(1,0) coordinate(v1)
        ;
    \end{circuitikz}
    \caption{Boost converter}
     \label{fig:boost converter}
\end{center}
%\end{tikzpicture}
\end{figure}


The specification of the boost converter is given in \autoref{table:Parameters of the circuit}:

\begin{table}[ht]
    \centering  % Zentriert die Tabelle
    \begin{tabular}{llll}
        \toprule
        
        Input voltage: &  $U_{\mathrm{1}} = \SI{12}{\volt}$ & Output voltage: & $U_{\mathrm{2}} = \SI{60}{\volt}$ \\ 
        Output current: & $I_2 = \SI{2}{A}$  & Minimal output power: & $P_{\mathrm{out,min}} = \SI{10}{\watt}$ \\ 
        Output voltage ripple: & $\triangle u_{\mathrm{2}} = \SI{120}{\milli\volt}$  & Swichting frequency: & $f_{\mathrm{s}} = \SI{100}{\kilo\hertz}$ \\ 
        \bottomrule
    \end{tabular}
    \caption{Parameters of the boost converter.}  % Beschriftung der Tabelle
    \label{table:Parameters of the circuit}
\end{table}

First, all losses of the boost converter including the resistance of the inductor are neglected.

\subtask{Derive the duty cycle, which leads to the specific output voltage of $U_{\mathrm{2}} = \SI{60}{\volt}$. The steady state is considered:$i_{\mathrm{L}}(t=0)=i_{\mathrm{L}}(t=T_{\mathrm{s}})$} 

\begin{solutionblock}
    \begin{solutionfigure}[ht]
    \centering
    \begin{tabular}{cc}
        ECD $T_{\mathrm{s,on}}$ & ECD $T_{\mathrm{s,off}}$\\
        \begin{circuitikz}[european currents,european resistors,american inductors,]
            \draw
            (0,0) coordinate(u1o)
            to [V,v^=$U_2$] ++(0,-3) coordinate(u1u)
            (u1o) to [short,-] ++(-0.25,0) coordinate(o)
            (u1u) to [short,-] ++(-1,0) coordinate(u2u)
            (u2u) to [short,-,v^<=$u_\text{s}$,mirror,voltage=straight] ++(0,3) coordinate(u2o)
            (u2o) to [short,-] ++(0.25,0) coordinate(u) ++ (-3.5,0) coordinate(u3o)
            (u3o) to [V,v_=$U_1$] ++(0,-3) coordinate(u3u)
            (u3u) to [short,-] ++(3.5,0) coordinate(t)
            
            (u3o) ++(+2,0) to [L,l_=$L$,v^<=$u_\text{L}(t)$,mirror,voltage=straight]   
             ++(-1,0) coordinate(uqo)
            (u3o) to [short,-] ++(1,0) coordinate(v1)
            (u2o) to [short,-] ++(-1.15,0) coordinate(v2)
            (u2o) to [short,-,i^<=$i_L$(t)] ++(-1,0)
            ;
        \end{circuitikz} 
    &
    
        \begin{circuitikz}[european currents,european resistors,american inductors,]
            \draw
            (0,0) coordinate(u2o)
            to [V,v^=$U_2$, voltage=straight] ++(0,-3) coordinate(u2u)
                (u2o) to [short,-,i^<=$i_L$(t)] ++(-1,0) to [L,l_=$L$,v^<=$u_\text{L}(t)$,mirror,voltage=straight] ++(-3,0) coordinate(Ll) ++(-1,0) coordinate(uqo)
                (-4,0) to [V,v_=$U_1$] ++(0,-3) coordinate(u2u)
                (-4,-3)  to [short,-] ++(4,0) coordinate(r)
                ;
        \end{circuitikz}
    \end{tabular}
    \caption{ECD of the boost converter for different switching states.}
    \label{fig:switching_states_step-down_converter}
\end{solutionfigure}
    ECD is short for equivalent circuit diagram.
    As there are no losses at the transistor, the voltage $U_{\mathrm{1}}$ is equal to $u_{\mathrm{L}}$ for the switch-on period $T_{\mathrm{on}}$.
    \begin{equation}
        U_{\mathrm{1}} = u_{\mathrm{L}} = L \frac{\Delta i_{\mathrm{L}} }{\Delta t} 
    \end{equation}
    This subtask does not consider the complete period $T_{\mathrm{s}}$, but only the switch-on period $T_{\mathrm{on}}$. Furthermore, the initial state of the current $i_{\mathrm{L}}$ must also be considered.
    That is why we can completely rearrange the equation.
    \begin{equation}
        i_{\mathrm{L}}(T_{\mathrm{on}}) =  \frac{U_{\mathrm{1}} }{L}T_{\mathrm{on}} +  i_{\mathrm{L}}(0)
    \end{equation}
    The following meshing can be used for the switch-off time $T_{\mathrm{off}}$.
    \begin{equation}
        U_{\mathrm{1}} = u_{\mathrm{L}} + U_{\mathrm{2}}
    \end{equation}
    Now insert the differential expression for $u_{\mathrm{L}}$. 
    \begin{equation}
        L \frac{\Delta i_{\mathrm{L}} }{\Delta t} = U_{\mathrm{1}} - U_{\mathrm{2}}
    \end{equation}
    The equation for the current $i_{\mathrm{L}}$ at the switch-off time $T_{\mathrm{sw,off}}$ results from the term describing the discharge of the inductance from $T_{\mathrm{sw,off}}$ together with the initial condition of the current at this time.
    \begin{equation}
        i_{\mathrm{L}}(t) = \frac{U_{\mathrm{1}}-U_{\mathrm{2}} }{L} (t-T_{\mathrm{sw,on}})+i_{\mathrm{L}}(T_{\mathrm{sw,on}})
    \end{equation}
    This results in the equation for the entire current curve over the period: 
    \begin{equation}
        i_{\mathrm{L}}(t) = \frac{U_{\mathrm{1}}-U_{\mathrm{2}} }{L} (t-T_{\mathrm{sw,on}})+\frac{U_{\mathrm{1}}}{L}T_{\mathrm{sw,on}}+i_{\mathrm{L}}(0)
    \end{equation}
    If you now consider the steady state $i_{\mathrm{L}}(t=0)=i_{\mathrm{L}}(t=T_{\mathrm{s}})$, this results in the following:
    \begin{equation}
        i_{\mathrm{L}}(0) = \frac{U_{\mathrm{1}}-U_{\mathrm{2}} }{L} (T_{\mathrm{sw,off}})+\frac{U_{\mathrm{1}}}{L}T_{\mathrm{sw,on}}+i_{\mathrm{L}}(0)
    \end{equation}
    Now the initial condition $i_{\mathrm{L}}(0)=0$ is assumed.
    \begin{equation}
        0 = \frac{U_{\mathrm{1}}-U_{\mathrm{2}} }{L} (T_{\mathrm{sw,off}})+\frac{U_{\mathrm{1}}}{L}T_{\mathrm{sw,on}}
    \end{equation}

    \begin{equation}
        U_{\mathrm{1}}T_{\mathrm{sw,on}}= (-U_{\mathrm{1}}+U_{\mathrm{2}})T_{\mathrm{sw,off}}
    \end{equation}

    \begin{equation}
        U_{\mathrm{1}}(T_{\mathrm{sw,off}}+T_{\mathrm{sw,on}})=  U_{\mathrm{2}}T_{\mathrm{sw,off}}
    \end{equation}
    The switch-off period $T_{\mathrm{sw,off}}$ plus the switch-on period $T_{\mathrm{sw,on}}$ is the total period $T_{\mathrm{s}}$.

    The following equation can now be established for the duty cycle.
    \begin{equation}
        \frac{U_{\mathrm{1}}}{U_{\mathrm{2}}}= \frac{T_{\mathrm{sw,off}}}{T_{\mathrm{s}}}= \frac{T_{\mathrm{s}}}{T_{\mathrm{s}}}-\frac{T_{\mathrm{sw,on}}}{T_{\mathrm{s}}} = 1-D
    \end{equation}

    \begin{equation}
        D = 1-\frac{U_{\mathrm{1}}}{U_{\mathrm{2}}} = 1- \frac {\SI{12}{\volt}} {\SI{60}{\volt}} = 0.8
    \end{equation}
\end{solutionblock}


\subtask{Determine the average input current $\overline i_{\mathrm{L}}$}
\begin{solutionblock}
    With neglecting the losses $P_{\mathrm{l}} = \SI{0}{\watt}$ of the boost converter and assuming a good smoothing of the inductor current, the average value $\overline i_{\mathrm{L}}$ is equal $i_{\mathrm{1}}$ and can be calculated using the power balance.
    \begin{equation}
        P_{\mathrm{2}} = U_{\mathrm{2}} I_{\mathrm{2}} = {\SI{60}{\volt}} \cdot {\SI{2}{\ampere}} = {\SI{120}{\watt}}
    \end{equation}
    Since no losses are taken, the power  $P_{\mathrm{1}}$ is equal to  $P_{\mathrm{2}}$.
    This results in the average current $\overline i_{\mathrm{L}}$:
    \begin{equation}
         I_{\mathrm{1}} = overline i_{\mathrm{L}} = \frac{P_{\mathrm{1}}}{U_{\mathrm{1}}}= \frac{\SI{120}{\watt}}{\SI{12}{\volt}} = \SI{12}{\ampere}
    \end{equation}
    An alternative solution would be to determine the average current $\overline i_{\mathrm{L}}$ via the duty cycle.
    \begin{equation}
    \frac{I_{\mathrm{1}}}{I_{\mathrm{2}}}=\frac{1}{1-D}
    \end{equation}
    \begin{equation}
     I_{\mathrm{1}}=\frac{I_{\mathrm{2}}}{1-D}=\frac{\SI{2}{\ampere}}{1-0.8}= \SI{10}{\ampere}
    \end{equation}
\end{solutionblock}

\subtask{ Define a suitable inductivity for the coil L, so that the boost converter is operating in continuous conduction mode for the minimum output power given in the specification above. Determine the maximal switch-off current $i_{\mathrm{V1}}$ of the transistor $\mathrm{V}_1$ for the rated output current $I_{\mathrm{2}} = \SI{2}{\ampere}$.}
\begin{solutionblock}
    
\end{solutionblock}
