%%%%%%%%%%%%%%%%%%%%%%%%%%%%%%%%%%%%%%%%%%%%%%%%%%%%%%%%%%%%%%%%%%%%%%%%%%
% CircuitWithOneTransistorAndOneLoadResistor
%%%%%%%%%%%%%%%%%%%%%%%%%%%%%%%%%%%%%%%%%%%%%%%%%%%%%%%%%%%%%%%%%%%%%%%%%%

\begin{figure}[h]
    \begin{center}
        \begin{circuitikz}[european currents,european resistors,american inductors]
            \draw
            (0,0) coordinate(N1) to [short] ++(1.5,0) coordinate(U1p)
            ++(2,0) node[nigfete,rotate=90](Trans){}
            (U1p) to [short,i=$i_1(t)$] (Trans.drain)
            (Trans.source) to [short,i=$i_2(t)$] ++(1.5,0) coordinate(U2p)
            to [short,-] ++(1,0) to [R,l^=$R_\text{L}$,  v=$u_2(t)$, voltage = straight ] ++(0,-3) to [short] ++(-1,0) coordinate(U2n) to [short] (1,-3) coordinate(U1n) to [short,-] ++(-1,0) coordinate(N10)
             (N1) to [V,v=$U_1$, voltage = straight] (N10)
            % (U1p) to [open,v^=$U_1$] (U1n)
             %(U2p) to [open,v_=$u_2(t)$] (U2n)
             (Trans.gate) to [short,o-] ++(0,-.3) to [sqV] ++(0,-1) 
             (Trans.gate) ++(-0.3,0) node(T1){$\text{T}_\text{1}$}
             
            ;
        \end{circuitikz}
    \end{center}
    \caption{Circuit with one transistor and one load resistor.}
        \label{fig:transistor circuit}
    \end{figure}