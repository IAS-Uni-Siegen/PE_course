\begin{solutionfigure}[htb]
\centering
\begin{tikzpicture}
    % Achsen zeichnen
    \draw[->] (0,0) -- (14,0) node[right] {$k$}; % x-Achse
    \draw[->] (0,0) -- (0,5) node[above] {$\frac{\hat{u}_\mathrm{2a}^\mathrm{(k)}}{\hat{u}_\mathrm{2a}^\mathrm{(1)}}$}; % y-Achse

    % Ticks und Beschriftungen auf der x-Achse
    \foreach \x in {1, 5, 7, 11, 13} {
        \draw (\x,0.05) -- (\x,-0.05) node[below] {\x};
    }
    \foreach \x in {2, 3, 4, 6, 8, 9, 10, 12} {
        \draw (\x,0.02) -- (\x,-0.02); % kleinere Ticks
    }

    % Ticks und Beschriftung auf der y-Achse
    \draw (-0.05,4.5) -- (0.05,4.5) node[left] {1};
    \draw (-0.05,3.6)  node[left] {0.8};
     \draw (-0.05,2.7)  node[left] {0.6};
    \draw (-0.05,1.8) node[left] {0.4};
    \draw (-0.05,0.9) node[left] {0.2};
    \draw (-0.05,0) node[left] {0};
    % Balken zeichnen
    \foreach \x/\y in {1/4.5, 5/0.9, 7/0.6, 11/0.5, 13/0.4} {
        \draw[thick] (\x,0) -- (\x,\y); % Balken
        \draw[thick] (\x-0.1,\y) -- (\x+0.1,\y); % Querstrich oben
    }
\end{tikzpicture}
\caption{Normalization to the amplitude of the fundamental oscillation.}
\label{fig:Normalization to the amplitude of the fundamental oscillation}
\end{solutionfigure}
