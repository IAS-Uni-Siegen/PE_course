\begin{figure}[h!]

   %   \documentclass{standalone}
   %   \usepackage{pgfplots}
   %   \pgfplotsset{compat=1.18} % Kompatibilität für neuere Versionen
          \centering
          \begin{tikzpicture}
               \pgfplotsset{set layers}
          \begin{axis}[
           % x/y range adjustment
           scale only axis,
           ymin=-4, ymax=1.5,
           xmin=0, xmax=360,
           axis x line=none, 
           samples=500,
           axis y line=center,
           axis x line=middle,
           extra y ticks=0,
           % Label text
           xlabel={$\omega t / \text{rad}$},
           ylabel={$u/\mathrm{V}$},
           % Label adjustment
           x label style={at={(axis description cs:1,0.5)},anchor=west},
           y label style={at={(axis description cs:0,.97)},anchor=south,yshift=0.2cm},
           width=0.6\textwidth,
           height=0.3\textwidth,
           % x-Ticks
           xtick={0,60,120,180,240, 300, 360},
           xticklabels={0,$\frac{\pi}{3}$,$\frac{2\pi}{3}$,$\pi$,$\frac{4\pi}{3}$, $\frac{5\pi}{3}$, $2\pi$},
           xticklabel style = {anchor=north},
           % y-Ticks
           ytick={-1,-0.5,0,0.5,1},
           yticklabels={-1,-0.5,0,0.5,1},
           yticklabel style = {anchor=east},
           % Grid layout
           grid,
           %grid style={line width=.1pt, draw=gray!10},
           %major grid style={line width=.2pt,draw=gray!90},
       ] 
       % modulation s
       \addplot[blue, domain= 0:360, solid] {0.75*sin(x)};
     % carrier signal c
      \draw[thin, black] (0,-1) -- (60,1) -- (60,-1) -- (120,1) -- (120,-1) -- (180,1) -- (180,-1) -- (240,1) -- (240,-1) -- (300,1) -- (300,-1) -- (360,1) -- (360,-1);
       %\draw (0,-1) \foreach \x in {0 ,60, 120, 180, 240, 300} {-- ++(60,2) -- ++(0,-2) };
       
       % Label of s
       \node[black, fill=white, inner sep = 1pt, anchor = south] at (axis cs:85,0.8) {$s^{*}(t)$};
       % Label of c
       \node[black, fill=white, inner sep = 1pt, anchor = south] at (axis cs:200,0.4) {$c(t)$};
          \end{axis}
          \begin{axis}[
           % x/y range adjustment
           scale only axis,
           ymin=-6, ymax=1.5,
           xmin=0, xmax=360,
           samples=500,
           axis y line=center,
           axis x line=bottom,
           % Label text
           xlabel={$\omega t / \text{rad}$},
           ylabel={$u/\mathrm{V}$},
           % Label adjustment
           x label style={at={(axis description cs:1,0)},anchor=west},
           y label style={at={(axis description cs:0,.97)},anchor=south,yshift=0.2cm},
           width=0.6\textwidth,
           height=0.3\textwidth,
           % x-Ticks
           xtick={0,60,120,180,240, 300, 360},
           xticklabels={0,$\frac{\pi}{3}$,$\frac{2\pi}{3}$,$\pi$,$\frac{4\pi}{3}$, $\frac{5\pi}{3}$, $2\pi$},
           xticklabel style = {anchor=north},
           % y-Ticks
           ytick={-5,-4.5,-4,-3.5,-3},
           yticklabels={-1,-0.5,0,0.5,1},
           yticklabel style = {anchor=east},
           % Grid layout
           grid,
           %grid style={line width=.1pt, draw=gray!10},
           %major grid style={line width=.2pt,draw=gray!90},
       ] 
 
     % switching times
     \draw[thin, black, dashed] (46,1.5) -- (46,-6);
     \draw[thin, black, dashed] (60,1.5) -- (60,-6);  
     \draw[thin, black, dashed] (110,1.5) -- (110,-6);
     \draw[thin, black, dashed] (120,1.5) -- (120,-6);
     \draw[thin, black, dashed] (158,1.5) -- (158,-6);
     \draw[thin, black, dashed] (180,1.5) -- (180,-6);
     \draw[thin, black, dashed] (202,1.5) -- (202,-6);
     \draw[thin, black, dashed] (240,1.5) -- (240,-6);
     \draw[thin, black, dashed] (249,1.5) -- (249,-6);
     \draw[thin, black, dashed] (300,1.5) -- (300,-6);
     \draw[thin, black, dashed] (315,1.5) -- (315,-6);
     \begin{solutionblock}
             
     % Fundamental voltage
     \addplot[blue, domain= 0:360, solid] {0.75*sin(x)-4};
     % Output voltage
     \draw[thin, red, solid] (0,-3) -- (46,-3) -- (46,-5) -- (60,-5) -- (60,-3) -- (110,-3) -- (110,-5) -- (120,-5);
     \draw[thin, red, solid] (120,-5) -- (120,-3) -- (158,-3) -- (158,-5) -- (180,-5) -- (180,-3) -- (202,-3);
     \draw[thin, red, solid] (202,-3) -- (202,-5) -- (240,-5) -- (240,-3) -- (249,-3) -- (249,-5) -- (300,-5);
     \draw[thin, red, solid] (300,-5) -- (300,-3) -- (315,-3) -- (315, -5) -- (360, -5);
 
       % Label of u_2
       \node[black, fill=white, inner sep = 1pt, anchor = south] at (axis cs:90,-2.75) {$u_\mathrm{2}(t)$};
       % Label of u^1_2
       \node[black, fill=white, inner sep = 1pt, anchor = south] at (axis cs:275,-4.5) {$u^\mathrm{(1)}_\mathrm{2}(t)$};
     \end{solutionblock}
          \end{axis}                      
          \end{tikzpicture}
          \caption{Output voltage $u_\mathrm{2}(t)$ and fundamental voltage component $u^\mathrm{(1)}_\mathrm{2}(t)$.}
          \label{sfig:ex07_sub1.1_modulation}
  \end{figure}