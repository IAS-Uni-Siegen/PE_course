%%%%%%%%%%%%%%%%%%%%%%%%%%%%%%%%%%%%%%%%%%%%%%%%%%%%%%%%%%%%%%%%%%%%%%%%%%
%  Inductor voltage within the switching period for case1
%%%%%%%%%%%%%%%%%%%%%%%%%%%%%%%%%%%%%%%%%%%%%%%%%%%%%%%%%%%%%%%%%%%%%%%%%%

\begin{solutiontable}[ht]
    \centering  % Zentriert die Tabelle
    \begin{tabular}{ll}
        \toprule
        $U_\mathrm{L} = U_\mathrm{1}$ & $0\leq t \leq D_2 \cdot T_s$ \\
        $U_\mathrm{L} = U_\mathrm{1}-U_\mathrm{2}$ & $D_2 \cdot T_s \leq t  \leq D_1 \cdot T_s$ \\
        $U_\mathrm{L} = -U_\mathrm{2}$ & $D_1 \cdot T_s \leq t \leq T_s$ \\
        \bottomrule
    \end{tabular}
    \caption{$U_\mathrm{L}$ within the switching period.} 
    \label{table:VoltageAtInductorInCase1_2}
\end{solutiontable}