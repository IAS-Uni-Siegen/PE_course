%%%%%%%%%%%%%%%%%%%%%%%%%%%%%%%%%%%%%%%%%%%%%%%%%%%%%%%%%%%%%
%% Examtask 1: Step-down converter with output filter %%
%%%%%%%%%%%%%%%%%%%%%%%%%%%%%%%%%%%%%%%%%%%%%%%%%%%%%%%%%%%%%

\task{Step-down converter}

%%%%%%%%%%%%%%%%%%%%%%%%%%%%%%%%%%%%%%%%%%%%%%
\taskGerman{Tiefsetzsteller}
%%%%%%%%%%%
% Combination of Tasks 1.2 + 1.3 from our exercises (little bit of steady state analysis plus losses / efficiency calculation)
% Reduction to (at most) 5 or 6 subquestions, i.e. a subset of the original exercises limited to some core questions for the buck converter
% Use a completely different set of parameter / signal values to change the setting compared to the exercise tasks, maybe also add a little 'pre story' for which application the buck converter should be used
% However, the 'spirit' of this first exam task should be definitely entry friendly, i.e., provide the students with a good start into the PE exam enabling to gain confidence for the next tasks 
%%%%%%%%%%%

In industrial control systems, a $\SI{50}{\volt}$ DC power supply is commonly used to power various components.
Additionally, several sensors and servo motors require a stable $\SI{12}{\volt}$ DC power supply.
For this purpose an efficient step-down (buck) converter is required to provide high currents,
especially when multiple servo motors or actuators are operating simultaneously.

\begin{germanblock}
In industriellen Steuerungssystemen wird häufig eine $\SI{50}{\volt}$ DC-Stromversorgung verwendet,
um verschiedene Komponenten zu versorgen. Zusätzlich benötigen mehrere Sensoren und Servomotoren
eine stabile $\SI{12}{\volt}$ DC-Spannung. Dafür wird ein effizienter Tiefsetzsteller benötigt,
der hohe Ströme liefern kann – insbesondere, wenn mehrere Servomotoren oder Aktoren gleichzeitig 
in Betrieb sind.    
\end{germanblock}


% figure figStepDownConverterOutputFilter
%%%%%%%%%%%%%%%%%%%%%%%%%%%%%%%%%%%%%%%%%%%%%%%%%%%%%%%%%%%%%%%%%%%%%%%%%%
% StepDownConverterOutputFilter
%%%%%%%%%%%%%%%%%%%%%%%%%%%%%%%%%%%%%%%%%%%%%%%%%%%%%%%%%%%%%%%%%%%%%%%%%%

\begin{figure}[htb]
    \begin{center}
        \begin{circuitikz}[european currents,european resistors,american inductors]
            \draw
            (0,0) coordinate(U+) to [short,-] ++(0.5,0)  
            node[currarrow](Tl){} -- ++(1,0) ++(0.5,0) node[nigfete,rotate=90](Trans){} -- ++(1.5,0) coordinate(Tr)
                to [short,-*] ++(0.5,0) coordinate(junc1)   -- ++(0.5,0) coordinate(Ll) to [L,l_=$L$,i^=$i_\text{L}$]
                ++(2,0) coordinate(Lr) to [short,-*] ++(0.5,0) coordinate(junc2)  -- ++(2,0)
                coordinate(Rt) to [R,l_=$R$,i_=$i_\text{2}$,v^=$U_\text{2}$,voltage shift=0.5, voltage=straight] ++(0,-3) coordinate(Rb)
                to [short,-*] ++(-2,0) coordinate(junc3) to [short,-*] ++(-3,0) coordinate(junc4) to [short] ++(-2,0)
                coordinate(junc5) to [short,-] ++(-2,0) coordinate(U-)
                (junc2) to [C,l_=$C$](junc3)
                (junc4) to [short]  ++(0,0.5) coordinate(Db) to[D-,l^=$D$]  ++(0,2) coordinate(Dt) to [short] (junc1)
                (Trans.G)  to [sqV] ++(0,-1)(junc5) 
            
            (U+) to [V=$U_1$] (U-)
            
            (Trans)  node[anchor=south,color=black]{$T_1$}	
            (Tl)  node[anchor=south,color=black]{$i_\text{1}$}	
            ;
        \end{circuitikz}
    \end{center}
    \caption{Circuit with one transistor, filter and one load resistor.}
    \label{fig:step_down_converter_output_filter}
\end{figure}



\begin{table}[ht]
    \centering  % Zentriert die Tabelle
    \begin{tabular}{llll}
        \toprule
        \multicolumn{2}{l}{\textbf{General parameters:}} & \multicolumn{2}{l}{\textbf{IGBT:}} \\ 
        Input voltage: &  $U_{\mathrm{1}} = \SI{50}{\volt}$ & Collector-emitter voltage: & $u_{\mathrm{on},\mathrm{CE}} = \SI{2}{\volt}$ \\
        Output voltage: & $U_2 = \SI{12}{\volt}$  & Switch-on losses: & $E_{\text{on},\mathrm{D}} = \SI{20}{\micro\joule}$ \\
        Output current: & $I_2 = \SI{70}{\ampere}$  & Switch-off losses: &  $E_{\text{off},\mathrm{D}} = \SI{40}{\micro\joule}$ \\
        Switching frequency: & $f_\mathrm{s} = \SI{100}{\kilo\hertz}$  & &  \\
        \midrule
        \textbf{Inductance:} & $L = \SI{20}{\micro\henry}$ & & \\
        \multicolumn{4}{c}{The diode is considered as ideal and the filter capacitor is $C \rightarrow \infty$.}  \\ 
        \bottomrule
    \end{tabular}
    \caption{Parameters of the circuit.}  % Beschriftung der Tabelle
    \label{table:ex01_Parameters of the circuit}
\end{table}

\subtask{At what duty cycle $D$ should the step-down converter be operated?
         Calculate and sketch the voltage $u_\mathrm{L}(t)$ and  current $i_\mathrm{L}(t)$ over 2 periods.}{4}%
\begin{hintblock}
    The voltage drop across the transistor must be taken into account.
\end{hintblock}

\subtaskGerman{Bei welchem Tastverhältnis $D$ sollte der Tiefsetzsteller betrieben werden?
Skizzieren Sie die Spannung $u_\mathrm{L}(t)$ und den Strom $i_\mathrm{L}(t)$ über zwei Perioden.}
\begin{germanhintblock}
    Der Spannungsabfall über den Transistor ist zu berücksichtigen.
\end{germanhintblock}

% Solution of subtask
\begin{solutionblock}
    The duty cycle corresponds to
    \begin{equation*}
        \begin{aligned}
            D&=\frac{U_\mathrm{out}}{U_\mathrm{in}}=\frac{U_\mathrm{2}}{U_\mathrm{1}-u_\mathrm{on,CE}} \\
            &=\frac{\SI{12}{\volt}}{\SI{50}{\volt}-\SI{2}{\volt}}=0.25.
        \end{aligned}
        \label{eq:dutycycle}
    \end{equation*}
    If the transistor conducts, the voltage $u_\mathrm{s}$ results in
    \begin{equation*}
        u_\mathrm{s,on}=U_\mathrm{1}-u_\mathrm{on,CE}=\SI{50}{\volt}-\SI{2}{\volt}=\SI{48}{\volt}.
        \label{eq:u_on}
    \end{equation*}

    If the transistor does not conduct, the diode conducts and $u_\mathrm{s}$ yielding
    \begin{equation*}
        u_\mathrm{s,off}=\SI{0}{\volt}.
        \label{eq:u_off}
    \end{equation*}

    This leads to the result
    \begin{equation*}
        u_\mathrm{L}= u_\mathrm{s}-U_\mathrm{2}
        \begin{cases}
            u_\mathrm{L,on}= \SI{48}{\volt}-\SI{12}{\volt} = \SI{36}{\volt}, \quad  &\text{if $T_1$ conducts}, \\
            u_\mathrm{L,off}= -\SI{12}{\volt}, \quad &\text{if $T_1$ does not conduct}.
        \end{cases}
        \label{eq:i_inductor}
    \end{equation*}

    Using the voltage at the inductor while the transistor conducts, the current ripple yields:
    \begin{equation*}
            i_\mathrm{L,ripple}= \frac{u_\mathrm{L,on}}{L} \cdot T_\mathrm{on} 
            =\frac{u_\mathrm{L,on} \cdot D}{L f_\mathrm{s}}
            =\frac{ \SI{36}{\volt} \cdot 0.25} {\SI{20}{\micro\henry} 
            \cdot \SI{100}{\kilo\hertz}} = \SI{4.5}{\ampere}.
        \label{eq:i_inductor}
    \end{equation*}
    The maximum and minimum inductor current result in:
    \begin{equation*}
        \begin{aligned}
            i_\mathrm{L,max}&= I_\mathrm{2} + 0.5 \cdot i_\mathrm{L,ripple} = \SI{70}{\ampere} + \SI{2.25}{\ampere} = \SI{72.25}{\ampere} \\
            i_\mathrm{L,min}&= I_\mathrm{2} - 0.5 \cdot i_\mathrm{L,ripple} = \SI{70}{\ampere} - \SI{2.25}{\ampere} = \SI{67.75}{\ampere}.
        \end{aligned}
        \label{eq:i_inductor}
    \end{equation*}
        
    % solutionfigure
    %%%%%%%%%%%%%%%%%%%%%%%%%%%%%%%%%%%%%%%%%%%%%%%%%%%%%%%%%%%%%%%%%%%%%%%%%%
% SolutionFigure
%%%%%%%%%%%%%%%%%%%%%%%%%%%%%%%%%%%%%%%%%%%%%%%%%%%%%%%%%%%%%%%%%%%%%%%%%%

\begin{solutionfigure}[htb]  
    \begin{center}
        \definecolor{orange}{rgb}{1.0,0.5,0}
        \begin{tikzpicture}
            % groupplot begin
            \begin{groupplot}[group style={
                group size=1 by 2,
                xlabels at=edge bottom,
                y descriptions at=edge left,
                },
                domain=0:15,
                % x/y range adjustment
                xmin=0, xmax=10.25,
                samples=500,
                axis y line=center,
                axis x line=middle,
                extra y ticks=0,
                % Label text
                xlabel={$t / \SI{}{\micro\second}$},,
                % Label adjustment
                x label style={at={(axis description cs:1,0.5)},anchor=west},
                y label style={at={(axis description cs:-.05,.97)},anchor=south},
                width=0.95\textwidth,
                height=0.3\textwidth,
                % x-Ticks
                xtick={0,1,2,3,4,5,6,7,8,9,10},
                xticklabels={0,2,4,6,8,10,12,14,16,18,20},
                xticklabel style = {yshift=0.1cm, xshift=0.15cm,anchor=north},
                % Grid layout
                grid=both,
                grid style={line width=.1pt, draw=gray!10},
                major grid style={line width=.2pt,draw=gray!50},
                ]
                % Ul
                \nextgroupplot[
                    ymin=-22, ymax=42,
                    samples=500,
                    extra y ticks=0,
                    % x-Label text
                    xlabel={$t / \SI{}{\micro\second}$},,                    
                    % y-Label
                    y label style={at={(axis description cs:-.05,1)},anchor=south},
                    height=0.25\textwidth,
                    ylabel={\color{black}$u_\mathrm{L}(t)/\mathrm{V}$},
                    % y-Ticks
                    ytick={-20,-10,0,10,20,30,40},
                    yticklabels={-20,,0,,20,,40},
                    yticklabel style = {anchor=east},
                ]


                \addplot[color=blue,mark=none,solid,line width=1pt] coordinates{
                    (0, 36)
                    (1.33, 36)
                    (1.33, -12.8)
                    (5, -12.8)
                    (5, 36)
                    (6.33, 36)
                    (6.33, -12.8)
                    (10, -12.8)
                    (10, 36)
                    (10.1, 36)
                };
                % il
                \nextgroupplot[
                    ymin=64, ymax=78,
                    samples=500,
                    % x-Label text
                    xlabel={$t / \SI{}{\micro\second}$},,
                    x label style={at={(axis description cs:1,0)},anchor=west},                    
                    % y-Label
                    y label style={at={(axis description cs:-.05,1)},anchor=south},
                    height=0.25\textwidth,
                    ylabel={\color{black}$i_\text{L}(t)/\mathrm{A}$},
                    ytick={65,70,75},
                    yticklabels={65,70,75},
                    yticklabel style = {anchor=east},                        
                    xticklabel style = {xshift=0cm,anchor=north},
                    ]                
                    % Voltage ul(wt)
                    \addplot[red, domain= 0:1.33,solid,line width=1pt] {67.75+x*4.5/1.33};
                    \addplot[red, domain= 1.33:5,solid,line width=1pt] {72.25-(x-1.33)*4.5/3.67};
                    \addplot[red, domain= 5:6.33,solid,line width=1pt] {67.75+(x-5)*4.5/1.33};
                    \addplot[red, domain= 6.33:10,solid,line width=1pt] {72.25-(x-6.33)*4.5/3.67};
                    \addplot[red, domain= 10:10.1,solid,line width=1pt] {67.75+(x-10)*4.5/1.33};
            \end{groupplot}
        \end{tikzpicture}
    \end{center}
    \caption{Relevant voltage and current signals.}
    \label{fig:transistor_circuit}
\end{solutionfigure}       


\end{solutionblock}

\subtask{ How large is the power demand of the load, if the step-down converter operates
          in boundary conduction mode (BCM)?}{2}

\subtaskGerman{Wie groß ist die aufgenommene Leistung der Last, wenn der Tiefsetzsteller 
                im Lückgrenzbetrieb arbeitet?}

% Solution of subtask
\begin{solutionblock}
    The current at boundary conduction mode depends on the current ripple according
    \begin{equation*}
       I_\mathrm{2,BCM}=\frac{i_\mathrm{L,ripple}}{2} =\frac{\SI{4.5}{\ampere}}{2}=\SI{2.25}{\ampere}.
       \label{eq:i_bcm}
    \end{equation*}
    Using the current at boundary conduction mode, the power of the load is obtained by 
    \begin{equation*}
        P_\mathrm{load,BCM}=U_\mathrm{2} \cdot I_\mathrm{2}=\SI{12}{\volt} \cdot \SI{2.25}{\ampere}=\SI{27}{\watt}.
        \label{eq:p_bcm}
    \end{equation*}
\end{solutionblock}
    
\subtask{ In which case the step-down converter operates in discontinous conduction mode (DCM) and what is the effect 
          and the potential risk of this mode?}{1}%
\subtaskGerman{In welchem Fall arbeitet der Tiefsetzsteller im Lückbetrieb 
               und welche Auswirkungen sowie potenziellen Risiken hat dieser Betriebsmodus?}

% Solution of subtask
\begin{solutionblock}
    If the power demand is less than $P_\mathrm{load,BCM}$ the output voltage increases. 
    The step-down converter operated in DCM-mode, if the power consumption is less than $P_\mathrm{load,BCM}$. In this case the current through the inductor is $\SI{0}{\ampere}$ for a certain time at the end of a switching period.
    Risk: compared to CCM, this leads to a voltage increase, which could lead to the damage of overvoltage sensitive components.
\end{solutionblock}
    
\subtask{Calculate the switching power loss and the total power loss of the IGBT.}{1}%

\subtaskGerman{Berechnen Sie die Schaltverluste sowie die Gesamtleistungsverluste des IGBTs.}

% Solution of subtask
\begin{solutionblock}
    The power loss of the transistor consists of the power loss while conducting the current and the switching losses.
    \begin{equation*}
        P_\mathrm{T_1}=u_\mathrm{on,CE} I_\mathrm{2} D + \left( E_\mathrm{on}+E_\mathrm{off} \right) \cdot f_\mathrm{s}
        = \SI{2}{\volt} \cdot \SI{70}{\ampere} \cdot 0.25 + \left( \SI{20}{\micro\joule} + \SI{40}{\micro\joule} \right) 
        \cdot \SI{100}{\kilo\hertz} = \SI{41}{\watt}.
        \label{eq:p_IGBT}
    \end{equation*}
\end{solutionblock}



\subtask{Calculate the efficiency $\eta$ of the step-down converter.}{2}

\subtaskGerman{Berechnen Sie den Wirkungsgrad $\eta$ des Tiefsetzstellers.}

% Solution of subtask
\begin{solutionblock}
    The efficiency is calculated by the power demand of the load divided by the total input power.
    The power demand of the load is
    \begin{equation*}
        P_\mathrm{load}= U_\mathrm{2} \cdot I_\mathrm{2} = \SI{12}{\volt} \cdot \SI{70}{\ampere} = \SI{840}{\watt}.
        \label{eq:P_load}
    \end{equation*}
    Taking into account the power demand, the efficiency is calculated by:
    \begin{equation*}
        \eta=\frac{P_\mathrm{load}}{P_\mathrm{total}} = \frac{P_\mathrm{load}}{P_\mathrm{load} + P_\mathrm{T_1}}
            = \frac{\SI{840}{\watt}}{\SI{840}{\watt} + \SI{41}{\watt}}= 0.95.
        \label{eq:eta}
    \end{equation*}
\end{solutionblock}
