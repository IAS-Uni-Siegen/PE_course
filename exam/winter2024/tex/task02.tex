%%%%%%%%%%%%%%%%%%%%%%%%%%%%%%%%%%%%%%%%%%%%%%%%%%%%%%%%%%%%%
%% Examtask 2: Four quadrant converter with pulse width modulation %%
%%%%%%%%%%%%%%%%%%%%%%%%%%%%%%%%%%%%%%%%%%%%%%%%%%%%%%%%%%%%%

\task{Four-quadrant converter with pulse width modulation}

\taskGerman{Vier-Quadranten-Steller mit Pulsweitenmodulation}

% Insipired by Task 2 of the winter 2023/24 exam of Prof. Pacas (cf. \\filer.uni-siegen.de\NAS\IAS\300_Teaching\359_Lecture_notes_Prof_Pacas\Leistungselektronik\Alte Klausuren Leistungselektronik)
% However, there should be some variations to the original task:
% - Add the information that the PWM is interleaving (while not providing a second reference or carrier signal, as this should 
%   be identifed by the students when it comes to the switching pattern) and also not providing a switching table or something
% - add an additional question asking for the conducting / current carrying components during each sub switching interval 
%   (could be added to the plot), that is which transistors/diodes carry the current at each time step
% - change the values of the voltage, inductor and initial current conditions
% - change the nomenclature of the task to the usual definitions of our course


The components of the four quadrant converter according \autoref{fig:Dual_Active_Bridge} are considered as ideal.
The converter's data is displayed in \autoref{table:four_quad_converter}. The inner load voltage is constant: $u_{2\mathrm{i}}(t)=U_{2\mathrm{i}}$.

\begin{germanblock}
Die Komponenten des in \autoref{fig:Dual_Active_Bridge} abgebildeten Vier-Quadranten-Stellers sind als ideal zu betrachten. Dessen Daten sind 
in \autoref{table:four_quad_converter} angegeben. Die innere Lastspannung ist konstant: $u_{2\mathrm{i}}(t)=U_{2\mathrm{i}}$.
\end{germanblock}

% figure figStepDownConverterOutputFilter
%%%%%%%%%%%%%%%%%%%%%%%%%%%%%%%%%%%%%%%%%%%%%%%%%%%%%%%%%%%%%
%% Single-phase DC inverter with inductive filter %%
%%%%%%%%%%%%%%%%%%%%%%%%%%%%%%%%%%%%%%%%%%%%%%%%%%%%%%%%%%%%%

\begin{figure}[htb]
    \begin{center}
        \begin{circuitikz}[american inductors]
            % Reduce scale of diodes
            \ctikzset{diodes/scale=0.5}
            % Add voltage U1p
            \draw (0,0) coordinate (U1) to [open, o-o, v = $U_1\hspace{0.5cm}$, voltage = straight] ++(0,-3.5) coordinate (Gnd)
            % Add current
            (U1) to [short, o-, i=$i_1(t)$] ++(2,0) coordinate (jT1c)
            % Add T1
            (jT1c)  ++(0,-0.75) node[nigbt](T1){} ++(0,-0.75) coordinate (jT1e)
            % Add connection to u2
            (jT1e) to [short, -*] ++(0,-0.25) coordinate (ju2)
            % Add u2 inductor
            (ju2)  to [short, -] ++(2,0) coordinate (jula)          
            % Add wheeling diode junction for DT1
            (jT1c)  to [short, -*] ++(0.5,0)  coordinate (jDT1k)
            (ju2)  to [short, -*] ++(0.5,0)  coordinate (jDT1a)
            % Add wheeling diode for T1
            (jDT1a) to [D, l_=$D_1$] (jDT1k)            
            % Add u2 inductor
            (jula) to [L, l=$L$, name = L] ++(1.25,0) coordinate (ju2p)          
            % Add junction to T2
            (jT1e) to [short] ++(0,-0.5) coordinate (jT2c)
            % Add T2
            (jT2c) ++(0,-0.75) node[nigbt](T2){} ++(0,-0.75) coordinate (jT2e)
            % Add wheeling diode junction for DT2
            (jT2e)  to [short, -*] ++(0.5,0)  coordinate (jDT2a)
            % Add wheeling diode for T2
            (jDT2a) to [D, l_=$D_2$] (jDT1a)            
            % Add connection to T3
            (jT1c) to [short, *-] ++(7,0) coordinate (jT3c)
            % Add T3
            (jT3c) ++(0,-0.75) node[nigbt,xscale=-1](T3){} ++(0,-0.75) coordinate (jT3e)
            % Add junction to ju2
            (jT3e) to [short, -*] ++(0,-0.25) coordinate (jmu2)
            % Add wheeling diode junction for DT3
            (jT3c)  to [short, -*] ++(-0.5,0)  coordinate (jDT3k)
            (jmu2)  to [short, -*] ++(-0.5,0)  coordinate (jDT3a)
            % Add wheeling diode for T3
            (jDT3a) to [D, l=$D_3$] (jDT3k)            
            % Add junction to T4
            (jmu2) to [short] ++(0,-0.25) coordinate (jT4c)
            % Add T4
            (jT4c) ++(0,-0.75) node[nigbt,xscale=-1](T4){} ++(0,-0.75) coordinate (jT4e)
            % Add connection to T2
            (jT4e) to [short, -*] (jT2e)
            % Add wheeling diode junction for DT4
            (jT4e)  to [short, -*] ++(-0.5,0)  coordinate (jDT4a)
            % Add wheeling diode for T2
            (jDT4a) to [D, l=$D_4$] (jDT3a)            
            % Add connection to Gnd U1
            (jT2e) to [short, -] (Gnd)
            % Add u2e
            (ju2p) to [sV=$u_{2\mathrm{i}}(t)$] ++(2,0) coordinate (ju2m)
            % Add connection
            (ju2m) to [short, -] (jmu2);


            % Add component name of transistors
            \draw let \p1 = (T1.B) in node[anchor=east] at (\x1,\y1) {$T_1$};
            \draw let \p1 = (T2.B) in node[anchor=east] at (\x1,\y1) {$T_2$};
            \draw let \p1 = (T3.B) in node[anchor=west] at (\x1,\y1) {$T_3$};
            \draw let \p1 = (T4.B) in node[anchor=west] at (\x1,\y1) {$T_4$};
            % Add current arrows i2
            \draw (ju2) ++(1.5,0) node[currarrow](i2){}
            (i2)  node[anchor=south,color=black]{$i_\mathrm{2}(t)$}
            % Add voltage arrows u2a
            (ju2) ++(1.4,-0.25) to [open,v^=$$,voltage = straight] ++(0,-1.4)
            (ju2) ++ (2.0,-0.6) node[anchor=north,color=black]{$u_\mathrm{2a}(t)$}
            % Add voltage arrows u2b
            (jmu2) ++(-1.4,-0.25) to [open,v^=$$,voltage = straight] ++(0,-1.4)
            (jmu2) ++ (-2.0,-0.6) node[anchor=north,color=black]{$u_\mathrm{2b}(t)$}
            % Add voltage arrows u2ab
            (jmu2) ++(-1.2,-0.5) coordinate (jmu2x)
            (ju2) ++(1.2,-0.5)  to [open,v^=$$,voltage = straight] (jmu2x)
            (ju2) ++ (3.5,-0.5) node[anchor=north,color=black]{$u_\mathrm{2ab}(t)$};


            % (ju2x) ++(0,-0.8) to [open,v^=$u_\mathrm{2}(t)$,voltage = straight] ++(3.8,0);
        \end{circuitikz}
    \end{center}
    \caption{Four-quadrant converter.}
    \label{fig:Dual_Active_Bridge}
\end{figure}



\begin{table}[ht]
    \centering  % Zentriert die Tabelle
    \begin{tabular}{llll}
        \toprule
        Input voltage: &  $U_{\mathrm{1}} = \SI{450}{\volt}$ & Current at $t=\SI{0}{\second}$: & $i_{\mathrm{2}}(0) = \SI{30}{\ampere}$ \\ 
        Inner voltage: & $U_{\mathrm{2i}} = \SI{150}{\volt}$  & Inductance: & $L = \SI{60}{\micro\henry}$ \\ 
        \bottomrule
    \end{tabular}
    \caption{Parameters of the four-quadrant converter.}  % Beschriftung der Tabelle
    \label{table:four_quad_converter}
\end{table}

The four quadrant converter is controlled by a PWM with interleaved switching. $T_1$ is connected to the  output of the PWM, 
which uses the non-inverted reference voltage. On the other hand, $T_3$ is connected to the output of the PWM, 
which uses the inverted reference voltage.
The triangular carrier voltage and the reference voltage are displayed in the the solution diagram.
Note that the reference signal changes at $t=\SI{24}{\micro\second}$ and $t=\SI{36}{\micro\second}$.

\vspace{1em}
\color{gray}
Der Vier-Quadranten-Steller wird durch eine PWM mit phasenversetztem Schalten gesteuert.
Der Ausgang der Steuerung, der die nicht invertierte Referenzspannung verwendet, ist mit $T_1$ verbunden, während der Ausgang, der die invertierte Referenzspannung verwendet, mit $T_3$ verbunden ist.
Der Dreiecksträger und die Referenzspannung werden im Lösungsdiagramm dargestellt.
Beachten Sie, dass das Referenzsignal bei $t=\SI{24}{\micro\second}$ and $t=\SI{36}{\micro\second}$ wechselt.
\color{black}

\subtask{Add the voltage signals of $u_{\mathrm{a}(t)}$, $u_{\mathrm{b}(t)}$ and $u_{\mathrm{ab}(t)}$ to the diagram in \autoref{fig:solutionfigureTemplateTask2}
and and complete the axis scaling of the ordinates.}{3}

\subtaskGerman{Zeichnen Sie $u_{\mathrm{a}(t)}$, $u_{\mathrm{b}(t)}$ and $u_{\mathrm{ab}(t)}$ in das Lösungsblatt \autoref{fig:solutionfigureTemplateTask2} ein
 und ergänzen Sie die Achsenskalierung der Ordinaten.}
 
\subtask{Calculate the current signals $i_\mathrm{2}(t)$ and $i_\mathrm{1}(t)$ and add them to template diagram.}{4}
\begin{hintblock}
    If you are not able to exactly calculate the current signals, you can qualitatively add them to the template diagram for partial points.
\end{hintblock}

\subtaskGerman{Berechnen Sie die Ströme von $i_\mathrm{2}(t)$ und $i_\mathrm{1}(t)$ und zeichnen Sie diese in Vorlage ein.}
\begin{germanhintblock}
    Wenn Sie die Ströme nicht exakt berechnen können, können Sie diese qualitativ in die Vorlage für Teilpunkte eintragen.
\end{germanhintblock}

% Solution of subtask
\begin{solutionblock}
    The voltage at the inductor $L$ is calculated by
    \begin{equation*}
        u_\mathrm{L}(t) = u_\mathrm{2ab}(t)-U_\mathrm{2i}.
        \label{eq:u_inductor}
    \end{equation*}
    The slope of the current is determined by the voltage at the inductance.
    \begin{equation*}
        \frac{\mathrm{d}}{\mathrm{d}t}i_\mathrm{2}(t) = \frac{1}{L}u_\mathrm{L}(t).
        \label{eq:i2_slope_gen}
    \end{equation*}
    Using the two equations above for the slope of $i_\mathrm{2}(t)$ at the three different voltages, we obtain:
    \begin{equation*}
        \frac{\mathrm{d}}{\mathrm{d}t}i_\mathrm{2}(t) =
        \begin{cases}
            u_\mathrm{2ab}(t)=\SI{450}{\volt}:  &\frac{\mathrm{d}}{\mathrm{d}t}i_\mathrm{2}(t)
            =\frac{1}{\SI{60}{\micro\henry}}(\SI{450}{\volt}-\SI{150}{\volt})=\SI{5}{\frac{\ampere}{\micro\second}}, \\
            u_\mathrm{2ab}(t)=\SI{0}{\volt}:  &\frac{\mathrm{d}}{\mathrm{d}t}i_\mathrm{2}(t)
            =\frac{1}{\SI{60}{\micro\henry}}(-\SI{150}{\volt})=\SI{-2.5}{\frac{\ampere}{\micro\second}}, \\
            u_\mathrm{2ab}(t)=\SI{-450}{\volt}: &\frac{\mathrm{d}}{\mathrm{d}t}i_\mathrm{2}(t)
            =\frac{1}{\SI{60}{\micro\henry}}(\SI{-450}{\volt}-\SI{150}{\volt})=\SI{-10}{\frac{\ampere}{\micro\second}}.
        \end{cases}
        \label{eq:i2_slope_res}
    \end{equation*}

    The voltage at the inductor $L$ is constant for discrete time intervals. So $i_\mathrm{2}(t)$ is calculated by
    \begin{equation*}
        \begin{aligned}        
            &i_\mathrm{2}(t) = i_\mathrm{2}(\tau)+\frac{\mathrm{d}}{\mathrm{d}t}i_\mathrm{2}(\tau) \cdot (t-\tau) \\
            &\text{with $\tau$ is start of a constant time interval.}
        \end{aligned}
        \label{eq:i2_res}
    \end{equation*}
    The current $i_\mathrm{1}(t)$ corresponds to:
    \begin{equation*}
        i_\mathrm{1}(t) =
        \begin{cases}
            \SI{0}{\ampere} \quad &u_\mathrm{2ab}(t) = \SI{0}{\volt} \\
            i_\mathrm{2}(t) \quad &u_\mathrm{2ab}(t) \neq \SI{0}{\volt}.
        \end{cases}        
    \end{equation*}    
    These results are used to sketch $i_\mathrm{2}(t)$ and $i_\mathrm{1}(t)$ within the solution sheet.    

\end{solutionblock}

\subtask{Mark which semiconductors carry the current $i_\mathrm{2}(t)$ in the template diagram. For example, if $T_\mathrm{1}$ and $D_\mathrm{4}$ conduct 
 during the range $\SI{10}{\micro\second}$ and $\SI{14}{\micro\second}$, the range must be marked with 2 vertical 
lines and $T_\mathrm{1}/D_\mathrm{4}$ entered there.}{3}

\subtaskGerman{Kennzeichnen Sie die jeweils leitenden Halbleiterkomponenten, welche den Strom $i_\mathrm{2}(t)$ führen. Wenn beispielsweise $T_\mathrm{1}$ und $D_\mathrm{4}$ den Strom $i_\mathrm{2}(t)$ innerhalb
des Bereichs $\SI{10}{\micro\second}$ und $\SI{14}{\micro\second}$ leiten, ist der Bereich mit 2 vertikalen Linien zu markieren und 
in diesen $T_\mathrm{1}/D_\mathrm{4}$ einzutragen.}


\input{fig/Task02/FigTask2Template}

% Solution of subtasks
\begin{solutionblock}
\input{fig/Task02/sFigTask2}
\end{solutionblock}
