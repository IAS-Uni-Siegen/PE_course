\section{DC-DC converters}
\title{DC-DC converters}  

\begin{frame}[plain]
    \titlepage
\end{frame}

%%%%%%%%%%%%%%%%%%%%%%%%%%%%%%%%%%%%%%%%%%%%%%%%%%%%%%%%%%%%%
%% Step-down converter %%
%%%%%%%%%%%%%%%%%%%%%%%%%%%%%%%%%%%%%%%%%%%%%%%%%%%%%%%%%%%%%
\subsection{Step-down converter}

%%%%%%%%%%%%%%%%%%%%%%%%%%%%%%%%%%%%%%%%%%%%%%%%%%%%%%%%%%%%%
%% Overview and assumption %%
%%%%%%%%%%%%%%%%%%%%%%%%%%%%%%%%%%%%%%%%%%%%%%%%%%%%%%%%%%%%%

\begin{frame}[b]
\frametitle{Overview and assumption}
    We consider the following assumptions:
    \begin{itemize}
        \item The switch is ideal, that is, infinitely fast.
        \item The input voltage is constant: $u_\mathrm{in}(t) = U_\mathrm{in}$.
        \item The output voltage is constant: $u_\mathrm{out}(t) = U_\mathrm{out}$.
        \item The input voltage is greater than the output voltage: $U_\mathrm{in} > U_\mathrm{out}$.
    \end{itemize}
    \begin{figure}
        \begin{circuitikz}[]
            \draw (0,2) to [open, o-o, v = $\hspace{2cm}u_\mathrm{out}(t)$, voltage = straight] ++(0,-2)
            to ++(-6,0)
            to [open, o-o, v<= $u_\mathrm{in}(t) \hspace{2cm}$, voltage = straight] ++(0,2)
            (-6,2) to  [short, i=$i_\mathrm{in}(t)$] ++(1.35,0);   
            \draw (-5.375,2) ++(0.625,0) node [cuteopenswitchshape, anchor = out, rotate=180] (S) {}
            let \p1 = (S.mid) in (S.in) to  [short, i=$i_\mathrm{L}(t)$] ++(1,0)
            to [inductor, l=$L$, v = $u_\mathrm{L}(t)$, voltage = straight] ++(2,0)
            to [short, i=$i_\mathrm{out}(t)$] (0,2) 
            ([yshift = -0.3cm]S.mid) to [short, o-*](\x1,0);
            \draw (-3.5,2) to [open, v = $\hspace{1.75cm}u_\mathrm{s}(t)$, voltage = straight] ++(0,-2);
        \end{circuitikz}
        \caption{Step-down converter (ideal switch representation)}
        \label{fig:step-down-converter-simple}
    \end{figure}
\end{frame}

%%%%%%%%%%%%%%%%%%%%%%%%%%%%%%%%%%%%%%%%%%%%%%%%%%%%%%%%%%%%%
%% Switch states %%
%%%%%%%%%%%%%%%%%%%%%%%%%%%%%%%%%%%%%%%%%%%%%%%%%%%%%%%%%%%%%

\begin{frame}[b]
    \frametitle{Switch states}
     The voltage at the switch is given by
     \begin{equation}
            u_\mathrm{s}(t) = \begin{cases}
                U_\mathrm{in}, & t\in [k T_\mathrm{s}, k T_\mathrm{s} + T_\mathrm{e}],\\
                0, & t\in [k T_\mathrm{s}+ T_\mathrm{e}, (k+1) T_\mathrm{s}]
            \end{cases}
     \end{equation}
     with $k\in\mathbb{N}$ being the $k$-th switching period, $T_\mathrm{s}$ the switching period, and $T_\mathrm{e}$ the switch-on time. 
        \begin{figure}
            \centering	
            \begin{subfigure}{0.45\textwidth}
                \centering
                \hspace{-0.75cm}
                \begin{circuitikz}[]
                    \draw (0,2) to [open, o-o, v = $\hspace{2cm}U_\mathrm{out}$, voltage = straight] ++(0,-2)
                    to ++(-5,0)
                    to [open, o-o, v<= $U_\mathrm{in} \hspace{2cm}$, voltage = straight] ++(0,2)
                    (-5,2) to  [short, i=${i_\mathrm{in}(t)=i_\mathrm{L}(t)}$] ++(2,0)
                    to [inductor, l=$L$, v = $u_\mathrm{L}(t)$, voltage = straight] ++(2,0)
                    to [short, i=$i_\mathrm{out}(t)$] (0,2);
                    \draw (-4,2) to [open, v = ${\hspace{2.4cm}U_\mathrm{s}=U_\mathrm{in}}$, voltage = straight] ++(0,-2);
                \end{circuitikz}
                \caption{Switch-on time}
            \end{subfigure}%
            \hspace{0.5cm}
            \begin{subfigure}{0.45\textwidth}
                \centering
                \begin{circuitikz}[]
                    \draw (0,2) to [open, o-o, v = $\hspace{2cm}U_\mathrm{out}$, voltage = straight] ++(0,-2)
                    to ++(-5,0)
                    to [open, o-o, v<= $U_\mathrm{in} \hspace{2cm}$, voltage = straight] ++(0,2)
                    (-5,2) to  [short, i=${i_\mathrm{in}=0}$] ++(0.75,0)
                    (-3.75,0) to [short, *-] ++(0,2)
                    to [short, i=$i_\mathrm{L}(t)$] ++(0.75,0)
                    to [inductor, l=$L$, v = $u_\mathrm{L}(t)$, voltage = straight] ++(2,0)
                    to [short, i=$i_\mathrm{out}(t)$] (0,2);
                    \draw (-3.4,2) to [open, v = ${\hspace{2.4cm}U_\mathrm{s}=0}$, voltage = straight] ++(0,-2);
                \end{circuitikz}
                \caption{Switch-off time}
            \end{subfigure}
        \caption{Switch states of the step-down converter} 
        \label{fig:step-down-converter-switch-states}
        \end{figure}
    \end{frame}

%%%%%%%%%%%%%%%%%%%%%%%%%%%%%%%%%%%%%%%%%%%%%%%%%%%%%%%%%%%%%
%% Basic terms and definitions %%
%%%%%%%%%%%%%%%%%%%%%%%%%%%%%%%%%%%%%%%%%%%%%%%%%%%%%%%%%%%%%
\begin{frame}[c]
    \frametitle{Basic terms and definitions}
    \centering
     \begin{tabular}{c l c l}
        $T_\mathrm{on}$ & Switch-on time  & $T_\mathrm{off}$ & Switch-off time\\[1em]
        $T_\mathrm{s}= T_\mathrm{on} + T_\mathrm{off}$ & Switching period & $f_\mathrm{s} = 1/T_\mathrm{s}$ & Switching frequency\\[1em]
        $D = T_\mathrm{on}/T_\mathrm{s}$ & Duty cycle &  & 
     \end{tabular}
     \begin{figure}
        \begin{tikzpicture}
            \begin{axis}[
                xlabel={$t/T_\mathrm{s}$},
                ylabel={$u_\mathrm{s}(t)/U_\mathrm{in}$},
                ymin=-0.05, ymax=1.05,
                xmin=-0.1, xmax=1.1,
                width = 0.5\textwidth,
                height = 0.4\textheight,
                grid,
                thick,
                clip = false,
                ]
                % plt u_out = 1 for t = 0 to D*Ts and u_out = 0 for t = D*Ts to Ts in a single plot
                \addplot[signalblue] coordinates {(-0.1,0) (0,0) (0,1) (0.4,1) (0.4,0) (1,0) (1,1) (1.1,1)};
                \draw [thick,<->]  (0,0.5) -- node[above,fill=white]{$T_\mathrm{on}$}(0.4, 0.5); 
                \draw [thick,<->]  (0.4,0.5) -- node[above]{$T_\mathrm{off}$}(1.0, 0.5);
                \draw [thick,<->]  (0.0,1.2) -- node[above]{$T_\mathrm{s}$}(1.0, 1.2); 
            \end{axis}
        \end{tikzpicture}
    \end{figure}
    \end{frame}

%%%%%%%%%%%%%%%%%%%%%%%%%%%%%%%%%%%%%%%%%%%%%%%%%%%%%%%%%%%%%
%% Steady-state analysis %%
%%%%%%%%%%%%%%%%%%%%%%%%%%%%%%%%%%%%%%%%%%%%%%%%%%%%%%%%%%%%%
\begin{frame}
    \frametitle{Steady-state analysis}
    The inductor current from \figref{fig:step-down-converter-simple} is represented by the \hl{differential equation}
    \begin{equation}
        L \frac{\mathrm{d}i_\mathrm{L}(t)}{\mathrm{d}t} =  u_\mathrm{L}(t) =  u_\mathrm{s}(t) - U_\mathrm{out}.
    \end{equation}
    During the \hl{switch-on period} we have
    \begin{equation}
        i_\mathrm{L}(t) = i_\mathrm{L}(k T_\mathrm{s}) + \frac{U_\mathrm{in}-U_\mathrm{out}}{L} (t - k T_\mathrm{s}), \quad t\in [k T_\mathrm{s}, k T_\mathrm{s} + T_\mathrm{on}]
        \label{eq:inductor-current-switch-on-simple-step-down}
    \end{equation}
    and during the \hl{switch-off period} we receive
    \begin{equation}
        \begin{split}
            i_\mathrm{L}(t) &= i_\mathrm{L}(k T_\mathrm{s} + T_\mathrm{on}) - \frac{U_\mathrm{out}}{L} (t - k T_\mathrm{s} - T_\mathrm{on})\\
            \               &= i_\mathrm{L}(k T_\mathrm{s}) + \frac{U_\mathrm{in}-U_\mathrm{out}}{L} T_\mathrm{on} - \frac{U_\mathrm{out}}{L} (t - k T_\mathrm{s} - T_\mathrm{on}), \quad 
            t\in [k T_\mathrm{s} + T_\mathrm{on}, (k+1) T_\mathrm{s}]. 
        \end{split} 
        \label{eq:inductor-current-switch-off}
    \end{equation}
\end{frame}

%%%%%%%%%%%%%%%%%%%%%%%%%%%%%%%%%%%%%%%%%%%%%%%%%%%%%%%%%%%%%
%% Steady-state analysis (cont.) %%
%%%%%%%%%%%%%%%%%%%%%%%%%%%%%%%%%%%%%%%%%%%%%%%%%%%%%%%%%%%%%
\begin{frame}
    \frametitle{Steady-state analysis (cont.)}
    In steady state the inductor current is periodic with period $T_\mathrm{s}$, that is, $$i_\mathrm{L}(t) = i_\mathrm{L}(t + T_\mathrm{s}).$$
    From \eqref{eq:inductor-current-switch-off} we obtain for $t= k T_\mathrm{s}$
    \begin{equation}
        \begin{split}
        \overbrace{i_\mathrm{L}(k T_\mathrm{s})}^{\mbox{Start of period}} &= \hspace{0.5cm}  \overbrace{i_\mathrm{L}(k T_\mathrm{s}) + \frac{U_\mathrm{in}-U_\mathrm{out}}{L} T_\mathrm{on} - \frac{U_\mathrm{out}}{L} (T_\mathrm{s} - T_\mathrm{on})}^{\mbox{End of period}}\\
        \Leftrightarrow \qquad 0 \hspace{1.1cm}&= \hspace{0.5cm}\frac{U_\mathrm{in}-U_\mathrm{out}}{L} T_\mathrm{on} - \frac{U_\mathrm{out}}{L} (T_\mathrm{s} - T_\mathrm{on})\\
        \Leftrightarrow \qquad 0 \hspace{1.1cm}&= \hspace{0.5cm}U_\mathrm{in} T_\mathrm{on} - U_\mathrm{out} T_\mathrm{s}.
    \end{split}
    \end{equation}
    Rewritting delivers the \hl{output voltage} as
    \begin{equation}
        U_\mathrm{out} = \frac{T_\mathrm{on}}{T_\mathrm{s}} U_\mathrm{in} = D U_\mathrm{in}.
    \end{equation}
\end{frame}

%%%%%%%%%%%%%%%%%%%%%%%%%%%%%%%%%%%%%%%%%%%%%%%%%%%%%%%%%%%%%
%% Steady-state time-domain behavior %%
%%%%%%%%%%%%%%%%%%%%%%%%%%%%%%%%%%%%%%%%%%%%%%%%%%%%%%%%%%%%%
\begin{frame}[fragile]
    \frametitle{Steady-state time-domain behavior}
    \begin{figure}
        \begin{tikzpicture}
            \pgfmathsetmacro{\D}{0.6} % duty cycle
            \begin{groupplot}[group style={group size=1 by 3, xticklabels at = edge bottom}, height=0.34\textheight, width=0.875\textwidth, xmin=0, xmax=4, grid,clip = false, ymin = 0, ymax =1.1]

                % Top plot: voltage at the switch
                \nextgroupplot[ylabel = {$u_\mathrm{s}(t)$}, ytick = {0, 0.5, 1}, yticklabels = {0, , $U_1$}]
                    \pgfplotsinvokeforeach{0,...,3}{
                        \edef\AddPlot{\noexpand\addplot[signalblue, thick] coordinates {({0 + #1},0) ({0 + #1},1) ({\D + #1},1) ({\D + #1},0) ({1 + #1},0) ({1 + #1},1)};}
                        \AddPlot
                    }
                    \draw[signalblue, thick, dashed] (axis cs:0, \D) -- (axis cs:4, \D); % dashed line at U_2 (average)
                    \node[above, inner sep = 2pt, anchor = south] at (axis cs:1.5+\D/2, \D) {$U_2$}; % label U_2
                    \draw [thick,<->]  (0,0.5) -- node[below]{$T_\mathrm{on}$}(\D, 0.5); % T_on 
                    \draw [thick,<->]  (\D,0.5) -- node[below]{$T_\mathrm{off}$}(1.0, 0.5); % T_off
                    \draw [thick,<->]  (0.0,-0.2) -- node[below]{$T_\mathrm{s}$}(1.0, -0.2); % T_s 


                % Middle plot: inductor current
                \nextgroupplot[ylabel = {$i_\mathrm{L}(t)$}, ytick = {0, 0.5, 1}, yticklabels = {0, $\overline{i}_\mathrm{L}$, }]
                    \pgfplotsinvokeforeach{0,...,3}{
                        \edef\AddPlot{\noexpand\addplot[signalred, thick] coordinates {({0 + #1},0.25) ({\D + #1},0.75) ({1 + #1},0.25)};}
                        \AddPlot
                    }
                    \draw[signalred, thick, dashed] (axis cs:0,0.5) -- (axis cs:4,0.5); % dashed line at average current
                    \draw[{Latex[length=2mm]}-, thin] (axis cs:\D+0.02,0.75) -- node[right=1mm, fill=white, inner sep = 1pt]{$\max\{i_\mathrm{L}\}$}(axis cs:\D+0.3,0.9); % indicate max current
                    \draw[-{Latex[length=2mm]}, thin] (axis cs:0.75,0.2) node[right=1mm, fill=white, inner sep = 1pt, anchor = east]{$\min\{i_\mathrm{L}\}$} -- (axis cs:1-0.02,0.25); % indicate min current
                    \draw[thin] (axis cs:2+\D/4,0.25+0.125) -- (axis cs:2+\D/4,0.75-0.125) -- (axis cs:2+\D*3/4,0.75-0.125); % indicate positive current slopde
                    \node[above, inner sep = 2pt, anchor = south] at (axis cs:2+\D/2, 0.75-0.125) {$\nicefrac{U_1-U_2}{L}$}; % label positive current slope
                    \draw[thin] (axis cs:2.25+3*\D/4,0.75-0.125) -- (axis cs:2.75+\D/4,0.75-0.125) -- (axis cs:2.75+\D/4,0.25+0.125); % indicate negative current slope
                    \node[above, inner sep = 2pt, anchor = south] at (axis cs:2.5+\D/2, 0.75-0.125) {$\nicefrac{-U_2}{L}$}; % label negative current slope
                
                % Bottom plot: input current
                \nextgroupplot[ylabel = {$i_\mathrm{in}(t)$}, xlabel={$t/T_\mathrm{s}$}, ytick = {0, 0.5, 1}, yticklabels = {0, $\overline{i}_\mathrm{L}$}]
                    \pgfplotsinvokeforeach{0,...,3}{
                        \edef\AddPlot{\noexpand\addplot[signalred, thick] coordinates {({0 + #1},0.25) ({\D + #1},0.75) ({\D + #1},0) ({1 + #1},0) ({1 + #1},0.25)};}
                        \AddPlot
                    }
                    \draw[signalred, thick, dashed] (axis cs:0,0.5 * \D) -- (axis cs:4, 0.5 * \D); % dashed line at average current
                    \node[above, inner sep = 2pt, anchor = south, fill = white] at (axis cs:1.5+\D/2, 0.5 * \D) {$\overline{i}_\mathrm{in}$}; % label average current
            \end{groupplot}
        \end{tikzpicture}
    \end{figure}
\end{frame}

%%%%%%%%%%%%%%%%%%%%%%%%%%%%%%%%%%%%%%%%%%%%%%%%%%%%%%%%%%%%%
%% Alternative steady-state analysis: average values %%
%%%%%%%%%%%%%%%%%%%%%%%%%%%%%%%%%%%%%%%%%%%%%%%%%%%%%%%%%%%%%
\begin{frame}
    \frametitle{Alternative steady-state analysis: average values}
    From the previous slide we know that the average inductor voltage is zero in steady state
    \begin{equation}
        \overline{u}_\mathrm{L} = \frac{1}{T_\mathrm{s}} \int_{0}^{T_\mathrm{s}} u_\mathrm{L}(t) \mathrm{d}t = \overline{u}_\mathrm{L} = 0,
    \end{equation}
    since otherwise the average inductor current would change between periods, compare
    $$
    L \frac{\mathrm{d}i_\mathrm{L}(t)}{\mathrm{d}t} =  u_\mathrm{L}(t).
    $$
    From \figref{fig:step-down-converter-simple} we can apply Kirchhoff's voltage law to obtain
    \begin{equation}
        u_\mathrm{s}(t) = u_\mathrm{L}(t) + u_\mathrm{out}(t) \quad \Rightarrow \quad \overline{u}_\mathrm{s} = \overline{u}_\mathrm{L} + U_\mathrm{out} \quad \Leftrightarrow \quad  \overline{u}_\mathrm{s} = U_\mathrm{out}.
        \label{eq:Kirchhoff-switch-voltage}
    \end{equation}
    The average switch voltage is given by
    \begin{equation}
        \overline{u}_\mathrm{s} = \frac{1}{T_\mathrm{s}} \int_{0}^{T_\mathrm{s}} u_\mathrm{s}(t) \mathrm{d}t = \frac{1}{T_\mathrm{s}} \int_{0}^{T_\mathrm{on}} U_\mathrm{in} \mathrm{d}t = U_\mathrm{in} \frac{T_\mathrm{on}}{T_\mathrm{s}} = U_\mathrm{in} D.
        \label{eq:average-switch-voltage}
    \end{equation}
\end{frame}

%%%%%%%%%%%%%%%%%%%%%%%%%%%%%%%%%%%%%%%%%%%%%%%%%%%%%%%%%%%%%
%% Alternative steady-state analysis: average values (cont.) %%
%%%%%%%%%%%%%%%%%%%%%%%%%%%%%%%%%%%%%%%%%%%%%%%%%%%%%%%%%%%%%
\begin{frame}
    \frametitle{Alternative steady-state analysis: average values (cont.)}
    Combining \eqref{eq:Kirchhoff-switch-voltage} and \eqref{eq:average-switch-voltage} we obtain
    \begin{equation}
        \frac{U_\mathrm{out}}{U_\mathrm{in}} =  D.
    \end{equation}
    In addition, we can calculate the average input current as
    \begin{equation}
        \overline{i}_\mathrm{in} = \frac{1}{T_\mathrm{s}} \int_{0}^{T_\mathrm{s}} i_\mathrm{in}(t) \mathrm{d}t = \frac{1}{T_\mathrm{s}} \int_{0}^{T_\mathrm{on}} i_\mathrm{L}(t) \mathrm{d}t = \frac{T_\mathrm{on}}{T_\mathrm{s}} \overline{i}_\mathrm{L} = D \overline{i}_\mathrm{L}.
    \end{equation}
    Since $i_\mathrm{out}(t) = i_\mathrm{L}(t)$ applies, we can conclude
    \begin{equation}
        \frac{U_\mathrm{out}}{U_\mathrm{in}} = \frac{\overline{i}_\mathrm{in}}{\overline{i}_\mathrm{out}} = D.
    \end{equation}
    Hence, the duty cycle $D$ has a similar interpretation for the DC-DC step-down converter as the turn ratio for an ideal transformer in the AC domain.
\end{frame}

%%%%%%%%%%%%%%%%%%%%%%%%%%%%%%%%%%%%%%%%%%%%%%%%%%%%%%%%%%%%%
%% Stationary averaged model of the step-down converter %%
%%%%%%%%%%%%%%%%%%%%%%%%%%%%%%%%%%%%%%%%%%%%%%%%%%%%%%%%%%%%%
\begin{frame}
    \frametitle{Stationary averaged model of the step-down converter}
    \begin{figure}
        \begin{circuitikz}[]
            \draw (-3,0) to [short, o-, i=$\overline{i}_\mathrm{in}$] ++(1.5,0)
            to [current source, i=$D \overline{i}_\mathrm{out}$] ++(0,-2)
            to [short, -o] ++(-1.5,0);
            \draw (-3,0) to [open, v = $\hspace{-0.25cm}U_\mathrm{in}$, voltage = straight] ++(0,-2);
            \draw (3,0) to [short, o-, i_<=$\overline{i}_\mathrm{out}$] ++(-1.5,0)
            to [voltage source, v_= $D U_\mathrm{in}$] ++(0,-2)
            to [short, -o] ++(1.5,0);
            \draw (3,0) to [open, v = $\hspace{1.5cm}U_\mathrm{out}$, voltage = straight] ++(0,-2);
        \end{circuitikz}
        \caption{Stationary averaged model of the step-down converter}
        \label{fig:step-down-converter-averaged}
    \end{figure}
    \begin{varblock}{Switching vs. linear power conversion}
        In contrast to the linear power conversion approaches from \figref{fig:linear_power_conversion} and \figref{fig:linear_power_conversion_transistor}, the switching step-down converter transforms the current and voltage levels with the same factor $D$ which results from the (idealized) loss-less transformation of energy. 
    \end{varblock}
\end{frame}

%%%%%%%%%%%%%%%%%%%%%%%%%%%%%%%%%%%%%%%%%%%%%%%%%%%%%%%%%%%%%
%% Current ripple %%
%%%%%%%%%%%%%%%%%%%%%%%%%%%%%%%%%%%%%%%%%%%%%%%%%%%%%%%%%%%%%
\begin{frame}
    \frametitle{Current ripple}
    Due to the switching operation of the step-down converter, the inductor current exhibits an inherent ripple. The peak-to-peak \hl{current ripple} is given by
    \begin{equation}
        \begin{split}
            \Delta i_\mathrm{L} &= \max\{i_\mathrm{L}(t)\} - \min\{i_\mathrm{L}(t)\}  = i_\mathrm{L}(t=T_\mathrm{on}) - i_\mathrm{L}(t=T_\mathrm{s})\\
                                &=\frac{U_\mathrm{in} - U_\mathrm{out}}{L} T_\mathrm{on} = \frac{U_\mathrm{out}}{L} T_\mathrm{off}\\
                                & = \frac{D(1-D)T_\mathrm{s}}{L}U_\mathrm{in}.
        \end{split}
        \label{eq:current-ripple-simple-step-down}
    \end{equation}
    The current ripple has two main implications:
    \begin{itemize}
        \item The output power is not constant but varies with the current ripple.
        \item The root mean square (RMS) current is higher than the average current.
    \end{itemize}
    The latter point should be investigated in more detail as it influences the design and loss characteristics of the converter.
\end{frame}

%%%%%%%%%%%%%%%%%%%%%%%%%%%%%%%%%%%%%%%%%%%%%%%%%%%%%%%%%%%%%
%% Current ripple (cont.) %%
%%%%%%%%%%%%%%%%%%%%%%%%%%%%%%%%%%%%%%%%%%%%%%%%%%%%%%%%%%%%%
\begin{frame}
    \frametitle{Current ripple (cont.)}
    We define 
    \begin{equation}
        \Delta I_\mathrm{L} = \sqrt{\frac{1}{T_\mathrm{s}} \int_{0}^{T_\mathrm{s}} \left(i_\mathrm{L}(t) - \overline{i}_\mathrm{L}\right)^2 \mathrm{d}t}
    \end{equation}
    as the RMS deviation of the inductor current from its average value. As the average-corrected inductor current has a triangular shape (cf. \figref{fig:inductor-current-ripple}) we can calculate the RMS current as
    \begin{equation}
        \Delta I_\mathrm{L} = \frac{1}{\sqrt{3}} \frac{\Delta i_\mathrm{L}}{2} = \frac{D(1-D)T_\mathrm{s}U_\mathrm{in}}{2\sqrt{3}L}.
        \label{eq:RMS-current-ripple}
    \end{equation}
    \begin{figure}
        \begin{tikzpicture}
            \pgfmathsetmacro{\D}{0.6} % duty cycle
            \begin{axis}[
                xlabel={$t/T_\mathrm{s}$},
                ylabel={$i_\mathrm{L}(t)-\overline{i}_\mathrm{L}$},
                ymin=-0.3, ymax=0.3,
                xmin=-0.1, xmax=1.1,
                width = 0.7\textwidth,
                height = 0.4\textheight,
                grid,
                thick,
                clip = true,
                ytick = {-0.2, 0, 0.2}, 
                yticklabels = {$\nicefrac{-\Delta i_\mathrm{L}}{2}$, 0, $\nicefrac{\Delta i_\mathrm{L}}{2}$}
                ]
                \addplot[signalred] coordinates {(\D-1,0.22) (0,-0.2) (\D,0.2) (1, -0.2) (1+\D,0.2)};
                \draw[signalred, thick, dashed] (axis cs:\D-1,0.2/1.73) -- (axis cs:1+\D, 0.2/1.73); % dashed line at RMS current
                \node[above, inner sep = 2pt, anchor = north, fill = white] at (axis cs:0.1, 0.2/1.73) {$\Delta I_\mathrm{L}$};
                \draw [thin, <->]  (0.6,-0.2) -- node[left,fill=white]{$\Delta i_\mathrm{L}$}(0.6, 0.2); 
            \end{axis}
        \end{tikzpicture}
        \caption{Inductor current ripple}
        \label{fig:inductor-current-ripple}
    \end{figure}
\end{frame}

%%%%%%%%%%%%%%%%%%%%%%%%%%%%%%%%%%%%%%%%%%%%%%%%%%%%%%%%%%%%%
%% Current ripple (cont.) %%
%%%%%%%%%%%%%%%%%%%%%%%%%%%%%%%%%%%%%%%%%%%%%%%%%%%%%%%%%%%%%
\begin{frame}
    \frametitle{Current ripple (cont.)}
    The (total) RMS value of the inductor current (triangular signal with offset) is given by
    \begin{equation}
        I_\mathrm{L} = \sqrt{\overline{i}^2_\mathrm{L} + \Delta I^2_\mathrm{L}}.
    \end{equation}
    Considering the internal resistance $R_\mathrm{i}$ of the inductor, the ohmic power loss in the inductor is
    \begin{equation}
        P_\mathrm{L} = R_\mathrm{i} I^2_\mathrm{L} = R_\mathrm{i} \left(\overline{i}^2_\mathrm{L} + \Delta I^2_\mathrm{L}\right).
    \end{equation}
    The power loss in the inductor is thus composed of a constant part $\overline{P}_\mathrm{L} = R_\mathrm{i} \overline{i}^2_\mathrm{L}$, which is related to the power transfer from input to output, and a ripple part $\Delta P_\mathrm{L} = R_\mathrm{i} \Delta I^2_\mathrm{L}$.   
    \begin{varblock}{Current ripple and power losses}
        The current ripple produces additional losses in the inductor. From \eqref{eq:RMS-current-ripple} it seems tempting to increase the switching frequency $f_\mathrm{s}$ to reduce the ripple, but, this will increase switching losses (compare \figref{fig:idealized_switch_model}). Hence, there is a trade-off decision between switching and conduction losses.
    \end{varblock}
\end{frame}

%%%%%%%%%%%%%%%%%%%%%%%%%%%%%%%%%%%%%%%%%%%%%%%%%%%%%%%%%%%%%
%% Current ripple and duty cycle %%
%%%%%%%%%%%%%%%%%%%%%%%%%%%%%%%%%%%%%%%%%%%%%%%%%%%%%%%%%%%%%
\begin{frame}
    \frametitle{Current ripple and duty cycle}
    Rewriting the current ripple expression
    \begin{equation*}
            \Delta i_\mathrm{L} = \frac{D(1-D)T_\mathrm{s}}{L}U_\mathrm{in} = (D-D^2)\frac{T_\mathrm{s}U_\mathrm{in}}{L}.
    \end{equation*}
    and calculating the derivative with respect to the duty cycle $D$ delivers
    \begin{equation}
        \frac{\mathrm{d}\Delta i_\mathrm{L}}{\mathrm{d}D} = \frac{T_\mathrm{s}U_\mathrm{in}}{L} - 2D\frac{T_\mathrm{s}U_\mathrm{in}}{L}.
    \end{equation}
    Setting the derivative to zero, we find the duty cycle $D_\mathrm{max}$ as
    \begin{equation}
        \frac{\mathrm{d}\Delta i_\mathrm{L}}{\mathrm{d}D} = 0 \quad \Leftrightarrow \quad D_\mathrm{max} = \frac{1}{2}
        \label{eq:duty-cycle-max}
    \end{equation}
    which is associated with the maximum current ripple since the second derivative 
    \begin{equation}
        \frac{\mathrm{d}^2\Delta i_\mathrm{L}}{\mathrm{d}D^2} = -\frac{2T_\mathrm{s}U_\mathrm{on}}{L}
    \end{equation}
    is negative. 
\end{frame}

%%%%%%%%%%%%%%%%%%%%%%%%%%%%%%%%%%%%%%%%%%%%%%%%%%%%%%%%%%%%%
%% Current ripple and duty cycle (cont.) %%
%%%%%%%%%%%%%%%%%%%%%%%%%%%%%%%%%%%%%%%%%%%%%%%%%%%%%%%%%%%%%
\begin{frame}
    \frametitle{Current ripple and duty cycle (cont.)}
    From \eqref{eq:duty-cycle-max} we can conclude that the maximum current ripple is given by
    \begin{equation}
        \Delta i_\mathrm{L, \max} = \frac{1}{4}\frac{T_\mathrm{s}U_\mathrm{in}}{L} \quad \Rightarrow \quad \Delta i_\mathrm{L} = 4D(1-D) \Delta i_\mathrm{L, \max}.
    \end{equation}

    \begin{figure}
        \begin{tikzpicture}
            \begin{axis}[
                xlabel={$D$},
                ylabel={$\Delta i_\mathrm{L}$},
                ymin=0, ymax=1.1,
                xmin=0, xmax=1,
                width = 0.6\textwidth,
                height = 0.5\textheight,
                grid,
                thick,
                clip = true,
                xtick = {0, 0.25, 0.5, 0.75, 1.0}, 
                ytick = {0, 0.5, 1.0}, 
                yticklabels = {0, $\nicefrac{\Delta i_\mathrm{L, max}}{2}$, $\Delta i_\mathrm{L, max}$}
                ]
                \addplot[signalblue, domain=0:1, samples=100] {x*(1-x)*4};
                \draw[dashed] (axis cs:0.5,0) -- (axis cs:0.5,1); % dashed line at max current ripple
                \node[above, inner sep = 2pt, anchor = east, fill = white] at (axis cs:0.5, 0.5) {$D_\mathrm{max}$};
            \end{axis}
        \end{tikzpicture}
        \caption{Inductor current ripple as a function of the duty cycle}
        \label{fig:inductor-current-ripple-duty}
    \end{figure}
\end{frame}

%%%%%%%%%%%%%%%%%%%%%%%%%%%%%%%%%%%%%%%%%%%%%%%%%%%%%%%%%%%%%
%% Step-down converter with output capacitor %%
%%%%%%%%%%%%%%%%%%%%%%%%%%%%%%%%%%%%%%%%%%%%%%%%%%%%%%%%%%%%%
\subsection{Step-down converter with output capacitor}

%%%%%%%%%%%%%%%%%%%%%%%%%%%%%%%%%%%%%%%%%%%%%%%%%%%%%%%%%%%%%
%% Step-down converter with output capacitor: overview and assumption %%
%%%%%%%%%%%%%%%%%%%%%%%%%%%%%%%%%%%%%%%%%%%%%%%%%%%%%%%%%%%%%

\begin{frame}[b]
    \frametitle{Step-down converter with output capacitor: overview and assumption}
        We consider the following assumptions:
        \begin{itemize}
            \item The switch is ideal, that is, infinitely fast.
            \item The input voltage is constant: $u_\mathrm{in}(t) = U_\mathrm{in}$.
            \item The output current is constant: $i_\mathrm{out}(t) = I_\mathrm{out}$.
            \item The input voltage is greater than the output voltage: $U_\mathrm{in} > u_\mathrm{out}(t)$.
        \end{itemize}
        \begin{figure}
            \begin{circuitikz}[]
                \draw (0,2) to [open, o-o, v = $\hspace{2cm}u_\mathrm{out}(t)$, voltage = straight] ++(0,-2)
                to ++(-7,0)
                to [open, o-o, v<= $u_\mathrm{in}(t) \hspace{2cm}$, voltage = straight] ++(0,2)
                (-7,2) to  [short, i=$i_\mathrm{in}(t)$] ++(1.35,0);   
                \draw (-6.375,2) ++(0.625,0) node [cuteopenswitchshape, anchor = out, rotate=180] (S) {}
                let \p1 = (S.mid) in (S.in) to  [short, i=$i_\mathrm{L}(t)$] ++(1,0)
                to [inductor, l=$L$, v = $u_\mathrm{L}(t)$, voltage = straight] ++(2,0)
                to [short] ++(1,0)
                to [short, i=$i_\mathrm{out}(t)$] (0,2) 
                ([yshift = -0.3cm]S.mid) to [short, o-*](\x1,0);
                \draw (-4.5,2) to [open, v = $\hspace{1.75cm}u_\mathrm{s}(t)$, voltage = straight] ++(0,-2);
                \draw (-1.5,2) to [capacitor, *-*, l=$C$, i>^=$i_\mathrm{C}(t)$] ++(0,-2);
            \end{circuitikz}
            \caption{Step-down converter (ideal switch representation) with output capacitor}
            \label{fig:step-down-converter-simple-output-cap}
        \end{figure}
    \end{frame}

%%%%%%%%%%%%%%%%%%%%%%%%%%%%%%%%%%%%%%%%%%%%%%%%%%%%%%%%%%%%%
%% Steady-state analysis %%
%%%%%%%%%%%%%%%%%%%%%%%%%%%%%%%%%%%%%%%%%%%%%%%%%%%%%%%%%%%%%
\begin{frame}
    \frametitle{Steady-state analysis}
    From \eqref{eq:inductor-current-switch-on-simple-step-down} we know that the inductor current during the switch-on period is given by
    \begin{equation*}
        i_\mathrm{L}(t) = i_\mathrm{L}(k T_\mathrm{s}) + \frac{U_\mathrm{in}-u_\mathrm{c}(t)}{L} (t - k T_\mathrm{s}), \quad t\in [k T_\mathrm{s}, k T_\mathrm{s} + T_\mathrm{on}]. 
    \end{equation*}
   Note that the inductor current is now dependent on $u_\mathrm{c}(t)$:
    \begin{itemize}
        \item Formally, we need to consider the impact of the varying output capacitor voltage.
        \item This would lead to a second-order differential equation which is more complex to solve.
        \item We will simplify the analysis by assuming that the impact of the output capacitor voltage variation on the inductor current is negligible: $u_\mathrm{c}(t) \approx U_\mathrm{out}=\overline{u}_\mathrm{c}$.
    \end{itemize} 
    \vspace{-0.5cm}
    \begin{varblock}{Simplification comment}
        The above assumption is valid for sufficiently large output capacitors with only small voltage ripples. Otherwise, the output voltage ripple and the inductor current ripple will be significantly coupled and require a more thoughtful analysis.
    \end{varblock}
\end{frame}

%%%%%%%%%%%%%%%%%%%%%%%%%%%%%%%%%%%%%%%%%%%%%%%%%%%%%%%%%%%%%
%% Steady-state analysis (cont.) %%
%%%%%%%%%%%%%%%%%%%%%%%%%%%%%%%%%%%%%%%%%%%%%%%%%%%%%%%%%%%%%
\begin{frame}
    \frametitle{Steady-state analysis (cont.)}
     The \hl{capacitor's voltage differential equation} is given by
    \begin{equation}
        C \frac{\mathrm{d}u_\mathrm{c}(t)}{\mathrm{d}t} = i_\mathrm{C}(t) = i_\mathrm{L}(t) - I_\mathrm{out}.
        \label{eq:capacitor-voltage-differential-equation-step-down-converter}
    \end{equation}
    While $I_\mathrm{out}$ is considered a known constant, we first need to determine the inductor current $i_\mathrm{L}(t)$. Combining \eqref{eq:inductor-current-switch-on-simple-step-down} and \eqref{eq:current-ripple-simple-step-down} we obtain
    \begin{equation}
         i_\mathrm{L}(k T_\mathrm{s})  = I_\mathrm{out} - \frac{\Delta i_\mathrm{L}}{2} = I_\mathrm{out} - \frac{U_\mathrm{in} - U_\mathrm{out}}{L} \frac{T_\mathrm{on}}{2}
    \end{equation}
    and
    \begin{equation}
        i_\mathrm{L}(k T_\mathrm{s} + T_\mathrm{on})  = I_\mathrm{out} + \frac{\Delta i_\mathrm{L}}{2} = I_\mathrm{out} + \frac{U_\mathrm{in} - U_\mathrm{out}}{L} \frac{T_\mathrm{on}}{2}
   \end{equation}
    as the \hl{initial conditions for the inductor current} in steady state. 
\end{frame}

%%%%%%%%%%%%%%%%%%%%%%%%%%%%%%%%%%%%%%%%%%%%%%%%%%%%%%%%%%%%%
%% Steady-state analysis (cont.) %%
%%%%%%%%%%%%%%%%%%%%%%%%%%%%%%%%%%%%%%%%%%%%%%%%%%%%%%%%%%%%%
\begin{frame}
    \frametitle{Steady-state analysis (cont.)}
     The capacitor's current during the \hl{switch-on period} is given by
    \begin{equation}
        \begin{split}
            i_\mathrm{C}(t) &= i_\mathrm{L}(t) - I_\mathrm{out} = \frac{U_\mathrm{in}-U_\mathrm{out}}{L} (t - \frac{T_\mathrm{on}}{2}-k T_\mathrm{s})\\
                            &= -\frac{\Delta i_\mathrm{L}}{2} + \frac{U_\mathrm{in}-U_\mathrm{out}}{L} (t -k T_\mathrm{s}), \quad t\in [k T_\mathrm{s}, k T_\mathrm{s} + T_\mathrm{on}]
        \end{split}
        \label{eq:capacitor-current-switch-on-step-down-converter}
    \end{equation}
    and during the \hl{switch-off period} we receive
    \begin{equation}
        \begin{split}
            i_\mathrm{C}(t) &= i_\mathrm{L}(t) - I_\mathrm{out} = \frac{U_\mathrm{in} - U_\mathrm{out}}{L} \frac{T_\mathrm{on}}{2} -\frac{U_\mathrm{out}}{L} (t - k T_\mathrm{s} - T_\mathrm{on})\\
                            &= \frac{\Delta i_\mathrm{L}}{2} - \frac{U_\mathrm{out}}{L} (t - k T_\mathrm{s} - T_\mathrm{on}), \quad t\in [k T_\mathrm{s} + T_\mathrm{on}, (k+1) T_\mathrm{s}].
        \end{split}
        \label{eq:capacitor-current-switch-off-step-down-converter}
    \end{equation}
    Based on the made assumptions, the capacitor's current is raising and falling linearly during the switch-on and switch-off periods, that is, it corresponds to the previous inductor current ripple analysis.
\end{frame}

%%%%%%%%%%%%%%%%%%%%%%%%%%%%%%%%%%%%%%%%%%%%%%%%%%%%%%%%%%%%%
%% Steady-state analysis (cont.) %%
%%%%%%%%%%%%%%%%%%%%%%%%%%%%%%%%%%%%%%%%%%%%%%%%%%%%%%%%%%%%%
\begin{frame}
    \frametitle{Steady-state analysis (cont.)}
    Inserting \eqref{eq:capacitor-current-switch-on-step-down-converter} and \eqref{eq:capacitor-current-switch-off-step-down-converter} in \eqref{eq:capacitor-voltage-differential-equation-step-down-converter} and integrating the differential equation delivers
    \begin{equation}
        u_\mathrm{c}(t) = u_\mathrm{c}(k T_\mathrm{s}) -\frac{\Delta i_\mathrm{L}}{2 C}t + \frac{U_\mathrm{in}-U_\mathrm{out}}{LC} (\frac{t^2}{2} - t k T_\mathrm{s}), \quad t\in [k T_\mathrm{s}, k T_\mathrm{s} + T_\mathrm{on}]
        \label{eq:capacitor-voltage-switch-on-step-down-converter}
    \end{equation}
    and 
    \begin{equation}
        u_\mathrm{c}(t) = u_\mathrm{c}(k T_\mathrm{s} + T_\mathrm{on}) + \frac{\Delta i_\mathrm{L}}{2 C}t - \frac{U_\mathrm{out}}{LC} (\frac{t^2}{2} - t k T_\mathrm{s} - tT_\mathrm{on}), \quad t\in [k T_\mathrm{s} + T_\mathrm{on}, (k+1) T_\mathrm{s}].
    \end{equation}
    \begin{figure}
        \begin{tikzpicture}
            \pgfmathsetmacro{\D}{0.6} % duty cycle
            \begin{axis}[
                xlabel={$t/T_\mathrm{s}$},
                ylabel={$i_\mathrm{C}(t)$},
                ymin=-0.3, ymax=0.3,
                xmin=-0.1, xmax=1.1,
                width = 0.7\textwidth,
                height = 0.4\textheight,
                grid,
                thick,
                clip = true,
                ytick = {-0.2, 0, 0.2}, 
                yticklabels = {$\nicefrac{-\Delta i_\mathrm{L}}{2}$, 0, $\nicefrac{\Delta i_\mathrm{L}}{2}$}
                ]
                \addplot[signalred] coordinates {(\D-1,0.22) (0,-0.2) (\D,0.2) (1, -0.2) (1+\D,0.2)};
                \draw [thin, <->]  (0.6,-0.2) -- node[left,fill=white]{$\Delta i_\mathrm{L}$}(0.6, 0.2); 
            \end{axis}
        \end{tikzpicture}
    \end{figure}
\end{frame}

%%%%%%%%%%%%%%%%%%%%%%%%%%%%%%%%%%%%%%%%%%%%%%%%%%%%%%%%%%%%%
%% Steady-state analysis (cont.) %%
%%%%%%%%%%%%%%%%%%%%%%%%%%%%%%%%%%%%%%%%%%%%%%%%%%%%%%%%%%%%%
\begin{frame}
    \frametitle{Steady-state analysis (cont.)}
    To simplify the following calculation steps, we will focus on the first period, i.e., setting $k=0$. The capacitor voltage for $t=T_\mathrm{on}$ results in
    \begin{equation}
        \begin{split}   
        u_\mathrm{c}(t=T_\mathrm{on}) &= u_\mathrm{c}(0) -\frac{\Delta i_\mathrm{L}}{2 C}T_\mathrm{on} + \frac{U_\mathrm{in}-U_\mathrm{out}}{LC} \frac{T_\mathrm{on}^2}{2}\\
                                      &= u_\mathrm{c}(0) -\frac{U_\mathrm{in}-U_\mathrm{out}}{LC} \frac{T_\mathrm{on}^2}{2} + \frac{U_\mathrm{in}-U_\mathrm{out}}{LC} \frac{T_\mathrm{on}^2}{2}\\
                                      &= u_\mathrm{c}(0).
        \end{split}
    \end{equation}
    Likewise, the capacitor voltage for $t=T_\mathrm{s}$ results in
    \begin{equation}
        \begin{split}   
        u_\mathrm{c}(t=T_\mathrm{s}) &= u_\mathrm{c}(T_\mathrm{on}) -\frac{\Delta i_\mathrm{L}}{2 C}T_\mathrm{s} + \frac{U_\mathrm{in}-U_\mathrm{out}}{LC} \frac{T_\mathrm{on}T_\mathrm{s}}{2}\\
                                      &= u_\mathrm{c}(0) -\frac{U_\mathrm{in}-U_\mathrm{out}}{LC} \frac{T_\mathrm{on}T_\mathrm{s}}{2} + \frac{U_\mathrm{in}-U_\mathrm{out}}{LC} \frac{T_\mathrm{on}T_\mathrm{s}}{2}\\
                                      &= u_\mathrm{c}(0).
        \end{split}
    \end{equation}
    Hence, the capacitor voltage's initial condition $u_\mathrm{c}(0)$ is also valid for $t=T_\mathrm{on}$ and $t=T_\mathrm{s}$. Hence, the maximum voltage ripple must be between the switching events. 
\end{frame}

%%%%%%%%%%%%%%%%%%%%%%%%%%%%%%%%%%%%%%%%%%%%%%%%%%%%%%%%%%%%%
%% Steady-state analysis (cont.) %%
%%%%%%%%%%%%%%%%%%%%%%%%%%%%%%%%%%%%%%%%%%%%%%%%%%%%%%%%%%%%%
\begin{frame}
    \frametitle{Steady-state analysis (cont.)}
    Utilizing \eqref{eq:capacitor-voltage-switch-on-step-down-converter} and calculating the derivative with respect to $t$ we obtain
    \begin{equation}
        \begin{split}
            \frac{\mathrm{d}u_\mathrm{c}(t)}{\mathrm{d}t} &= -\frac{\Delta i_\mathrm{L}}{2 C} + \frac{U_\mathrm{in}-U_\mathrm{out}}{LC} t\\
                                                          &= -\frac{U_\mathrm{in}-U_\mathrm{out}}{LC}\frac{T_\mathrm{on}}{2} +\frac{U_\mathrm{in}-U_\mathrm{out}}{LC} t, \quad t\in [0, T_\mathrm{on}].
        \end{split}
    \end{equation}
    Setting the derivative to zero, we find the time $t_\mathrm{min}$ at which the minimum voltage occurs as
    \begin{equation}
        \frac{\mathrm{d}u_\mathrm{c}(t)}{\mathrm{d}t} = 0 \quad \Rightarrow \quad t_\mathrm{min} = \frac{T_\mathrm{on}}{2}
    \end{equation}
    since the second derivative is positive. The minimum voltage is then given by
    \begin{equation}
        u_\mathrm{c}(t_\mathrm{min}) = u_\mathrm{c}(0) - \frac{U_\mathrm{in}-U_\mathrm{out}}{LC} \frac{T_\mathrm{on}^2}{8} = u_\mathrm{c}(0) - \Delta i_\mathrm{L}\frac{T_\mathrm{on}}{8C}.
    \end{equation}
\end{frame}


