\section{DC-DC converters}
\title{DC-DC converters}  

\begin{frame}[plain]
    \titlepage
\end{frame}

%%%%%%%%%%%%%%%%%%%%%%%%%%%%%%%%%%%%%%%%%%%%%%%%%%%%%%%%%%%%%
%% Step-down converter (unidirectional) %%
%%%%%%%%%%%%%%%%%%%%%%%%%%%%%%%%%%%%%%%%%%%%%%%%%%%%%%%%%%%%%
\subsection{Step-down converter}

%%%%%%%%%%%%%%%%%%%%%%%%%%%%%%%%%%%%%%%%%%%%%%%%%%%%%%%%%%%%%
%% Overview and assumption %%
%%%%%%%%%%%%%%%%%%%%%%%%%%%%%%%%%%%%%%%%%%%%%%%%%%%%%%%%%%%%%

\begin{frame}[b]
\frametitle{Overview and assumption}
    We consider the following assumptions:
    \begin{itemize}
        \item The switch is ideal, that is, lossless and infinitely fast.
        \item The input voltage is constant: $u_\mathrm{in}(t) = U_\mathrm{in}$.
        \item The output voltage is constant: $u_\mathrm{out}(t) = U_\mathrm{out}$.
        \item The input voltage is greater than the output voltage: $U_\mathrm{in} > U_\mathrm{out}$.
    \end{itemize}
    \begin{figure}
        \begin{circuitikz}[]
            \draw (0,2) to [open, o-o, v = $\hspace{2cm}u_\mathrm{out}(t)$, voltage = straight] ++(0,-2)
            to ++(-6,0)
            to [open, o-o, v<= $u_\mathrm{in}(t) \hspace{2cm}$, voltage = straight] ++(0,2)
            (-6,2) to  [short, i=$i_\mathrm{in}(t)$] ++(1.35,0);   
            \draw (-5.375,2) ++(0.625,0) node [cuteopenswitchshape, anchor = out, rotate=180] (S) {}
            let \p1 = (S.mid) in (S.in) to  [short, i=$i_\mathrm{L}(t)$] ++(1,0)
            to [inductor, l=$L$, v = $u_\mathrm{L}(t)$, voltage = straight] ++(2,0)
            to [short, i=$i_\mathrm{out}(t)$] (0,2) 
            ([yshift = -0.3cm]S.mid) to [short, o-*](\x1,0);
            \draw (-3.5,2) to [open, v = $\hspace{1.75cm}u_\mathrm{s}(t)$, voltage = straight] ++(0,-2);
        \end{circuitikz}
        \caption{Step-down converter (ideal switch representation)}
        \label{fig:step-down-converter-simple}
    \end{figure}
\end{frame}

%%%%%%%%%%%%%%%%%%%%%%%%%%%%%%%%%%%%%%%%%%%%%%%%%%%%%%%%%%%%%
%% Switch states %%
%%%%%%%%%%%%%%%%%%%%%%%%%%%%%%%%%%%%%%%%%%%%%%%%%%%%%%%%%%%%%

\begin{frame}[b]
    \frametitle{Switch states}
     The voltage at the switch is given by
     \begin{equation}
            u_\mathrm{s}(t) = \begin{cases}
                U_\mathrm{in}, & t\in [k T_\mathrm{s}, k T_\mathrm{s} + T_\mathrm{e}],\\
                0, & t\in [k T_\mathrm{s}+ T_\mathrm{e}, (k+1) T_\mathrm{s}]
            \end{cases}
     \end{equation}
     with $k\in\mathbb{N}$ being the $k$-th switching period, $T_\mathrm{s}$ the switching period, and $T_\mathrm{e}$ the switch-on time. 
        \begin{figure}
            \centering	
            \begin{subfigure}{0.45\textwidth}
                \centering
                \hspace{-0.75cm}
                \begin{circuitikz}[]
                    \draw (0,2) to [open, o-o, v = $\hspace{2cm}U_\mathrm{out}$, voltage = straight] ++(0,-2)
                    to ++(-5,0)
                    to [open, o-o, v<= $U_\mathrm{in} \hspace{2cm}$, voltage = straight] ++(0,2)
                    (-5,2) to  [short, i=${i_\mathrm{in}(t)=i_\mathrm{L}(t)}$] ++(2,0)
                    to [inductor, l=$L$, v = $u_\mathrm{L}(t)$, voltage = straight] ++(2,0)
                    to [short, i=$i_\mathrm{out}(t)$] (0,2);
                    \draw (-4,2) to [open, v = ${\hspace{2.4cm}U_\mathrm{s}=U_\mathrm{in}}$, voltage = straight] ++(0,-2);
                \end{circuitikz}
                \caption{Switch-on time}
            \end{subfigure}%
            \hspace{0.5cm}
            \begin{subfigure}{0.45\textwidth}
                \centering
                \begin{circuitikz}[]
                    \draw (0,2) to [open, o-o, v = $\hspace{2cm}U_\mathrm{out}$, voltage = straight] ++(0,-2)
                    to ++(-5,0)
                    to [open, o-o, v<= $U_\mathrm{in} \hspace{2cm}$, voltage = straight] ++(0,2)
                    (-5,2) to  [short, i=${i_\mathrm{in}=0}$] ++(0.75,0)
                    (-3.75,0) to [short, *-] ++(0,2)
                    to [short, i=$i_\mathrm{L}(t)$] ++(0.75,0)
                    to [inductor, l=$L$, v = $u_\mathrm{L}(t)$, voltage = straight] ++(2,0)
                    to [short, i=$i_\mathrm{out}(t)$] (0,2);
                    \draw (-3.4,2) to [open, v = ${\hspace{2.4cm}U_\mathrm{s}=0}$, voltage = straight] ++(0,-2);
                \end{circuitikz}
                \caption{Switch-off time}
            \end{subfigure}
        \caption{Switch states of the step-down converter} 
        \label{fig:step-down-converter-switch-states}
        \end{figure}
    \end{frame}

%%%%%%%%%%%%%%%%%%%%%%%%%%%%%%%%%%%%%%%%%%%%%%%%%%%%%%%%%%%%%
%% Basic terms and definitions %%
%%%%%%%%%%%%%%%%%%%%%%%%%%%%%%%%%%%%%%%%%%%%%%%%%%%%%%%%%%%%%
\begin{frame}[c]
    \frametitle{Basic terms and definitions}
    \centering
     \begin{tabular}{c l c l}
        $T_\mathrm{on}$ & Switch-on time  & $T_\mathrm{off}$ & Switch-off time\\[1em]
        $T_\mathrm{s}= T_\mathrm{on} + T_\mathrm{off}$ & Switching period & $f_\mathrm{s} = 1/T_\mathrm{s}$ & Switching frequency\\[1em]
        $D = T_\mathrm{on}/T_\mathrm{s}$ & Duty cycle &  & 
     \end{tabular}
     \begin{figure}
        \begin{tikzpicture}
            \begin{axis}[
                xlabel={$t/T_\mathrm{s}$},
                ylabel={$u_\mathrm{s}(t)/U_\mathrm{in}$},
                ymin=-0.05, ymax=1.05,
                xmin=-0.1, xmax=1.1,
                width = 0.5\textwidth,
                height = 0.4\textheight,
                grid,
                thick,
                clip = false,
                ]
                % plt u_out = 1 for t = 0 to D*Ts and u_out = 0 for t = D*Ts to Ts in a single plot
                \addplot[signalblue] coordinates {(-0.1,0) (0,0) (0,1) (0.4,1) (0.4,0) (1,0) (1,1) (1.1,1)};
                \draw [thick,<->]  (0,0.5) -- node[above,fill=white]{$T_\mathrm{on}$}(0.4, 0.5); 
                \draw [thick,<->]  (0.4,0.5) -- node[above]{$T_\mathrm{off}$}(1.0, 0.5);
                \draw [thick,<->]  (0.0,1.2) -- node[above]{$T_\mathrm{s}$}(1.0, 1.2); 
            \end{axis}
        \end{tikzpicture}
    \end{figure}
    \end{frame}

%%%%%%%%%%%%%%%%%%%%%%%%%%%%%%%%%%%%%%%%%%%%%%%%%%%%%%%%%%%%%
%% Steady-state analysis %%
%%%%%%%%%%%%%%%%%%%%%%%%%%%%%%%%%%%%%%%%%%%%%%%%%%%%%%%%%%%%%
\begin{frame}
    \frametitle{Steady-state analysis}
    The inductor current from \figref{fig:step-down-converter-simple} is represented by the \hl{differential equation}
    \begin{equation}
        L \frac{\mathrm{d}i_\mathrm{L}(t)}{\mathrm{d}t} =  u_\mathrm{L}(t) =  u_\mathrm{s}(t) - U_\mathrm{out}.
    \end{equation}
    During the \hl{switch-on period} we have
    \begin{equation}
        i_\mathrm{L}(t) = i_\mathrm{L}(k T_\mathrm{s}) + \frac{U_\mathrm{in}-U_\mathrm{out}}{L} (t - k T_\mathrm{s}), \quad t\in [k T_\mathrm{s}, k T_\mathrm{s} + T_\mathrm{on}]
    \end{equation}
    and during the \hl{switch-off period} we receive
    \begin{equation}
        \begin{split}
            i_\mathrm{L}(t) &= i_\mathrm{L}(k T_\mathrm{s} + T_\mathrm{on}) - \frac{U_\mathrm{out}}{L} (t - k T_\mathrm{s} - T_\mathrm{on})\\
            \               &= i_\mathrm{L}(k T_\mathrm{s}) + \frac{U_\mathrm{in}-U_\mathrm{out}}{L} T_\mathrm{on} - \frac{U_\mathrm{out}}{L} (t - k T_\mathrm{s} - T_\mathrm{on}), \quad 
            t\in [k T_\mathrm{s} + T_\mathrm{on}, (k+1) T_\mathrm{s}]. 
        \end{split} 
        \label{eq:inductor-current-switch-off}
    \end{equation}
\end{frame}

%%%%%%%%%%%%%%%%%%%%%%%%%%%%%%%%%%%%%%%%%%%%%%%%%%%%%%%%%%%%%
%% Steady-state analysis (cont.) %%
%%%%%%%%%%%%%%%%%%%%%%%%%%%%%%%%%%%%%%%%%%%%%%%%%%%%%%%%%%%%%
\begin{frame}
    \frametitle{Steady-state analysis (cont.)}
    In steady state the inductor current is periodic with period $T_\mathrm{s}$, that is, $$i_\mathrm{L}(t) = i_\mathrm{L}(t + T_\mathrm{s}).$$
    From \eqref{eq:inductor-current-switch-off} we obtain for $t= k T_\mathrm{s}$
    \begin{equation}
        \begin{split}
        \overbrace{i_\mathrm{L}(k T_\mathrm{s})}^{\mbox{Start of period}} &= \hspace{0.5cm}  \overbrace{i_\mathrm{L}(k T_\mathrm{s}) + \frac{U_\mathrm{in}-U_\mathrm{out}}{L} T_\mathrm{on} - \frac{U_\mathrm{out}}{L} (T_\mathrm{s} - T_\mathrm{on})}^{\mbox{End of period}}\\
        \Leftrightarrow \qquad 0 \hspace{1.1cm}&= \hspace{0.5cm}\frac{U_\mathrm{in}-U_\mathrm{out}}{L} T_\mathrm{on} - \frac{U_\mathrm{out}}{L} (T_\mathrm{s} - T_\mathrm{on})\\
        \Leftrightarrow \qquad 0 \hspace{1.1cm}&= \hspace{0.5cm}U_\mathrm{in} T_\mathrm{on} - U_\mathrm{out} T_\mathrm{s}.
    \end{split}
    \end{equation}
    Rewritting delivers the \hl{output voltage} as
    \begin{equation}
        U_\mathrm{out} = \frac{T_\mathrm{on}}{T_\mathrm{s}} U_\mathrm{in} = D U_\mathrm{in}.
    \end{equation}
\end{frame}

%%%%%%%%%%%%%%%%%%%%%%%%%%%%%%%%%%%%%%%%%%%%%%%%%%%%%%%%%%%%%
%% Steady-state time-domain behavior %%
%%%%%%%%%%%%%%%%%%%%%%%%%%%%%%%%%%%%%%%%%%%%%%%%%%%%%%%%%%%%%
\begin{frame}[fragile]
    \frametitle{Steady-state time-domain behavior}
    \begin{figure}
        \begin{tikzpicture}
            \pgfmathsetmacro{\D}{0.6} % duty cycle
            \begin{groupplot}[group style={group size=1 by 3, xticklabels at = edge bottom}, height=0.34\textheight, width=0.875\textwidth, xmin=0, xmax=4, grid,clip = false, ymin = 0, ymax =1.1]

                % Top plot: voltage at the switch
                \nextgroupplot[ylabel = {$u_\mathrm{s}(t)$}, ytick = {0, 0.5, 1}, yticklabels = {0, , $U_1$}]
                    \pgfplotsinvokeforeach{0,...,3}{
                        \edef\AddPlot{\noexpand\addplot[signalblue, thick] coordinates {({0 + #1},0) ({0 + #1},1) ({\D + #1},1) ({\D + #1},0) ({1 + #1},0) ({1 + #1},1)};}
                        \AddPlot
                    }
                    \draw[signalblue, thick, dashed] (axis cs:0, \D) -- (axis cs:4, \D); % dashed line at U_2 (average)
                    \node[above, inner sep = 2pt, anchor = south] at (axis cs:1.5+\D/2, \D) {$U_2$}; % label U_2
                    \draw [thick,<->]  (0,0.5) -- node[below]{$T_\mathrm{on}$}(\D, 0.5); % T_on 
                    \draw [thick,<->]  (\D,0.5) -- node[below]{$T_\mathrm{off}$}(1.0, 0.5); % T_off
                    \draw [thick,<->]  (0.0,-0.2) -- node[below]{$T_\mathrm{s}$}(1.0, -0.2); % T_s 


                % Middle plot: inductor current
                \nextgroupplot[ylabel = {$i_\mathrm{L}(t)$}, ytick = {0, 0.5, 1}, yticklabels = {0, $\overline{i}_\mathrm{L}$, }]
                    \pgfplotsinvokeforeach{0,...,3}{
                        \edef\AddPlot{\noexpand\addplot[signalred, thick] coordinates {({0 + #1},0.25) ({\D + #1},0.75) ({1 + #1},0.25)};}
                        \AddPlot
                    }
                    \draw[signalred, thick, dashed] (axis cs:0,0.5) -- (axis cs:4,0.5); % dashed line at average current
                    \draw[{Latex[length=2mm]}-, thin] (axis cs:\D+0.02,0.75) -- node[right=1mm, fill=white, inner sep = 1pt]{$\max\{i_\mathrm{L}\}$}(axis cs:\D+0.3,0.9); % indicate max current
                    \draw[-{Latex[length=2mm]}, thin] (axis cs:0.75,0.2) node[right=1mm, fill=white, inner sep = 1pt, anchor = east]{$\min\{i_\mathrm{L}\}$} -- (axis cs:1-0.02,0.25); % indicate min current
                    \draw[thin] (axis cs:2+\D/4,0.25+0.125) -- (axis cs:2+\D/4,0.75-0.125) -- (axis cs:2+\D*3/4,0.75-0.125); % indicate positive current slopde
                    \node[above, inner sep = 2pt, anchor = south] at (axis cs:2+\D/2, 0.75-0.125) {$\nicefrac{U_1-U_2}{L}$}; % label positive current slope
                    \draw[thin] (axis cs:2.25+3*\D/4,0.75-0.125) -- (axis cs:2.75+\D/4,0.75-0.125) -- (axis cs:2.75+\D/4,0.25+0.125); % indicate negative current slope
                    \node[above, inner sep = 2pt, anchor = south] at (axis cs:2.5+\D/2, 0.75-0.125) {$\nicefrac{-U_2}{L}$}; % label negative current slope
                
                % Bottom plot: input current
                \nextgroupplot[ylabel = {$i_\mathrm{in}(t)$}, xlabel={$t/T_\mathrm{s}$}, ytick = {0, 0.5, 1}, yticklabels = {0, $\overline{i}_\mathrm{L}$}]
                    \pgfplotsinvokeforeach{0,...,3}{
                        \edef\AddPlot{\noexpand\addplot[signalred, thick] coordinates {({0 + #1},0.25) ({\D + #1},0.75) ({\D + #1},0) ({1 + #1},0) ({1 + #1},0.25)};}
                        \AddPlot
                    }
                    \draw[signalred, thick, dashed] (axis cs:0,0.5 * \D) -- (axis cs:4, 0.5 * \D); % dashed line at average current
                    \node[above, inner sep = 2pt, anchor = south, fill = white] at (axis cs:1.5+\D/2, 0.5 * \D) {$\overline{i}_\mathrm{in}$}; % label average current
            \end{groupplot}
        \end{tikzpicture}
    \end{figure}
\end{frame}

%%%%%%%%%%%%%%%%%%%%%%%%%%%%%%%%%%%%%%%%%%%%%%%%%%%%%%%%%%%%%
%% Alternative steady-state analysis: average values %%
%%%%%%%%%%%%%%%%%%%%%%%%%%%%%%%%%%%%%%%%%%%%%%%%%%%%%%%%%%%%%
\begin{frame}
    \frametitle{Alternative steady-state analysis: average values}
    From the previous slide we know that the average inductor voltage is zero in steady state
    \begin{equation}
        \overline{u}_\mathrm{L} = \frac{1}{T_\mathrm{s}} \int_{0}^{T_\mathrm{s}} u_\mathrm{L}(t) \mathrm{d}t = \overline{u}_\mathrm{L} = 0,
    \end{equation}
    since otherwise the average inductor current would change between periods, compare
    $$
    L \frac{\mathrm{d}i_\mathrm{L}(t)}{\mathrm{d}t} =  u_\mathrm{L}(t).
    $$
\end{frame}