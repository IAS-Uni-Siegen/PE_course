%%%%%%%%%%%%%%%%%%%%%%%%%%%%%%%%%%%%%%%%%%%%%%%%%%%%%%%%%%%%%
%% Transistor-based AC/DC converters %%
%%%%%%%%%%%%%%%%%%%%%%%%%%%%%%%%%%%%%%%%%%%%%%%%%%%%%%%%%%%%%
\section{Transistor-based AC/DC converters}

%%%%%%%%%%%%%%%%%%%%%%%%%%%%%%%%%%%%%%%%%%%%%%%%%%%%%%%%%%%%%
%% Extending AC/DC converter to four quadrant operation %%
%%%%%%%%%%%%%%%%%%%%%%%%%%%%%%%%%%%%%%%%%%%%%%%%%%%%%%%%%%%%%
\begin{frame}
	\frametitle{Extending AC/DC converters to four quadrant operation}
	\begin{columns}
		\begin{column}{0.5\textwidth}
			
			\hl{Up to now}:
            \begin{itemize}
                \item Diode-based converters
                \begin{itemize}
                    \item Rectification only
                    \item Single quadrant operation
                \end{itemize}
                \item Thyristor-based converters
                \begin{itemize}
                    \item Rectification and inversion
                    \item Two quadrant operation
                \end{itemize}
            \end{itemize}
            \vspace{1em}
            \hl{Extension in this section:}
            \begin{itemize}
                \item Transistor-based converters
                \begin{itemize}
                    \item Rectification and inversion
                    \item Four quadrant operation
                \end{itemize}
            \end{itemize}
		\end{column}
		\begin{column}{0.5\textwidth}
			\begin{figure}
				\centering
				\begin{tikzpicture}
                    \begin{scope}[]
                        \fill[signalgreen, opacity=0.3] (0,0) -- (2,0) -- (2,2) -- (0,2) -- cycle;
                        \fill[signalgreen, opacity=0.3] (0,0) -- (2,0) -- (2,-2) -- (0,-2) -- cycle;
                    \end{scope}
                    \begin{scope}[]
                        \fill[shadecolor, opacity=0.3] (0,0) -- (-2,0) -- (-2,2) -- (0,2) -- cycle;
                        \fill[shadecolor, opacity=0.3] (0,0) -- (-2,0) -- (-2,-2) -- (0,-2) -- cycle;
                    \end{scope}
                    \draw[<->] (-2,0) -- (2,0) node[anchor=west] {$i$};
                    \draw[<->] (0,-2) -- (0,2) node[anchor=south] {$u$};
                    \node[anchor=center, align = center] at (1.0,1.0) {$\mathrm{I}$\\$P \geq 0$};
                    \node[anchor=center, align = center] at (-1.0,1.0) {$\mathrm{II}$\\$P \leq 0$};
                    \node[anchor=center, align = center] at (-1.0,-1.0) {$\mathrm{III}$\\$P \geq 0$};
                    \node[anchor=center, align = center] at (1.0,-1.0) {$\mathrm{IV}$\\$P \leq 0$};
                    \draw[thin] (1.75,0.75) to (2.4,0.25) node[anchor=north west] {Thyristors};
                    \draw[thin] (1.75,-0.75) to (2.4,-0.25) node[anchor=south west] {};
                    \draw[thin] (1.75,1.75) to (2.4,2.25) node[anchor=west] {Diodes};
					\node[anchor = center, xshift = 1mm] at (0, 3.5) {
						\begin{circuitikz}
							\node[twoportsplitshape, scale = 1.5](tp){};
							\draw (tp.left up) to [short, -o, i_<= $i_1$] ++(-0.75,0) coordinate(tpin1)
							(tp.left down) to [short, -o] ++(-0.75,0) coordinate(tpin2);
							\draw[->] ([xshift=-0.9cm]tp.left up) to node[anchor = east]{$u_1$} ([xshift=-0.9cm]tp.left down);
							\draw (tp.right up) to [short, -o, i= $i_2$] ++(0.75,0) coordinate(tpout1)
							(tp.right down) to [short, -o] ++(+0.75,0) coordinate(tpout2);
							\draw[->] ([xshift=1cm]tp.right up) to node[anchor = west]{$u_2$} ([xshift=1cm]tp.right down);
						\end{circuitikz}	
					} ;
				\end{tikzpicture}
			\end{figure}
		\end{column}
	\end{columns}
\end{frame}

%%%%%%%%%%%%%%%%%%%%%%%%%%%%%%%%%%%%%%%%%%%%%%%%%%%%%%%%%%%%%
%% Single-phase bridge converter %%
%%%%%%%%%%%%%%%%%%%%%%%%%%%%%%%%%%%%%%%%%%%%%%%%%%%%%%%%%%%%%
\subsection{Single-phase AC/DC bridge converter} 

%%%%%%%%%%%%%%%%%%%%%%%%%%%%%%%%%%%%%%%%%%%%%%%%%%%%%%%%%%%%%
%% Idealized switch representation %%
%%%%%%%%%%%%%%%%%%%%%%%%%%%%%%%%%%%%%%%%%%%%%%%%%%%%%%%%%%%%%
\begin{frame}
    \frametitle{Idealized switch representation of a single-phase AC/DC bridge converter}
    \begin{columns}
        \begin{column}{0.4\textwidth}
            Define \hl{switching function}:
            \begin{equation}
                s_i(t)=\begin{cases}
                    +1 & \text{upper position,}\\
                    -1 & \text{lower position.}
                \end{cases}
            \end{equation}
            Output voltage considering a voltage source at the input:
            \begin{equation}
                u_2(t)=\underbrace{\frac{1}{2}\left(s_1(t)-s_2(t)\right)}_{s(t)}u_1(t).
            \end{equation}
            Input current assuming a current source at the output:
            \begin{equation}
                i_1(t)=s(t)i_2(t).
            \end{equation}
        \end{column}
        \begin{column}{0.6\textwidth}
            \begin{figure}
                \begin{circuitikz}
                    \draw (0,0) node[cute spdt up arrow, xscale=-1] (Sw1) {};
                    \draw (2,-1.5) node[cute spdt down arrow, xscale=-1] (Sw2) {};
                    \draw (Sw2.in) to [short, -o] ++(1,0) coordinate (out2);
                    \draw (Sw1.in) to [short] ++(1.75,0) to [short, -o, i=$i_2(t)$] (Sw1.in -| out2) coordinate (out1);
                    \draw (out1) to [open, v^=$\hspace{0.5cm}u_2(t)$, voltage = straight] (out2);
                    \draw (Sw1-out 1.n) to [short, -*] ++(0,0.5) coordinate (int1);
                    \draw (Sw2-out 2.s) to [short] ++(0,-0.5) coordinate (int2);
                    \draw (int2) to [short] (int2 -| Sw1-out 2.s) coordinate (int3) to [short, *-] (Sw1-out 2.s);
                    \draw node[crossingshape, name=x1, rotate=-90] at (out1 -| Sw2-out 1.n) {};
                    \draw (x1.west) to [short] (int1 -| x1.west) to [short] (int1);
                    \draw (x1.east) to [short] (Sw2-out 1.n);
                    \draw (int1) to [short] ++(-1,0) to [short, i_<=$i_1(t)$, -o] ++(-1,0) coordinate (in1);
                    \draw (int3) to [short, -o] ++(-2,0) coordinate (in2);
                    \draw (in1) to [open, v=$u_1(t)\hspace{0.5cm}$, voltage = straight] (in2);
                    \draw node[anchor = east, xshift=-0.3cm] at (Sw1) {$s_1(t)$};
                    \draw node[anchor = east, xshift=-0.3cm] at (Sw2) {$s_2(t)$};
                \end{circuitikz}
                \caption{Idealized switch representation of a single-phase AC/DC bridge converter}
                \label{fig:idealized_switch_single_phase_bridge_converter}
            \end{figure}
        \end{column}
    \end{columns}
\end{frame}

%%%%%%%%%%%%%%%%%%%%%%%%%%%%%%%%%%%%%%%%%%%%%%%%%%%%%%%%%%%%%
%% Circuit realization %%
%%%%%%%%%%%%%%%%%%%%%%%%%%%%%%%%%%%%%%%%%%%%%%%%%%%%%%%%%%%%%
\begin{frame}
    \frametitle{Circuit realization}
    \begin{columns}
        \begin{column}{0.45\textwidth}
            \begin{itemize}
                \item Remember: complementary switching of $\{T_1, T_2\}$ and $\{T_3, T_4\}$ to prevent a DC-link short-circuit. 
                \item Possible (allowed) switching states:
            \end{itemize} 
            \vspace{-0.25cm}     
            \begin{center}
                \begin{tabular}{c c c c c c c}
                    \toprule
                    $T_1$ & $T_2$ &$T_3$ & $T_4$ & $s_1$ & $s_2$ & $s$\\
                    \midrule
                    on & off & off & on & $+1$ & $-1$ & $+1$\\
                    off & on & on & off & $-1$ & $+1$ & $-1$\\
                    on & off & on & off & $+1$ & $+1$ & $0$\\
                    off & on & off & on & $-1$ & $-1$ & $0$\\
                    \bottomrule
                \end{tabular}
            \end{center}  
        \end{column}
        \hfill
        \begin{column}{0.55\textwidth}
            \begin{figure}
                \begin{circuitikz}[]
                    \draw (0,4) coordinate (A) to [open, o-o, v = $u_1(t)\hspace{0.5cm}$, voltage = straight] ++(0,-5) coordinate (B)
                    (A) to [short, o-, i=$i_1(t)$] ++(2,0) coordinate (E)
                    to [Tnpn, n=npn1, invert, bodydiode] ++(0,-2) coordinate (C)
                    to [short, *-] ++(1,0) to [crossing] ++(2,0)   
                    to [short, i=$i_2(t)$, -o] ++(1,0) coordinate (G)
                    (C) to [short] ++(0,-1) 
                    to [Tnpn, n=npn2, invert, bodydiode] ++(0,-2) coordinate (D)
                    (E) to [short, *-] ++(2,0)
                    to [Tnpn, n=npn3, invert, bodydiode] ++(0,-2)
                    to [short] ++(0,-1) coordinate (F)
                    to [Tnpn, n=npn4, invert, bodydiode] ++(0,-2) 
                    to [short, -*] ++(-2,0)
                    to [short, -o] (B)
                    (F) to [short,*-o] ++(2,0) coordinate (H)
                    (G) to [open, o-o, v = $\hspace{1.9cm}u_2(t)$, voltage = straight] (H);
                    \draw let \p1 = (npn1.B) in node[anchor=east] at (\x1,\y1) {$T_1$};
                    \draw let \p1 = (npn2.B) in node[anchor=east] at (\x1,\y1) {$T_2$};
                    \draw let \p1 = (npn3.B) in node[anchor=east] at (\x1,\y1) {$T_3$};
                    \draw let \p1 = (npn4.B) in node[anchor=east] at (\x1,\y1) {$T_4$};
                \end{circuitikz}
                \caption{Full-bridge single phase AC/DC converter (identical to the one used in the DC/DC section in \figref{fig:DCDC-4Q-switch})}
                \label{fig:ACDC-4Q-switch}
            \end{figure}
        \end{column}
    \end{columns}
\end{frame}

%%%%%%%%%%%%%%%%%%%%%%%%%%%%%%%%%%%%%%%%%%%%%%%%%%%%%%%%%%%%%
%% Pulse width modulation (PWM) options %%
%%%%%%%%%%%%%%%%%%%%%%%%%%%%%%%%%%%%%%%%%%%%%%%%%%%%%%%%%%%%%
\begin{frame}
    \frametitle{Pulse width modulation (PWM) options}
    \begin{columns}
        \begin{column}{0.5\textwidth}
            \begin{figure}
                \begin{circuitikz}
                    \def\cwidth{1.5}
                    \def\cheight{1}
                    \draw[->] (0,0) to node[above]{$d(t)$} ++(1.5,0) node[adder, anchor = west, name=add1]{};
                    \draw node[ctrlblock, anchor = west, minimum width = \cwidth cm, minimum height = \cheight cm](carrier) at (0,-2.25) {}; 
                    \path (carrier.south west) coordinate (blockBottomLeft);
                    
                    % Triangular signal pattern within block
                    \begin{scope}
                        % Define the number of signal steps
                        \def\signalsteps{6}
                        
                        % Compute step width and height of the triangular pattern
                        \pgfmathsetmacro{\stepwidth}{\cwidth/\signalsteps}
                        \pgfmathsetmacro{\signalheight}{\cheight/(1.3)}

                        % Start drawing the triangular signal
                        \draw[signalblue, thick] 
                    ($(blockBottomLeft) + (0.025, 0.1*\cheight)$) % Starting point with a margin
                    \foreach \x in {1,...,\signalsteps} {
                        -- ($
                            (blockBottomLeft) + 
                            (\x*\stepwidth - 0*\stepwidth, {0.1*\cheight + mod(\x, 2)*\signalheight})
                        $)
                    };
                    \end{scope}
                    \draw[->] (carrier.east) -- (carrier.east -| add1.south) -- node[left]{$c(t)$} (add1.south) node[anchor = north west] {$-$};
                    \draw[->] (add1.east) -- ++(0.5,0) node[ctrlblock, anchor = west, minimum width = \cwidth cm, minimum height = \cheight cm](comp){};
                    
                    % Comperator block 
                    \begin{axis}[at={(comp)}, scale only axis, width = 0.8*\cwidth cm, height = 0.8*\cheight cm, anchor = center, xtick=\empty, ytick={0,1}, axis lines=middle, ymax=1.25, ymin = -1.25, font = \footnotesize, extra y ticks={-1}, extra y tick style = {yticklabel shift = -0.75cm}]
                        \addplot[thick, signalblue] coordinates {(-1,-1) (0,-1) (0,1) (1,1)};
                    \end{axis}
                    \draw[-] (comp.east) -- ++(0.5,0) coordinate (out1);
                    \draw[->] ($(comp.east)!0.5!(out1)$) to [short, *-] ++(0,-2.25) to ++(0.5,0) node[ctrlblock, anchor = west, minimum width = \cheight cm, minimum height = \cheight cm](comp2){-1};
                    \draw[->] (comp2.east) to ++(0.5, 0) node[anchor = west]{$s_2(t)$} coordinate (out2);
                    \draw[->] (out1) to (out1 -| out2) node[anchor = west]{$s_1(t)$};
                \end{circuitikz}
                \caption{PWM with \hl{complementary} switching}
                \label{fig:PWM_complementary}
                \end{figure}
        \end{column}
        \begin{column}{0.5\textwidth}
            \begin{figure}
                \begin{circuitikz}
                    \def\cwidth{1.5}
                    \def\cheight{1}
                    \draw[->] (0,0) to node[above]{$d(t)$} ++(1.5,0) node[adder, anchor = west, name=add1]{};
                    \draw node[ctrlblock, anchor = west, minimum width = \cwidth cm, minimum height = \cheight cm](carrier) at (0,-2.25) {}; 
                    \path (carrier.south west) coordinate (blockBottomLeft);
                    
                    % Triangular signal pattern within block
                    \begin{scope}
                        % Define the number of signal steps
                        \def\signalsteps{6}
                        
                        % Compute step width and height of the triangular pattern
                        \pgfmathsetmacro{\stepwidth}{\cwidth/\signalsteps}
                        \pgfmathsetmacro{\signalheight}{\cheight/(1.3)}

                        % Start drawing the triangular signal
                        \draw[signalblue, thick] 
                    ($(blockBottomLeft) + (0.025, 0.1*\cheight)$) % Starting point with a margin
                    \foreach \x in {1,...,\signalsteps} {
                        -- ($
                            (blockBottomLeft) + 
                            (\x*\stepwidth - 0*\stepwidth, {0.1*\cheight + mod(\x, 2)*\signalheight})
                        $)
                    };
                    \end{scope}
                    \draw[->] (carrier.east) -- (carrier.east -| add1.south) coordinate (c1) -- (add1.south) node[anchor = north west] {$-$};
                    \draw[->] (add1.east) -- ++(2,0) node[ctrlblock, anchor = west, minimum width = \cwidth cm, minimum height = \cheight cm](comp1){};

                    \draw node[ctrlblock, below=0.2cm of comp1, minimum width = \cwidth cm, minimum height = \cheight cm](comp2){};
                    \draw[<-] (comp2.west) -- ++(-1,0) node[adder, anchor = east, name=add2]{};
                    \draw node[crossingshape, name=x1] at (add1.south |- add2.west) {};
                    \draw[-] ($(0,0)!0.5!(add1.west)$) to [short,*-] (\tikztostart |-  x1.west) to [short] (x1.west);
                    \draw[->] (x1.east) to [short] (add2.west) node[anchor = north east] {$-$};
                    \draw[->] (c1) to [short, *-, l=$c(t)$] (c1 -| add2.south) to [short] (add2.south) node[anchor = north west] {$-$};
                    
                    % Comperator block #1
                    \begin{axis}[at={(comp1)}, scale only axis, width = 0.8*\cwidth cm, height = 0.8*\cheight cm, anchor = center, xtick=\empty, ytick={0,1}, axis lines=middle, ymax=1.25, ymin = -1.25, font = \footnotesize, extra y ticks={-1}, extra y tick style = {yticklabel shift = -0.75cm}]
                        \addplot[thick, signalblue] coordinates {(-1,-1) (0,-1) (0,1) (1,1)};
                    \end{axis}
                     % Comperator block #2
                     \begin{axis}[at={(comp2)}, scale only axis, width = 0.8*\cwidth cm, height = 0.8*\cheight cm, anchor = center, xtick=\empty, ytick={0,1}, axis lines=middle, ymax=1.25, ymin = -1.25, font = \footnotesize, extra y ticks={-1}, extra y tick style = {yticklabel shift = -0.75cm}]
                        \addplot[thick, signalblue] coordinates {(-1,-1) (0,-1) (0,1) (1,1)};
                    \end{axis}
                    \draw[->] (comp1.east) -- ++(0.5,0) node[anchor = west]{$s_1(t)$};
                    \draw[->] (comp2.east) -- ++(0.5,0) node[anchor = west]{$s_2(t)$};
                \end{circuitikz}
                \caption{PWM with \hl{interleaved} switching}
                \label{fig:PWM_interleaved}
                \end{figure}
        \end{column}
    \end{columns}
\end{frame}

%%%%%%%%%%%%%%%%%%%%%%%%%%%%%%%%%%%%%%%%%%%%%%%%%%%%%%%%%%%%%
%% PWM example with complementary switching %%
%%%%%%%%%%%%%%%%%%%%%%%%%%%%%%%%%%%%%%%%%%%%%%%%%%%%%%%%%%%%%
\begin{frame}
    \frametitle{PWM example with complementary switching} 
    \begin{figure}
        \begin{tikzpicture}
            \pgfplotsset{table/search path={fig/lec06}}
            \begin{groupplot}[group style={group size=1 by 5, xticklabels at = edge bottom, vertical sep=0.25cm}, height=0.31\textheight, width=0.875\textwidth, xmin=0, xmax=2*pi, grid,clip = false, ymin = -1.1, ymax =1.1, xtick = {0, pi/2, pi, 3/2*pi, 2*pi}, xticklabels = {0,$\nicefrac{1}{2}\pi$,$\pi$, $\nicefrac{3}{2}\pi$, $2\pi$}, ytick = {-1, 0, 1}, yticklabels = {$-1$, $0$, $1$}]

                 % Top plot: duty cycle reference and carrier signal
                \nextgroupplot[ylabel = {$d(t)/c(t)$}, legend pos=north east, legend columns=2]
                \addplot[signalred, thick] table[x=wt, y=d, col sep=comma] {PWM_single_phase_comp_example.csv}; 
                \addplot[signalblue, thick] table[x=wt, y=c, col sep=comma] {PWM_single_phase_comp_example.csv}; 
                \legend{$d(t)$, $c(t)$}

                % top middle plot: individual switching signals
                \nextgroupplot[ylabel = {$s_1(t)$}] 
                \addplot[signalgreen, thick] table[x=wt, y=s1, col sep=comma] {PWM_single_phase_comp_example.csv}; 

                % top middle plot: individual switching signals
                \nextgroupplot[ylabel = {$s_2(t)$}] 
                \addplot[signallavender, thick] table[x=wt, y=s2, col sep=comma] {PWM_single_phase_comp_example.csv}; 

                % bottom middle plot: combined switching signal
                \nextgroupplot[ylabel = {$s(t)$}] 
                \addplot[signalblue, thick] table[x=wt, y=s, col sep=comma] {PWM_single_phase_comp_example.csv}; 

                % bottom plot: approximation error 
                \nextgroupplot[ylabel = {$e(t)$}, xlabel={$\omega t$}, ymax = 0.175, ymin = -0.175, ytick = {-0.15, 0, 0.15}, yticklabels = {$-0.15$, $0$, $0.15$}] 
                \addplot[signalblue, thick] table[x=wt, y=e, col sep=comma] {PWM_single_phase_comp_example.csv}; 
            \end{groupplot}
        \end{tikzpicture}
    \end{figure}
\end{frame}

%%%%%%%%%%%%%%%%%%%%%%%%%%%%%%%%%%%%%%%%%%%%%%%%%%%%%%%%%%%%%
%% PWM example with interleaved switching %%
%%%%%%%%%%%%%%%%%%%%%%%%%%%%%%%%%%%%%%%%%%%%%%%%%%%%%%%%%%%%%
\begin{frame}
    \frametitle{PWM example with interleaved switching} 
    \begin{figure}
        \begin{tikzpicture}
            \pgfplotsset{table/search path={fig/lec06}}
            \begin{groupplot}[group style={group size=1 by 5, xticklabels at = edge bottom, vertical sep=0.25cm}, height=0.31\textheight, width=0.875\textwidth, xmin=0, xmax=2*pi, grid,clip = false, ymin = -1.1, ymax =1.1, xtick = {0, pi/2, pi, 3/2*pi, 2*pi}, xticklabels = {0,$\nicefrac{1}{2}\pi$,$\pi$, $\nicefrac{3}{2}\pi$, $2\pi$}, ytick = {-1, 0, 1}, yticklabels = {$-1$, $0$, $1$}]

                 % Top plot: duty cycle reference and carrier signal
                \nextgroupplot[ylabel = {$d(t)/c(t)$}, legend pos=north east, legend columns=2]
                \addplot[signalred, thick] table[x=wt, y=d, col sep=comma] {PWM_single_phase_int_example.csv}; 
                \addplot[signalblue, thick] table[x=wt, y=c, col sep=comma] {PWM_single_phase_int_example.csv}; 
                \addplot[signalred, thick, dashed] table[x=wt, y expr=-\thisrow{d}, col sep=comma] {PWM_single_phase_int_example.csv};
                \legend{$d(t)$, $c(t)$}

                % top middle plot: individual switching signals
                \nextgroupplot[ylabel = {$s_1(t)$}] 
                \addplot[signalgreen, thick] table[x=wt, y=s1, col sep=comma] {PWM_single_phase_int_example.csv}; 

                % top middle plot: individual switching signals
                \nextgroupplot[ylabel = {$s_2(t)$}] 
                \addplot[signallavender, thick] table[x=wt, y=s2, col sep=comma] {PWM_single_phase_int_example.csv}; 

                % bottom middle plot: combined switching signal
                \nextgroupplot[ylabel = {$s(t)$}] 
                \addplot[signalblue, thick] table[x=wt, y=s, col sep=comma] {PWM_single_phase_int_example.csv}; 

                % bottom plot: approximation error 
                \nextgroupplot[ylabel = {$e(t)$}, xlabel={$\omega t$}, ymax = 0.175, ymin = -0.175, ytick = {-0.15, 0, 0.15}, yticklabels = {$-0.15$, $0$, $0.15$}] 
                \addplot[signalblue, thick] table[x=wt, y=e, col sep=comma] {PWM_single_phase_int_example.csv}; 
            \end{groupplot}
        \end{tikzpicture}
    \end{figure}
\end{frame}

%%%%%%%%%%%%%%%%%%%%%%%%%%%%%%%%%%%%%%%%%%%%%%%%%%%%%%%%%%%%%
%% PWM approximation error analysis %%
%%%%%%%%%%%%%%%%%%%%%%%%%%%%%%%%%%%%%%%%%%%%%%%%%%%%%%%%%%%%%
\begin{frame}
    \frametitle{PWM approximation error analysis} 
    \begin{columns}
        \begin{column}{0.5\textwidth}
            \begin{figure}
                \begin{tikzpicture}[baseline=(current bounding box.center)] % complemntary switching
                    \tikzmath{
                                real \d, \t1;
                                \d = 0.6;
                                \t1 = 0.5*(0.5+\d/2);
                            }
                     \begin{groupplot}[group style={group size=1 by 4, xticklabels at = edge bottom, vertical sep=1em}, 
                         width=0.9\textwidth,
                         height=0.27\textheight,
                         axis x line=middle,
                         axis y line=left,
                         xmin=-0.1, xmax=1.1,
                         ymin=-1.10, ymax=1.10,
                         xtick={0, 1/2, 1},
                         xticklabels={$0$,$\frac{T_\mathrm{s}}{2}$,$T_\mathrm{s}$},
                         ytick={-1,0,1},
                         yticklabels={$-1$,$0$,$1$},
                         grid=both,
                         clip=false,
                         xlabel={$t$},
                         xlabel style={anchor=west}
                         ]
                     \nextgroupplot[ylabel = {$d(t)/c(t)$}, height=0.36\textheight]
                         \addplot[signalblue, thick] coordinates {(-0.1,-0.6) (0,-1) (0.5,1) (1,-1)(1.1,-0.6)};
                         \addplot[domain = -0.1:1.1, samples = 10, signalred, thick] {\d};   
                         \node[anchor=west] at (axis cs:1.1,\d) {$d$};
                         \draw[<->] (axis cs:\t1,-1) -- node[above, fill=white, inner sep=1pt,yshift=2pt]{\footnotesize$\frac{T_\mathrm{s}(2-d)}{2}$} (axis cs:1-\t1,-1);
                         \draw[<->] (axis cs:0,1) -- node[above, fill=white, inner sep=1pt,yshift=2pt]{\footnotesize$\frac{T_\mathrm{s}(1+d)}{4}$} (axis cs:\t1,1);
                         \draw[<->] (axis cs:1-\t1,1) -- node[above, fill=white, inner sep=1pt,yshift=2pt]{\footnotesize$\frac{T_\mathrm{s}(1+d)}{4}$} (axis cs:1,1);
                         \coordinate (a1) at (axis cs:\t1,1);
                         \coordinate (b1) at (axis cs:1-\t1,1);
                        
                        \nextgroupplot[ylabel = {$s_1(t)$}];
                            \addplot[signalgreen, thick] coordinates {(-0.1,1) (\t1,1) (\t1,-1) (1-\t1,-1) (1-\t1,1) (1.1,1)};
            
                        \nextgroupplot[ylabel = {$s_2(t)$}];
                        \addplot[signallavender, thick] coordinates {(-0.1,-1) (\t1,-1) (\t1,1) (1-\t1,1) (1-\t1,-1) (1.1,-1)};
            
                        \nextgroupplot[ylabel = {$u_2(t)$}, , height=0.36\textheight, xticklabel style={below, yshift=-0.75cm}];
                            \addplot[domain = -0.1:1.1, samples = 10, signalblue, thick, dashed, name path = A] {\d}; 
                            \node[anchor=west] at (axis cs:1.1,\d) {$\overline{u}_2$};
                            \addplot[signalblue, thick, name path = B] coordinates {(-0.1,1) (\t1,1) (\t1,-1) (1-\t1,-1) (1-\t1,1) (1.1,1)};
                            \coordinate (a2) at (axis cs:\t1,0);
                            \coordinate (b2) at (axis cs:1-\t1,0);
                            \draw[thin] (axis cs:0.55,-0.4) -- (axis cs:0.7,-0.6) node[right, anchor=west]{\footnotesize$e(t)$};
                            \tikzfillbetween[of=A and B, soft clip={domain=0:1}]{shadecolor, opacity=0.3};
                     \end{groupplot}
                     \begin{scope}[on background layer]
                        \draw [dashed] (a1) -- (a2);
                        \draw [dashed] (b1) -- (b2);
                    \end{scope}
             \end{tikzpicture}
             \vspace{-0.25cm}
             \caption{Pulse pattern for complementary PWM}
            \end{figure}
        \end{column}
        \begin{column}{0.5\textwidth}
            \begin{figure}
                \begin{tikzpicture}[baseline=(current bounding box.center)] % interleaved switching
                    \tikzmath{
                                real \d, \t1;
                                \d = 0.6;
                                \t1 = 0.5*(0.5+\d/2);
                                \t2 = (1-\d)/4;
                            }
                     \begin{groupplot}[group style={group size=1 by 4, xticklabels at = edge bottom, vertical sep=1em}, 
                         width=0.9\textwidth,
                         height=0.27\textheight,
                         axis x line=middle,
                         axis y line=left,
                         xmin=-0.1, xmax=1.1,
                         ymin=-1.10, ymax=1.10,
                         xtick={0, 1/2, 1},
                         xticklabels={$0$,$\frac{T_\mathrm{s}}{2}$,$T_\mathrm{s}$},
                         ytick={-1,0,1},
                         yticklabels={$-1$,$0$,$1$},
                         grid=both,
                         clip=false,
                         xlabel={$t$},
                         xlabel style={anchor=west}
                         ]
                     \nextgroupplot[ylabel = {$d(t)/c(t)$}, height=0.36\textheight]
                         \addplot[signalblue, thick] coordinates {(-0.1,-0.6) (0,-1) (0.5,1) (1,-1)(1.1,-0.6)};
                         \addplot[domain = -0.1:1.1, samples = 10, signalred, thick] {\d};   
                         \addplot[domain = -0.1:1.1, samples = 10, signalred, thick, dashed] {-\d};   
                         \node[anchor=west] at (axis cs:1.1,\d) {$d$};
                         \node[anchor=west] at (axis cs:1.1,-\d) {$-d$};
                         \draw[<->] (axis cs:\t1,-1) -- node[above, fill=white, inner sep=1pt,yshift=2pt]{\footnotesize$\frac{T_\mathrm{s}(2-d)}{2}$} (axis cs:1-\t1,-1);
                         \draw[<->] (axis cs:0,1) -- node[above, fill=white, inner sep=1pt,yshift=2pt]{\footnotesize$\frac{T_\mathrm{s}(1+d)}{4}$} (axis cs:\t1,1);
                         \draw[<->] (axis cs:1-\t1,1) -- node[above, fill=white, inner sep=1pt,yshift=2pt]{\footnotesize$\frac{T_\mathrm{s}(1+d)}{4}$} (axis cs:1,1);
                         \coordinate (a1) at (axis cs:\t1,1);
                         \coordinate (b1) at (axis cs:1-\t1,1);
                        
                        \nextgroupplot[ylabel = {$s_1(t)$}];
                            \addplot[signalgreen, thick] coordinates {(-0.1,1) (\t1,1) (\t1,-1) (1-\t1,-1) (1-\t1,1) (1.1,1)};
            
                        \nextgroupplot[ylabel = {$s_2(t)$}];
                        \addplot[signallavender, thick] coordinates {(-0.1,1) (\t2,1) (\t2,-1) (1-\t2,-1) (1-\t2,1) (1.1,1)};
            
                        \nextgroupplot[ylabel = {$u_2(t)$}, , height=0.36\textheight, xticklabel style={below, yshift=-0.75cm}];
                            \addplot[domain = -0.1:1.1, samples = 10, signalblue, thick, dashed, name path = A] {\d}; 
                            \node[anchor=west] at (axis cs:1.1,\d) {$\overline{u}_2$};
                            \addplot[signalblue, thick, name path = B] coordinates {(-0.1,0) (\t2,0) (\t2,1) (\t1,1) (\t1,0) (1-\t1,0) (1-\t1,1) (1-\t2,1) (1-\t2,0) (1.1,0)};
                            \coordinate (a2) at (axis cs:\t1,0);
                            \coordinate (b2) at (axis cs:1-\t1,0);
                            \tikzfillbetween[of=A and B, soft clip={domain=0:1}]{shadecolor, opacity=0.3};
                            \draw[thin] (axis cs:0.55,0.25) -- (axis cs:0.7,-0.6) node[right, anchor=west]{\footnotesize$e(t)$};
                     \end{groupplot}
                     \begin{scope}[on background layer]
                        \draw [dashed] (a1) -- (a2);
                        \draw [dashed] (b1) -- (b2);
                    \end{scope}
             \end{tikzpicture}
             \vspace{-0.25cm}
             \caption{Pulse pattern for interleaved PWM}
            \end{figure}
        \end{column}
    \end{columns}  
\end{frame}

%%%%%%%%%%%%%%%%%%%%%%%%%%%%%%%%%%%%%%%%%%%%%%%%%%%%%%%%%%%%%
%% PWM approximation error analysis (cont.)%%
%%%%%%%%%%%%%%%%%%%%%%%%%%%%%%%%%%%%%%%%%%%%%%%%%%%%%%%%%%%%%
\begin{frame}
    \frametitle{PWM approximation error analysis (cont.)} 
    To evaluate the error between the reference duty cycle $d(t)$ and the switched output voltage $u_2(t)$, we introduce the following normalized integral difference:
    \begin{equation}
        e(t) = \frac{1}{T_\mathrm{s}}\int_{0}^{t}\left(d(\tau)-s(\tau)\right)\mathrm{d}\tau.
    \end{equation}
    This error can be interpreted as the resulting current ripple assuming a pure inductive load $L$ and constant input voltage $u_1(t)=U_1$:
    \begin{equation}
        \Delta i_2(t) = \frac{T_\mathrm{s}U_1}{2 L } e(t).
    \end{equation}
\end{frame}